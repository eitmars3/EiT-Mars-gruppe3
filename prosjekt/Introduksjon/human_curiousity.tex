\subsection*{What is curiousity?}
\initial{B}efore we begin, it is relevant to take a moment to fully digest the course of human history, and where we are now.
We started humble.
From our beginnings as hunter-gatherers in a nomad lifestyle with no more tools than those fashioned from stone, and no sharing of knowledge save amongst the tribal units.
Given a few thousand years, we've gone from looking at the stars with wonder to looking at the stars with determination.
The human race has reached a stage where we are capable of putting human beings into metal boxes propelled by tiny explosions and launching them out of our planet.
All of this has, at some point, existed as an idea in the mind of a curious human being.
The discoveries in biology that have given us insights into how we are constructed, and the progresses of physics and engineering that have allowed us to build spacefaring craft, all has at some point only been a figment of someone's imagination.

When the Russian cosmonauts first escaped our planets gravity, a new realm of potential discovery was released.
The imagination has been exposed to an embarrassment of riches since then; to the curious human mind, having discovered that the universe is vast and open to our travel, and that the unique conditions of temperature, pressure and matter that have made *us* likely exist elsewhere in the universe.

Can we reach other worlds?
What will we find there?
\textit{Are we alone?}
The thrill of discovery is immense.
Understanding history as we do, we know that a large number of years have passed, a large number of discoveries have been made, and many wise lives have come and gone and contributed to the ladder of knowledge we are currently climbing.
To fully understand our current hunt for extra-terrestrial life, we need a context, and to fully respect it, we must know who contributed, and how.