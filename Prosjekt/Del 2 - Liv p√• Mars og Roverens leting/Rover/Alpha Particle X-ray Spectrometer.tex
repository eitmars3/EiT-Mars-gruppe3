\documentclass[5p]{elsarticle}
\journal{Veileder}	
\usepackage[utf8]{inputenc}
\usepackage[T1]{fontenc} 				
\usepackage[norsk]{babel}				
\usepackage{graphicx}       				
\usepackage{amsmath,amssymb} 				
\usepackage{siunitx}					
	\sisetup{exponent-product = \cdot}      	
 	\sisetup{output-decimal-marker  =  {,}} 	
 	\sisetup{separate-uncertainty = true}   	
\usepackage{booktabs}                     		
\usepackage[font=small,labelfont=bf]{caption}		
\usepackage{minitoc}

\catcode`æ=\active\defæ{\ae}	\catcode`Æ=\active\defÆ{\AE}
\catcode`ø=\active\defø{\o}	\catcode`Ø=\active\defØ{\OO}
\catcode`å=\active\defå{\aa}	\catcode`Å=\active\defÅ{\AA}

\makeatletter
\def\ps@pprintTitle{%
  \let\@oddhead\@empty
  \let\@evenhead\@empty
  \let\@oddfoot\@empty
  \let\@evenfoot\@oddfoot
}
\makeatother

\makeatletter
\setlength{\@fptop}{0pt}
\makeatother

\renewenvironment{abstract}{\global\setbox\absbox=\vbox\bgroup
\hsize=\textwidth\def\baselinestretch{1}%
\noindent\unskip\textbf{Introduksjon}
\par\medskip\noindent\unskip\ignorespaces}
{\egroup}



\setcounter{totalnumber}{5}
\renewcommand{\textfraction}{0.05}
\renewcommand{\topfraction}{0.95}
\renewcommand{\bottomfraction}{0.95}
\renewcommand{\floatpagefraction}{0.35}


\begin{document}

\begin{frontmatter}


\title{Instrumenter}

\author[]{Ingelin Garmann, Simen L. Hegge, Karten Olav Kjensmo, \\ Jonas Sandøy Misund, Martin Nordal \& Anna Solveig Julia Testani\`{e}re}
\address{Norges Teknisk-Naturvitenskapelige Universitet, N-7491 Trondheim, Norway}

\begin{abstract}
Dokument om instrumenter på Curiosity
\end{abstract}

\end{frontmatter}

\section*{Instrumenter}
\subsection*{Alpha Particle X-ray Spectrometer (APXS)}
APXS er et spektrometer som sitter på Curiositys robotarm.
Som sine forfølgere på alle tidligere rovere brukes dette spektrometeret til å identifisere kjemiske elementer i stein og jord.



\section*{Kilder}

\begin{thebibliography}{99}	% Denne referanselisten kan ikke ha flere enn 99 referanser.
	\bibitem{Exoboken}
		Kjetill {\O}degaard
		\emph{Exobiology: A hitch-hiker's guide to alien life}
		Institutt for bioteknologi,
		NTNU,
		2000.
	\bibitem{NASA-rover}
		NASA,
		\emph{Mars Science Laboratory Launch}
		Press kit,
		November 2011.
\end{thebibliography}


\end{document}