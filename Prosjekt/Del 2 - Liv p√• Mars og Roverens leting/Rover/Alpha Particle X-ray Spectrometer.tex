\documentclass[5p]{elsarticle}
\journal{Veileder}	
\usepackage[utf8]{inputenc}
\usepackage[T1]{fontenc} 				
\usepackage[norsk]{babel}				
\usepackage{graphicx}       				
\usepackage{amsmath,amssymb} 				
\usepackage{siunitx}					
	\sisetup{exponent-product = \cdot}      	
 	\sisetup{output-decimal-marker  =  {,}} 	
 	\sisetup{separate-uncertainty = true}   	
\usepackage{booktabs}                     		
\usepackage[font=small,labelfont=bf]{caption}		
\usepackage{minitoc}

\makeatletter
\def\ps@pprintTitle{%
  \let\@oddhead\@empty
  \let\@evenhead\@empty
  \let\@oddfoot\@empty
  \let\@evenfoot\@oddfoot
}
\makeatother

\makeatletter
\setlength{\@fptop}{0pt}
\makeatother

\renewenvironment{abstract}{\global\setbox\absbox=\vbox\bgroup
\hsize=\textwidth\def\baselinestretch{1}%
\noindent\unskip\textbf{Introduction}
\par\medskip\noindent\unskip\ignorespaces}
{\egroup}



\setcounter{totalnumber}{5}
\renewcommand{\textfraction}{0.05}
\renewcommand{\topfraction}{0.95}
\renewcommand{\bottomfraction}{0.95}
\renewcommand{\floatpagefraction}{0.35}


\begin{document}

\begin{frontmatter}


\title{Instruments}

\author[]{Ingelin Garmann, Simen L. Hegge, Karten Olav Kjensmo, \\ Jonas Sandøy Misund, Martin Nordal \& Anna Solveig Julia Testani\`{e}re}
\address{Norges Teknisk-Naturvitenskapelige Universitet, N-7491 Trondheim, Norway}

\begin{abstract}
Document on instruments on the Mars rover Curiosity
\end{abstract}

\end{frontmatter}

\section*{Instruments}
\subsection*{Alpha Particle X-ray Spectrometer (APXS)}
The APXS in a spectrometer sitting on Curiosity's robot arm.
As it's ancestors on the previous rovers the spectrometer is used to identify chemical elements in rock and soil on the Martian surface.

Most of the elements examined are mineral forming, like sodium, magnesium, aluminum, silicon, calcium, iron and sulfur.
APXS can also detect traces of important elements that takes part in different salts; sulfur, chlorine and bromide.
These trace elements can be a sign of former interactions with water.

Etter at APXS har sett på prøvene tar Curiosity videre valg om hvordan typer testen den skal foreta seg videre.
Informasjonen fra APXS sine prøver brukes også til å kartlegge hvordan steinformasjonene i områdene nær roveren kan ha utviklet seg.

Som kilde til alfapartikler og røntgenstråling brukes det radioaktive stoffet curium (Cm$^{244}$).
Dette har en halveringstid på 18.1 år, noe som passer godt med langvarige oppdrag.
Selv etter over syv år vil man ha en kilde uten nevneverdig lavere aktivitet.

Etter at roverene Spirit og Opportunity bruke APXS ble teknologien videreutviklet, og Curiosity har nå blant annet muligheten til å ta raskere og klarere målinger samt jobbe på dagtid.
Målingene kan tas raskere og klarere av to grunner.
Den ene er at Curiositys APXS kan ta målinger betydelig nærmere prøven enn det forgjengerne kunne.
Ellers har Curiosity og omtrent dobbelt så stor menge curium med seg, som øker intensiteten av strålingen.
Årsaken til at APXS bare kunne brukes på nattestid hos Spirit og Opportunity var at instrumentet krever en betraktelig lavere temperatur enn det er på overflaten av Mars på dagtid.
Dette er løst ved å installere en elektrisk kjøler basert på fastfaseteknologi.

Andre forbedringer som er gjort siden Spirit og Opportunity er blant annet programvaren som styrer hvor nære Curiositys APXS kan flyttes prøvene.

Teorien bak APXS bygger på kjerne- og strålingsfysikk.
Når et atom blir truffet av stråing, i form av f.eks. fotoner, alfapartikler, betapartikler og gammapartikler, eksiteres dette.
Det betyr at elektoner som er bundet til kjernen løftes til et høyere energinivå eller frigis.
Deretter vil et nytt elektron oppta plassen til det gamle, og atomet vil gi fra seg stråling.
Strålingen som utgis ved eksitering fra alfapartikler består av fotoner med en energi over 5-10 k$eV$; altså røntgenstråling.

Atomer med små kjerner vil ikke slippe ut fotoner med nok energi til å gi røntgenstråling.
Det fører til at for eksempel oksygen vanligvis ikke kan oppdages med røntgen.
Dette er et problem dersom man ønsker å bruke APXS til å oppdage for eksempel vann i mineraler på Mars' overflate.
En metode som kalles "scatter peak method" har blitt utviklet til å likevel kunne finne opp til 20\% vann i slike mineraler.
Dette ble blant annet gjort med salte jordprøver tatt av Spirit i Gusevkrateret.

Opp gjennom historien har APXS gitt mye informasjon om forholdene og sammensetningen av mineralene på Mars.
Dette er nok en av grunnene til at det fortsatt jobbes med å utvikle forbedringer av instrumentet, og at det har vært med på samtlige rovere sendt til den røde planeten.

\section*{Kilder}

\begin{thebibliography}{99}	% Denne referanselisten kan ikke ha flere enn 99 referanser.
	\bibitem{Exoboken}
		Kjetill Ødegaard
		\emph{Exobiology: A hitch-hiker's guide to alien life}
		Institutt for bioteknologi,
		NTNU,
		2000.
	\bibitem{NASA-rover}
		NASA,
		\emph{Mars Science Laboratory Launch}
		Press kit,
		November 2011.
\end{thebibliography}


\end{document}