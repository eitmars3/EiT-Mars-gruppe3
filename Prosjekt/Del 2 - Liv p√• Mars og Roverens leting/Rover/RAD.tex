(!!!no latex formatting yet!!!)

One essential objective for Curiosity's mission is to study the conditions for future human journeys to Mars, of which radiation is feared to be a considerable constraint.
 Curiosity carries its Radiation Assessment Detector (RAD) in order to map what kinds of radiation there are and what the levels are.
 RAD points upwards in a 65 degree field-of-view to detect radiation from outer space, both particle radiation and electromagnetic waves.
 In application where the purpose is to protect humans from radiation, what is also interesting is how much of the radiation that is ionising.
 Such radiation has enough energy to destroy chemical bindings, for example DNA molecules, events that in repeated occasions can lead to cancer.
 RAD can for instance distinguish between such ionizing radiation and harmless radiation.
 RAD has in fact been in use during the journey to Mars as well, on which a human would also have been exposed to the radiation.

Source:
"Radiation Assessment Detector (RAD)"
http://msl-scicorner.jpl.nasa.gov/Instruments/RAD/ (visited: 2015-03-24)