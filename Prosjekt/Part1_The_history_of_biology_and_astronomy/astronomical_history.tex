\subsection*{Egyptians: Astronomy as a tool for agriculture and belief}
\initial{S}ome celestial bodies like the Sun and the Moon were documented already for five thousands years ago! Indeed, Astronomy being a science of observation, Egyptians didn't need more than looking in the sky to discover bodies. Although more difficult to see, five planets could already be seen by the naked eye due to their size and brightness: Mercury, Venus, Jupiter, Mars and Saturn. However, we had to wait until Copernicus in 1543 to categorize Mars as a planet! Egyptians' main interests for the sky were mainly: agricultural and religious. The Egyptians study the positions and alignments of stars to build pyramids and the size of the moon determined periods of plantation. For the same purpose, they already used the 365 days calendar. Religion played also a great part in ancient astronomy as gods were view as constellations.

\subsection*{Ancient Greeks, Arabs and Indians: The age of theories}
\initial{I}t is not until the Greek period, starting around 600 BC, that humanity tried to really understand and explain Astronomy. To start with, they thought that earth was flat, surrounded by water and that the stars were emerging from it. Plato, followed by Ptolemy and Aristotle, were the pioneers of the Geocentric model: the sun, the moon and the planets move in harmony around earth, the center. Hundred years later, another Greek, Samos, proposed the opposite theory of Heliocentrism: the sun is now the center of the universe. Unfortunately, this theory, like many others, suffered the lack of communication and translation of the text between Greeks, Arabs and Indians, to stay alive. It can be quite frustrating for example to see that, the Indian astronomer,Aryabhata, already had some ideas about gravitation, eccentric elliptical, that the light was the reflection of the sum and the radii of some bodies without being noticed. One might wonder, how much we would know if the lack of communication during this period made the process of astronomy knowledge much longer.

\subsection*{European Renaissance: The age of reason. Experimentation, better technologies made the foundations of what we know}
\initial{W}e had to wait until the Renaissance for the rediscoveries of Ancient Greek papers and research and their translation. Scientist as Copernicus, Galileo and Kepler defined to the world the model of Heliocentric. It is definitely the age of reason and scientific truth. For example, astronomy and astrology are separated into two different fields. Technologies and communications became more advanced and enable physician to search deeper and understand. In 1609, an Italian physicist and astronomer named Galileo became the first person to point a telescope skyward. Although that telescope was small and the images fuzzy, Galileo was able to make out mountains and craters on the moon. With advancing technology, astronomers discovered many faint stars and the calculation of stellar distances. Telescope also Newton. With the help of new technologies and the writing of ancient Greek, the scientist of the renaissance period were able to make very accurate calculation on stars, planets and define the basic of astronomy as we know it today, for example the heliocentrism and the solar system, our planets and knowledge.

\subsection*{Space Age: Not only studying Astronomy but also Exploring!}
\initial{I}n 1957, The Soviet Union launched the first sputnik, a spacecraft placed in orbit around Earth, marking the beginning of space exploration. We are not only interest in studying anymore but also of exploring. The space age represents an effort as scientific as political. Indeed, this is happening during the Cold War and the Soviet Union and the USA race in different domains, including astronomy. This space race focus particularly on discovery by humans and machines in the solar system and the development of technology. Rockets were one of the most obvious forms of Space Age technology. the Creation of NASA and one of the highlights of the Space Age was definitely the Apollo program. The most famous of the Apollo aircraft is Apollo 11, which was the craft carrying Commander Neil Armstrong and his fellow astronauts Michael Collins and Buzz Aldrin to the Moon. On that mission, Armstrong and Aldrin were the first humans to land and walk on the Moon.