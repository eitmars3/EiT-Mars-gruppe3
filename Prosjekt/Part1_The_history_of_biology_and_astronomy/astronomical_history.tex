\subsection*{Egyptians: Astronomy as a tool for agriculture and belief}
\initial{S}ome celestial bodies like the Sun and the Moon were documented already for five thousands years ago!
Indeed, Astronomy being a science of observation, Egyptians didn't need more than looking in the sky to discover bodies.
Although more difficult to see, five planets could already be seen by the naked eye due to their size and brightness:
Mercury, Venus, Jupiter, Mars and Saturn.
However, we had to wait until Copernicus in 1543 to categorize Mars as a planet!
Egyptians' main interests for the sky were mainly: agricultural and religious.
The Egyptians study the positions and alignments of stars to build pyramids and the size of the moon determined periods of plantation.
For the same purpose, they already used the 365 days calendar.
Religion played also a great part in ancient astronomy as gods were view as constellations.

\subsection*{Ancient Greeks, Arabs and Indians: The age of theories}
\initial{I}t is not until the Greek period, starting around 600 BC, that humanity tried to really understand and explain astronomy. 
To start with, they thought that Earth was flat, surrounded by water and that the stars were emerging from it. 
Plato, followed by Ptolemy and Aristotle, were the pioneers of the Geocentric model: the sun, the moon and the planets move in harmony around Earth, the center. 
A hundred years later, another Greek, Samos, proposed the opposite theory called Heliocentrism: the sun is now the center of the universe. 
How fascinating to see how much they actually discovered at that time!  
Without tools such as telescope, without communication within the community of scientist around the world, Greeks, Arabs and Indians still made great assumptions on how our universe worked. 
For example, the Indian astronomer, Aryabatha, already had not only some ideas about gravitation but also that the movement of planets around the sun were elliptical and not circular. 
He even expressed the radii of the orbits for nine planets! 
One might wonder, how much we would know in the present time if ancient astronomers had the tools of communication we have today..?

\subsection*{European Renaissance: The age of reason and experimentation}
\initial{W}ith the help of new technologies and the rediscoveries and translation of the writing of the ancient Greek, the scientists of the renaissance period, in the 1600, were able to make very accurate calculations of the stars, planets and define and prove the basic of astronomy as we know it today.  
Scientists such as Copernicus, Galileo and Kepler defined to the world the model of Heliocentric. 
It is definitely the age of reason and scientific truth. 
We can notice for example, that it is during that time that astronomy and astrology are separated into two different fields. 
Technologies and communications became more advanced and enabled physicians to search deeper and with more accuracy. 
In 1609, the Italian astronomer Galileo was one of the first person to point a telescope skyward. 
Although that telescope was far from perfected, Galileo was able to see that the moon surface was made of craters and mountains instead of being smooth as we thought. 
As theirs views expanded dramatically by the telescope, astronomers were able to discover also many stars and to calculate stellar distances. 

\subsection*{Space Age: Not only studying Astronomy but also Exploring!}
\initial{I}n 1957, The Soviet Union launched the first sputnik, a spacecraft placed in orbit around Earth, marking the beginning of space exploration.
We are not only interest in studying anymore but also of exploring.
The space age represents an effort as scientific as political.
Indeed, this is happening during the Cold War and the Soviet Union and the USA race in different domains, including astronomy.
This space race focus particularly on discovery by humans and machines in the solar system and the development of technology.
Rockets were one of the most obvious forms of Space Age technology.
The Creation of NASA and one of the highlights of the Space Age was definitely the Apollo program. The most famous of the Apollo aircraft is Apollo 11, which was the craft carrying Commander Neil Armstrong and his fellow astronauts Michael Collins and Buzz Aldrin to the Moon.
On that mission, Armstrong and Aldrin were the first humans to land and walk on the Moon.