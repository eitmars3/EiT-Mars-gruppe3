



Some celestial bodies like the Sun and the Moon were documented already five thousand years ago! Indeed, astronomy being a science of observation, Egyptians did not need more than looking into the sky to discover bodies. Although more difficult to see, five planets could already be seen by the naked eye due to their sizeS and brightnessES: Mercury, Venus, Jupiter, Mars and Saturn. However, we had to wait until 1543 for Copernicus to categorize Mars as a planet! Egyptians' main interests for the sky were mainly agricultural and religious. The Egyptians studied the positions and alignments of the stars to build pyramids, and the size of the moon determined periods of cultivation. For the same purpose, they established the 365 days calendar based on the redundance of agricultural cycles. Religion also played a great part in ancient astronomy as gods were viewed as constellations. 


It is not until the Greek period, starting around 600 BC, that humanity tried to really understand and explain astronomy. To start with, they thought that Earth was flat, surrounded by water and that the stars were emerging from it. Plato, followed by Ptolemy and Aristotle, were the pioneers of the Geocentric model: the sun, the moon and the planets move in harmony around Earth, the center. A hundred years later, another Greek, Samos, proposed the opposite theory called Heliocentrism: the sun is now the center of the universe. Unfortunately, this theory, like many others, suffered the lack of communication and translation of the texts between Greeks, Arabs and Indians, to stay alive. On that subject, it can be quite frustrating for example to see that, the Indian astronomer, Aryabatha, already had some ideas about gravitation, eccentric elliptical orbits, that the light was the reflection of the sun and the radii of some bodies without being noticed. One might wonder, how much we would know today if ancient astronomers had the tools of communication we have today..? \\



We had to wait until the 16th century, a period called the Renaissance, for the rediscoveries of Ancient Greek papers and research and their translation. Scientists such as Copernicus, Galileo and Kepler defined to the world the model of Heliocentric. It is definitely the age of reason and scientific truth. For example, astronomy and astrology are separated into two different fields. Technologies and communications became more advanced and enabled physicians to search deeper and to understand more. In 1609, an Italian physicist and astronomer named Galileo became the first person to point a telescope skyward. Although that telescope was small and the images fuzzy, Galileo was able to make out mountains and craters on the moon. With advancing technology, astronomers discovered many faint stars and the calculation of stellar distances. Telescope also Newton. With the help of new technologies and the writing of ancient Greek, the scientists of the renaissance period were able to make very accurate calculations of the stars, planets and define the basic of astronomy as we know it today, for example the heliocentrism and the solar system, our planets and knowledge.

In 1957, The Soviet Union launched the first sputnik, a spacecraft placed in orbit around Earth, marking the beginning of space exploration. We are not only interested in theoretical studies, but also exploring. The space age represents an effort both scientific motivated by political issues. Indeed, that happened during the Cold War with the race between the Soviet Union and the USA in different domains, including astronomy. This space race focused particularly on discovery by humans and machines in the solar system and the development of technology. Rockets were one of the most obvious forms of Space Age technology. We can also notice the creation of NASA in 1958 and the Apollo program as being highlights of that period. Indeed, the most famous of the Apollo aircrafts, Apollo 11, was the first craft landing Neil Armstrong and his fellow astronauts on the Moon in 1969.