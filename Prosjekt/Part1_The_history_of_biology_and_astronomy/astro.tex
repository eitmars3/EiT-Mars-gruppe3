



Some celestial bodies like the Sun and the Moon were documented already five thousand years ago! 
Indeed, astronomy being a science of observation, Egyptians did not need more than looking into the sky to discover bodies. 
Although more difficult to see, five planets could already be seen by the naked eye due to their sizeS and brightnessES: Mercury, Venus, Jupiter, Mars and Saturn. 
However, we had to wait until 1543 for Copernicus to categorize Mars as a planet! 
Egyptians' main interests for the sky were mainly agricultural and religious. 
The Egyptians studied the positions and alignments of the stars to build pyramids, and the size of the moon determined periods of cultivation. 
For the same purpose, they established the 365 days calendar based on the redundance of agricultural cycles. 
Religion also played a great part in ancient astronomy as gods were viewed as constellations. 


It is not until the Greek period, starting around 600 BC, that humanity tried to really understand and explain astronomy. 
To start with, they thought that Earth was flat, surrounded by water and that the stars were emerging from it. 
Plato, followed by Ptolemy and Aristotle, were the pioneers of the Geocentric model: the sun, the moon and the planets move in harmony around Earth, the center. 
A hundred years later, another Greek, Samos, proposed the opposite theory called Heliocentrism: the sun is now the center of the universe. 
How fascinating to see how much they actually discovered at that time!  
Without tools such as telescope, without communication within the community of scientist around the world, Greeks, Arabs and Indians still made great assumptions on how our universe worked. 
For example, the Indian astronomer, Aryabatha, already had not only some ideas about gravitation but also that the movement of planets around the sun were elliptical and not circular. He even expressed the radii of the orbits for nine planets! 
One might wonder, how much we would know in the present time if ancient astronomers had the tools of communication we have today..? \\



With the help of new technologies and the rediscoveries and translation of the writing of the ancient Greek, the scientists of the renaissance period, in the 1600, were able to make very accurate calculations of the stars, planets and define and prove the basic of astronomy as we know it today.  
Scientists such as Copernicus, Galileo and Kepler defined to the world the model of Heliocentric. 
It is definitely the age of reason and scientific truth. 
We can notice for example, that it is during that time that astronomy and astrology are separated into two different fields. 
Technologies and communications became more advanced and enabled physicians to search deeper and with more accuracy. 
In 1609, the Italian astronomer Galileo was one of the first person to point a telescope skyward. 
Although that telescope was far from perfected, Galileo was able to see that the moon surface was made of craters and mountains instead of being smooth as we thought. 
As theirs views expanded dramatically by the telescope, astronomers were able to discover also many stars and to calculate stellar distances.

In 1957, The Soviet Union launched the first sputnik, a spacecraft placed in orbit around Earth, marking the beginning of space exploration. 
We are not only interested in theoretical studies, but also exploring. 
The space age represents an effort both scientific motivated by political issues. 
Indeed, that happened during the Cold War with the race between the Soviet Union and the USA in different domains, including astronomy. 
This space race focused particularly on discovery by humans and machines in the solar system and the development of technology. 
Rockets were one of the most obvious forms of Space Age technology. 
We can also notice the creation of NASA in 1958 and the Apollo program as being highlights of that period. 
Indeed, the most famous of the Apollo aircrafts, Apollo 11, was the first craft landing Neil Armstrong and his fellow astronauts on the Moon in 1969.



















