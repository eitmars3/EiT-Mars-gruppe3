\section{Kriterier for liv}

Å definere et begrep som de fleste av oss garantert mener de vet hva betyr, viser seg å ikke være så trivielt. Begrepet det siktes til er så allment kjent og generelt som ordet liv. Faktisk finnes det ingen universell definisjon på hva liv faktisk er. Isteden er det etablert en rekke kriterier som man mener må tilfredstilles for å kunne kalles liv. Hver for seg vil, og sikkert også alle tatt i betrakning, vil man kunne finne fenomener og maskiner som for Ola Nordmann ikke vil betraktes som liv. Dette åpner dermed for stor debatt og mangfoldige tolkninger av hva et liv innebefatter. 

Det første typiske kriteriet for at noe er levende er at det kan vokse. Vekst tolkes i denne sammenheng som å øke i størrelse i motsetning til personlig, mental vekst eller økonomisk vekst. For pattedyr ser man dette i form av fysisk utvikling fra fødsel, via barndom, til ungdom og voksen, frem til død. Insekter gjennomgår stadier kalt egg, larve og puppe frem mot voksent fullutviklet form. Begge disse tilfellene er allment karakterisert som liv, men det finnes også eksempler på uorganiske fenomener som gjennomgår vekst. Et klassisk eksempel er krystallvekst der faste formasjoner vokser frem i en klart organisert struktur. Jordoverflaten er et annet uorganisk fenomen som vokser. I grensene mellom såkalte plater, en inndeling av jordoverflaten, vil ny overflate dannes ved at magma strømmer opp fra jordens indre som lava og størkner til ny berggrunn. Andre steder forsvinner jordoverflaten ved at en plate subduseres altså blir presset under en annen plate og blir en del av jordens indre magma. På denne måten har også jordoverflaten et forløp på linje med et livsforløp ved at det oppstår, vokser og til slutt forsvinner. Det er dermed tydelig at vekst i seg selv ikke er et avgrenset nok kriterium for å beskrive liv slik vi mennesker ser på det i den dag i dag. 

Det neste kriteriet vil utelukke jordoverflaten og krystaller som liv. Det påstår at alt levende må kunne bedrive kjemiske reaksjoner betegnet med metabolisme eller stoffskifte. Dette innebærer at en levende organisme må kunne bygge opp nye, komplekse molekyler i en prosess kalt anabolisme, og samtidig kunne bryte ned komplekse molekyler til enkle forbindelser kalt katabolisme. Hos mennesket er resultatet av slike reaksjoner for eksempel at muskler bygges opp og at mat blir nedbrutt til næringsstoffer. På den annen side vil også en reaktor kunne opprettholde reaksjoner hvor kjemiske forbindelser blir dannet eller brutt ned.

Det kanskje mest innlysende kriteriet en levende organisme må ha er evnen til å reprodusere seg selv. Dyr og mennesker får avkom med forskjellige gener enn opphavet. Dette kalles kjønnet formering og krever sammensmelting av to kjønnsceller. Mindre komplekse former for liv reproduserer seg selv for eksempel ved å dele seg i to slik som bakterier, ved knoppdannelse slik som hos de fleste planter eller ved sporedannelse hos sporedyr. Ved slik ukjønnet formering vil avkom og opphav være genetisk identisk. Et virus derimot kan ikke reprodusere seg selv uten hjelp fra en vertscelle og et muldyr eller mulesel, avkommet til en hest og et esel, er som oftest steril. For ikke å snakke om mennesker, dyr og planter som er enten medfødt eller valgfritt sterile. Tekstdokumenter kan lett reproduseres i en kopimaskin, og arvestoff har blitt reprodusert med bioteknologiens PCR-maskiner. Det fremkommer altså at ting vi ser på som levende, en sterilisert katt for eksempel, ikke kan reprodusere seg selv, mens ikke levende papir kan reproduseres i mangfoldighet i en kopimaskin. Kriteriet i seg selv holder dermed ikke som definisjon på liv.

I dag er det av svært mange enighet om at livet har utviklet seg gjennom evolusjon. Artenes opprinnelse (1959) ble utgitt av Charles Darwin og tar for seg hvordan livet har utviklet seg på bakgrunn av prinsippet om at den sterkeste overlever (Survival of the Fittest). Dette har gjort seg gjeldene i arvestoff ved at de best tilpassede vil overleve, mens individer og arter med visse ulemper vil dø ut og deres arvestoff gå tapt. På en datamaskin kan en slik algoritme lett implementeres. Et eksempel er John Conway's Game of Life som illustrerer naturlig seleksjon ved å definere et sett med fire regler som avgjør liv, død og reproduksjon. 





Kilder:
https://snl.no/platetektonikk \[25.02.15 kl 13.00\]
https://snl.no/formering\%2Fbiologi