\section{Defining life - 5 criteria}
The seemingly trivial terms life and being alive have meanings most of us think we understand.
Surprisingly, an exact definition of life does not exist.
Instead there has been established a list of criteria we think need to be fulfilled in order to be called life.
Such a list is given in Kjetill Østgaards \textit{Exobiologi - A hitch-hikers's guide to alien life} \cite{Exoboken} and those criteria are the basis for this section.
Considered separately, and possibly also as a whole, these criteria does not exclude examples of phenomena and machines we would not consider to be alive.
The debate concerning what life really includes is therefore open to all kinds of meta discussions and creative thinking.

\subsection{Growth}
The first typical criterion defines life as something that grows.
Growth is here referring to physical growth leading to increased size, and not personal or mental growth.
For mammals, the growth cycle starts at birth as an infant, then proceeds through the stages of youth and adulthood before ending up as an elderly individual until death.
Insects proceed through stages called egg, larvae and pupae before reaching the fully developed form of adulthood.
Both these examples are undoubtedly forms of life as we know it, but examples of inorganic phenomena subject to growth are also common.
The classical example is crystal growth describing the formation of strictly organized solid lattices.
Another example is the surface of Earth itself.
In plate tectonic theory \cite{tectonic}, the cycle of the Earths crust can be explained by colliding and parting plates.
In a boundary zone where two plates move apart from each other, magma from inner Earth will erupt from the fissure as lava and form new crust.
In a collision zone a subduction will occur if one plate is forced underneath another becoming part of the inner mantle.
This growth cycle of course spans over a much longer timescale than that of a mammal, but so does a mammals cycle compared to most insects.
It is apparent that this criterion alone does not describe life as we know it.

\subsection{Metabolism}
Next criterion will exclude both the crust and crystals as life forms.
It claims that a living organism must be able to support chemical reactions called metabolism.
This means that life must be able to generate new, complex molecules in a process called anabolism and to break down complex molecules into simpler ones in a process called catabolism.
In mammals, the result from anabolism is physical growth and muscle development from nutrients retrieved in the catabolism of food.
On the contrary, a chemical reactor will be able to maintain both destructive and productive reactions without actually being alive.

\subsection{Reproduction}
Perhaps the most commonly known criterion is that life needs the ability to reproduce itself.
Humans and animals produce offspring containing a different set of genes than their parents.
This process is called sexual reproduction, requiring the fusion of two sex cells \cite{reprod}.
Less complex forms of life can reproduce by simply dividing into two as some bacteria, by pollination as with most plants or by spore formation as with fungi and protozoa.
Such asexual reproductive processes will result in an offspring containing identical genetic information as the origin.
A virus is not able to reproduce by itself and is dependent on its host cell in order to survive.
Thus a lot of people don't consider viruses to be life.
An even more extreme example is the offspring of a donkey and a horse.
The mule is a sterile animal, even so, most people would recognize a mule as a living organism regardless of this fact. 
Not to mention animals and humans that are born, or surgically, sterile.
Documents are easily reproduced in a copy machine, and biotechnologists have succeeded reproducing DNA in advanced PCR-machines.
The science of cloning is also widely developed and has succeeded duplicating a living organism without the methods of traditional reproduction.
In other words, the criterion of reproduction would exclude a sterile cat as being alive, but may include inorganic phenomena that are possible to reproduce by itself or others.

\subsection{Evolution}
In 1859, Charles Darwin introduced the well known theory of evolution in a book called On the Origin of Species \cite{Darwin}.
The basic idea of biological evolution is that all life on Earth share a common ancestor!
In other words, all species roaming the Earth today are very distant cousins.
As a criterion for life, this means that all living species have a placement, according to their inherited characteristics, on a family tree.
A family tree is a representation of the evolutionary relationship among species.
This theory states that life has evolved based on a principle called Survival of the Fittest.
This has resulted in the survival of the species, and the genes of the species, which were best adapted to their environment. 

On a computer, such an algorithm can easily be implemented.
The best known example is John Conway's Game of Life \cite{Conway} which illustrates natural selection by defining four simple rules determining if a cell will live, die and/or reproduce.
The rules determining the future of one cell are all concerned with its neighbours and environment:
\begin{enumerate}
\item Any live cell with fewer than two live neighbours dies, as if caused by under-population.
\item Any live cell with two or three live neighbours lives on to the next generation.
\item Any live cell with more than three live neighbours dies, as if by overcrowding.
\item Any dead cell with exactly three live neighbours becomes a live cell, as if by reproduction.
\end{enumerate}
Applying these rules to all cells in a grid will lead to changes in the total amount of cells and their position for the next generation.
After applying the rules for a number of generations, the grid will look very different, but is in fact a direct outcome of the initial condition.
Evolution has taken place!

\subsection{Motion}
The last criterion that is being considered is the phenomenon of motion.
All living organisms should be able to move and often as a response to different stimuli.
Mammals, fish and birds have either legs, fins or wings making them capable of moving their entire body.
The motion can be in an attempt to escape as a response to a threat, in chase for food or water in order to survive or simply just in order to transport itself to somewhere.
In almost every activity a human can perform, a specific motion is required.
Without being able to move, most of us would not feel very much alive.
Plants and vegetation are not able to move their entire body, but in order to effectively trap sunlight, plants can turn their leaves towards the sun.
A lot of non-living machines or vehicles are also able to move, and more and more can move on autopilot.
There also exist solar panels that can trace the sun much like a plant, and robots that can walk and perform tasks much like a human.
Whereas some living things, like coral reefs, do not move at all.
If motion was the only criterion to be fulfilled, life would include a much wider spectrum of phenomena.
Expressions like "This storm seems to come to life with its motion, power and shape'' and "Tales come to life in the forest" would all of a sudden have a literal interpretation. 

The past section presented and reflected on the criteria creating a definition of life.
Hopefully it became evident that a simple, straight forward definition is hard to formulate considering that the term \textit{life} must include, yet also, exclude so many phenomena.
Although there are tons of non-living examples on each of the presented criteria, the combination of properties concerning growth, metabolism, reproduction, evolution and motion are a valid basis for living creatures.
That means, of course, living creatures as we know them.
If these criteria also applies to extraterrestrial life, only the future can answer.
Perhaps completely new lifeforms, extremely different from all earthly life, will be discovered.
With that in mind, it is an advantage that the definition of life is not carved in stone.