\begin{tcolorbox}[colback=green!5,colframe=green!40!black,title=Curiosity: 10 facts about the rover]

\textbf{Purpose:} To investigate the climate and geology on Mars, and to find evidences for the presence of water.

\textbf{Travel:} From November 26th 2011 at Cape Canaveral Air Force to 6th August 2012 at the Gale Crater.

\textbf{Mission duration:} One Martian year, that is 98 weeks.

\textbf{Mass/size:} The mass is 900kg. The dimensions are 3m x 2,8m x 2,1m, about the size of a car.

\textbf{Movement:} The rover can travel by an average of 30m per hour.

\textbf{Instruments:} It has 17 cameras, most of them are used to drive and navigate.

\textbf{Energy source:} The onboard nuclear power plant carries plutonium-238. From the material's decay it can produce 125W of electricity.

\textbf{Computers:} The two identical computers are specially built to tolerate the radiation that occur on Mars. They have only 256MB of RAM and 2GB of flash memory each.

\textbf{Communication:} The rover uses its three antennas to communicate in the UHF band. Direct communication to the Earth takes about 7 minutes. The rover can also communicate via Mars satellites.

\textbf{Cost:} The overall cost was 2,5 billion USD, where 1,8 billion of it was used in context of freighting the rover to Mars.

\end{tcolorbox}