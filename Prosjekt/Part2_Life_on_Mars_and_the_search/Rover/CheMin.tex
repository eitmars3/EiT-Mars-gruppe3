\subsection{Curiosity's Chemistry \& Mineralogy (CheMin)}
\initial{C}uriosity is not only able to study the very surface of Mars.
With an arm attached, it can drill into the soil to excavate powdered samples from the ground.
These samples are studied with Curiosity's Chemistry \& Mineralogy (CheMin) instrument.
CheMin's main task is to conduct powder X-ray diffraction to study what minerals Martian soil consists of.
X-rays are sent through the sample powder so that these rays will create a diffraction pattern depending on each mineral component in the sample.
The diffracted rays are then captured by an X-ray sensitive 600x582 charge coupled device (CCD), that is a camera which operates with X-ray wave lengths.
The CCD may read, erase and recharge, take individual images in other words, 1000 or more times for each experiment to ensure reliable results.
Each exposure takes from 5 to 30 seconds, resulting in a 10 hour duration for each experiment. \cite{CheMin}