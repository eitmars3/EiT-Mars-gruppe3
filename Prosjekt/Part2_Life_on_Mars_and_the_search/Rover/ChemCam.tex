\subsection*{ChemCam}
\initial{A}s one of Curiosity's tasks is to analyse the composition of the Martian surface, it carries an onboard laboratory called Chemistry \& Camera (ChemCam).

The ChemCam is a composition of two separate instruments: a Laser-Induced Breakdown Spectrometer (LIBS) and a Remote Micro-Imager (RMI).

First the LIBS will use its 1067nm laser pulses to hit rocks, which small amounts of the targeted rock will reach temperatures enough for it to vaporise into plasma.
Because of the high temperature, the plasma will also emit light, telling what the rock consists of.
The emitted light will always be a seamless composition of all wave lengths, except that the presence of specific elements will leave characteristic marks where the light is gone at specific wave lengths.
These marks, typically illustrated as black stripes onto the spectrum of visible light, gives the glowing object its absorption spectrum.
The RMI is a combined microscope and camera able to focus from 1m to 7m away.
It will then measure what light the plasma has emitted, obtaining an absorption spectrum.
With the ability to distinguish between 6144 different wave lengths from 240nm to 850nm, the RMI can recognise most light elements. \cite{ChemCam}