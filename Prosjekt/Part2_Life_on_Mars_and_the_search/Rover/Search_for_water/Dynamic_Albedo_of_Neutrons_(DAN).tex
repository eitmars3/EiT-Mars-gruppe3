\documentclass[5p]{elsarticle}
\journal{Veileder}	
\usepackage[utf8]{inputenc}
\usepackage[T1]{fontenc} 				
\usepackage[norsk]{babel}				
\usepackage{graphicx}       				
\usepackage{amsmath,amssymb} 				
\usepackage{siunitx}					
	\sisetup{exponent-product = \cdot}      	
 	\sisetup{output-decimal-marker  =  {,}} 	
 	\sisetup{separate-uncertainty = true}   	
\usepackage{booktabs}                     		
\usepackage[font=small,labelfont=bf]{caption}		
\usepackage{minitoc}

\makeatletter
\def\ps@pprintTitle{%
  \let\@oddhead\@empty
  \let\@evenhead\@empty
  \let\@oddfoot\@empty
  \let\@evenfoot\@oddfoot
}
\makeatother

\makeatletter
\setlength{\@fptop}{0pt}
\makeatother

\renewenvironment{abstract}{\global\setbox\absbox=\vbox\bgroup
\hsize=\textwidth\def\baselinestretch{1}%
\noindent\unskip\textbf{Introduction}
\par\medskip\noindent\unskip\ignorespaces}
{\egroup}



\setcounter{totalnumber}{5}
\renewcommand{\textfraction}{0.05}
\renewcommand{\topfraction}{0.95}
\renewcommand{\bottomfraction}{0.95}
\renewcommand{\floatpagefraction}{0.35}


\begin{document}

\begin{frontmatter}


\title{Instruments}

\author[]{Ingelin Garmann, Simen L. Hegge, Karsten Olav Kjensmo, \\ Jonas Sandøy Misund, Martin Nordal \& Anna Solveig Julia Testani\`{e}re}
\address{Norges Teknisk-Naturvitenskapelige Universitet, N-7491 Trondheim, Norway}

\begin{abstract}
Document on instruments on the Mars rover Curiosity
\end{abstract}

\end{frontmatter}

\section*{Instruments}
\subsection*{Dynamic Albedo of Neutrons (DAN)}
The Dynamic Albedo of Neutrons is an instrument on Curiosity used for the search of hydrogen in minerals, or even in ice form beneath the Martian surface.
Although it is unlikely that there is any ice directly beneath the surface of the Gale Crater landing site.
The instrument can detect hydrogen up to 50 centimetres below the surface.

Hydrogen detected by DAN can be signs of former water that at the time got bound in crystals.
These crystals are called hydrated minerals, and are for example rock containing traces of hydrogen in their structure.

\section*{Sources}

\begin{thebibliography}{99}	% Denne referanselisten kan ikke ha flere enn 99 referanser.
	\bibitem{Exoboken}
		Kjetill Ødegaard
		\emph{Exobiology: A hitch-hiker's guide to alien life}
		Institutt for bioteknologi,
		NTNU,
		2000.
	\bibitem{NASA-rover}
		NASA,
		\emph{Mars Science Laboratory Launch}
		Press kit,
		November 2011.
\end{thebibliography}


\end{document}