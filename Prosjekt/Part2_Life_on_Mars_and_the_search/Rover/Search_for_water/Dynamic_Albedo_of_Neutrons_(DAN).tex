\subsection*{Dynamic Albedo of Neutrons (DAN)}
\initial{T}he Dynamic Albedo of Neutrons is an instrument on Curiosity used for the search of hydrogen in minerals or even in ice form beneath the Martian surface.
Although it is unlikely that there is any ice directly beneath the surface of the Gale Crater landing site.
The instrument can detect hydrogen up to 50 centimetres below the surface, by shooting neutrons into the ground and measure how they are scattered.

The instrument is a part of the Russian contribution to Curiosity and a broad collaboration between the United States and Russia.

In addition to water, DAN can be used to find suiting places to take samples for the other instruments aboard Curiosity.
The data is also complimentary with the camera images and study of the Martian surface and geology.

DAN works by aiming neutrons with high energy at the ground, and registering their energy after that have bounced back.
The loss in kinetic energy is characteristic for each element, and the amount of time it takes for the neutrons to get bak gives an indication of how much of said element is in the ground.
One of the elements that can be, and has been, detected by DAN is hydrogen.
Already before Curiosity was sent to Mars this technology was used to find water on the red planet.
Neutrons from the background radiation in the cosmos was used as a source when a satellite called Odyssey detected water back in 2002.
The hydrogen was interpreted as water close beneath the Martian surface.

DAN can also use the cosmic background radiation as a source for neutrons, but can also actively shoot neutrons towards the ground for higher resolution and more effective tests.

Hydrogen detected by DAN can be signs of former water that at the time got bound in crystals.
These crystals are called hydrated minerals, and can be left over from a wetter period on Mars.