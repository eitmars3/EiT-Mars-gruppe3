\begin{tcolorbox}[colback=red!5,colframe=DarkRed!40!black,title=Mars: 10 facts about the red planet]
\textbf{Location:} Mars is the 4th planet from the Sun, about 1,5 times far out as Earth.

\textbf{Size:} About half the diameter of the Earth, it is the second smallest. A $100 kg$ person would weight $38 kg$ on Mars.

\textbf{Magnetism:} It has no spinning iron core and thus no planet-wide magnetic field.

\textbf{Orbit:} The distance from the Sun due to the elliptical orbit varies significantly more than on Earth.

\textbf{Time:} A year lasts for $687$ Earth days. Martian day = \textit{sol} = \textit{24 hours, 39 minutes and 35 seconds}.

\textbf{Atmosphere:} The atmosphere mainly consists of carbon dioxide ($95,3 \%$), nitrogen ($2,7 \%$) and argon ($1,6\%$).
The atmosphere on Earth is almost 200 times denser than on Mars.

\textbf{Climate:} Temperatures varies between $-128C$ and $27 \degree C$ with an average of $-53 \degree C$.

\textbf{Geography:} Olympus Mons, the tallest known mountain, is $26 km$, which is approximately 3 times as tall as Mount Everest.
Valles Marineris is the solar systems' largest and deepest known system of valleys.

\textbf{Moons:} Mars has two moons: Deimos and Phobos.

\textbf{Surface:} The surface has minerals containing silicon, oxygen and metals. Its red color comes from iron oxide.
\end{tcolorbox}