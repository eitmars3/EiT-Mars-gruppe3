\begin{tcolorbox}[colback=green!5,colframe=green!40!black,title=Mars: 10 facts about the red planet]
\textbf{Location:} Mars is the 4th planet from the Sun and is the second smallest.

\textbf{Size:} About half the diameter of the Earth. A 100kg person would weight 38kg on Mars.

\textbf{Magnetism:} It has no spinning iron core and thus no planet-wide magnetic field.

\textbf{Orbit:} The distance from the Sun due to the elliptical orbit varies significantly more than on Earth.

\textbf{Time:} A year lasts for 687 Earth days. Martian day = sol = 24 hours, 39 minutes and 35 seconds.

\textbf{Atmosphere:} The atmosphere mainly consists of carbon dioxide (95,3\%), nitrogen (2,7\%) and argon (1,6\%).
Atmosphere on Earth has 100 times the pressure than on Mars.

\textbf{Climate:} Temperatures varies from -128C to 27C with an average of -53C.

\textbf{Geography:} Olympus Mons, the tallest known mountain, is 26km, which is approximately 3 times as tall as Mount Everest.
Valles Marineris is the solar systems' largest and deepest known system of valleys.

\textbf{Moons:} Mars has two moons - Deimos and Phobos.

\textbf{Surface:} The surface has minerals containing silicon, oxygen and metals. Its red colour comes from iron oxide.
\end{tcolorbox}