\section*{Entry, decent and landing (EDL)}
\initial{A}fter a successfully launch from Earth at Space Launch Complex 41 on Cape Canaveral Air Force Station in Florida at 10:02 a.m. EST, Nov. 26, 2011 a 563 270 400 km travel awaited for curiosity \cite{MissionTimeline} [2]. When reaching the Martian atmosphere the Entry, descent and landing phase (EDL) began, a spectacle popularly known as seven minutes of terror [2]. Because of Curiosity?s large weight and size, it could not utilize the landing procedures used for earlier rovers. In fact, this landing method has never been tried before. It is as unique as it is complicated, thus the name seven minutes of terror (referring to the time it takes Curiosity to land on the surface after touching the Martian atmosphere) [8]. 
The EDL sequence breaks down into four parts; guided entry, parachute descent, powered descent and the sky crane [3].

(bilde EDL)

\subsection*{Guided entry}
\initial{10} minutes before reaching the atmosphere the cruising stage is detached and burns up.
With a velocity of 21 240 km/h it slams into the Martian atmosphere at an altitude of 125 km [4]. Creating so much aerodynamic drag the heat shield reach a temperature of 2100?C glowing white hot [3] [5]. Being the first spacecraft to use precision landing techniques on Mars the name ?guided entry? refers to its ability to adjust it course during this first stage. The course adjustments will be controlled by four sets of two Reaction Control System (RCS) thrusters. Each pair capable of generation about 500 N (50 kg) of thrust [7]. Where the Mars Exploration Rovers could have landed anywhere within their respective 150 by 20 kilometers landing ellipses, Curiosity landed within a 20 kilometer ellipse [3].
The Martian atmosphere is extremely difficult to slow down in because it has just enough atmosphere that you have to deal with it (or else your spacecraft is destroyed), but still too little to get the job done [5]. The heat shield only manages to slow the spacecraft to a speed of 1620 km/h in which time the deceleration has created a maximum force equal to 15 G?s [3] [7]. This is still way too fast for a safe landing and brings us to the next step.

Parachute descent
At an altitude of 11 km, the largest and strongest supersonic parachute ever constructed for an extraterrestrial flight deploys [3] [6]. Deploying at such speeds creates a neck snapping 9 G?s [5]. The parachute with a diameter of nearly 16 m has 80 suspension lines measuring more than 50 m in length and would be capable of generating 24 500 kg of drag force [6]. About 8 km over the surface the heat shield poops of giving a clear view for the radar. The radar calculates altitude and speed determining when to initiate the next phase [3].
The parachute will not be able to slow the spacecraft to a speed lower than about 280 km/h. This is still way too fast for a landing. Only one solution; separate the rover from the backshell and parachute [3].

Powered descent
The descent stage (includes the rover) now in free fall fires eight Aerojet?s variable thrust mono propellant hydrazine rocket thrusters and doing a divert maneuver to avoid colliding with the backshell and parachute [3] [7]. Each of these rocket thrusters (called Mars Lander Engines or MLEs) are capable of producing 320 kg of lift [7]. The MLEs will slow the spacecraft to a velocity of 14.5 km/h while adjusting its position for the final stage. 

Sky crane
At an altitude of 27 m, the rover is detached from the descent stage and slowly lowered to the ground. While three nylon cords (communications between the rover and the descent stage is ensured by an umbilical line) lower the rover, the descent speed is reduced to 2.7 km/h. At 7.5 m, the cords are fully extended and the descent speed is kept constant at 2.7 km/h until the rover finally touches down [3] [7]. The descent stages continues it decent velocity while the rover use 2 seconds to confirm full weight on all wheels. The cords and umbilical line pyrotechnically detaches from Curiosity and the descent stages assume command of itself. Tilting 45-degrees, it performs a flyaway-maneuver crash-landing at least 150 m away [7]. 
Despite everything that could have gone wrong the rover landed unharmed on the surface of Mars at 10:32 p.m. PDT on Aug. 5 2012 [3].


References
[1]	http://mars.nasa.gov/msl/mission/timeline/launch/ - MissionTimeline
[2]	http://edition.cnn.com/2012/08/10/us/mars-curiosity/index.html?eref=mrss_igoogle_cnn
[3]	http://mars.nasa.gov/msl/mission/timeline/edl/
[4]	http://www.startalkradio.net/wp-content/uploads/2012/08/667372main_MSL-EDL-rev-2900.jpg
[5]	https://www.youtube.com/watch?v=h2I8AoB1xgU
[6]	http://mars.jpl.nasa.gov/msl/news/index.cfm?fuseaction=shownews&newsid=90
[7]	http://www.nasaspaceflight.com/2012/08/msl-curiosity-historic-martian-landing-at-gale-crater/
[8]	http://lightyears.blogs.cnn.com/2012/08/03/8058/