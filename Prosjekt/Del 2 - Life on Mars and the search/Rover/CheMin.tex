(!!!no latex formatting yet!!!)

Curiosity is not only able to study the very surface of Mars.
 With an arm attached it can drill into the soil to excavate powdered samples from the ground.
 These samples are studied with Curiosity's Chemistry & Mineralogy (CheMin).
 CheMin's main task is to conduct powder X-ray diffraction to study what minerals Martian soil consist of.
 X-rays are sent through the sample powder so that these rays will be diffracted in a pattern that depends on each mineral component in the sample.
 The diffracted rays are then captured by an X-ray sensitive 600x582 charge coupled device (CCD), that is a camera which operates with X-rays wave length.
 The CCD may read, erase and recharge, take individual images in other words, 1000 or more times for each experiment to ensure polite results.
 Each exposure takes from 5 to 30 seconds, resulting in about 10 hour duration for each experiment.

Source:
"Chemistry & Mineralogy (CheMin)"
http://msl-scicorner.jpl.nasa.gov/Instruments/CheMin/ (visited: 2015-03-24)