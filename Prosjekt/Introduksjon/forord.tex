\section*{Preface}
\initial{T}his paper has been written for the NTNU class TBT4850 Experts in Team.
It has been written by a team of engineers from different disciplines.
Martin Nordal and Karsten Kjensmo represent computer science, Ingelin Garmann is a process chemist, Simen Hegge is a constructional engineer, Jonas Misund is a physicist and Anna Testani\`{e}re is a mathematician.
A third of the group is international, therefore the paper is written in English. 
The EiT village topic is biology-oriented, and because the team had no biologists, the choice was made to take a broader approach. 
The aim of the paper is to bring the interested reader to a level of layman's competence, and to give an overview of the relevant scientific and philosophical topics of extra-terrestrial life.
It is intended to induce a sense of wonder and to present a foundation for curiousity on the theme. It looks to find an answer to the importance of human curiousity in the search for extra-terrestrial life, on which technical solutious this has led to, and how this search can develop in the future.  
The paper is written for the fascinated amateur \emph{by} the fascinated amateur, and presents the hunt for life on Mars through a broader scope. 
Developments in the field may have occured after this paper was written. 
