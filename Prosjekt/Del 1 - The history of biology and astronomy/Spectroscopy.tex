\subsection*{Spectroscopy, radiation and nuclear physics}
Since the 1600s physicists have studied radiation.
In the beginning it was mostly light in its observable spectrum, red through blue.
After a while we discovered that the visible spectrum of light was only a small part, and that it contained radio, x-ray and gamma radiation to mention a few.
Other kinds of radiation was soon discovered, like the radioactive radiation Henri Becquerel \cite{First_radioactivity} found in uranium.


All sorts of radiation is used in spectroscopy. Photons in light, the decay from radioactive sources; alpha, beta and gamma radiation and others.
It is the particles in the radiation that are important to spectroscopy.
Although the theory behind spectroscopy is advanced physics, the principle is simple.
Since every element has different properties one simply fires a particle at what you want to study, and measures the energy when it bounces off.
To figure out how the energy of the particle changes, we use quantum and particle physics.
This is where it becomes a little bit ''quirky''.


Quantum physics says that at the smalles level, all energy is quantified.
It comes in packets, and are not continous.
Every single electron has the same charge, photons of a particular wave length always has the same energy and the energy change when a particle hits an atom will always be one of a few distinct energies.
Particle physics incorporates these principles and explains what energies we should expect to find for different matter.
As a result, we can differentiate between platinum and aluminum simply by shooting it with particles and measuring the energies afterwards.


\bibitem{First_radioactivity}
	Wikipedia,
	\emph{Radioactive decay},
	http://en.wikipedia.org/wiki/Radioactive_decay,
	18.03.2015