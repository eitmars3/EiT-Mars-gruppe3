\subsection*{Spectroscopy, radiation and nuclear physics}
Since the 1600s physicists have studied radiation.
In the beginning it was mostly light in its observable spectrum, red through blue.


Although the theory behind spectroscopy is advanced physics, the principle behind it is simple.
Since every element has different properties one simply fires a particle at what you want to study, and measures the energy when it bounces off.
To figure out how the energy of the particle changes, we use quantum and particle physics.
This is where it becomes a little bit ''quirky''.
Quantum physics says that at the smalles level, all energy can be quantified.
This means that energy actually comes as a collection of small packets, it is not continuous.
Every single electron has the same charge, photons of a particular wave length always has the same energy and the energy change when a particle hits an atom will always be one of a few distinct energies.
Particle physics incorporates this theory and explains what energies we should expect to find for all different kinds of matter.
As a result, we can differentiate between platinum and aluminum simply by shining shooting it with particles and measuring the particles coming back.