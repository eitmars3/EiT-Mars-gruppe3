\section*{Physics crash course}

\subsection*{Entropy}
Imagine a room full of balls bouncing off the walls and each other.
Your image is probably a chaotic mess of balls traveling in all directions and speeds.
The order, or disorder, of the balls is what physicists call entropy.

Entropy is derived from thermodynamics, and it explains some of the most fundamental laws of our universe.

To understand what entropy really means, imagine all of the bouncing balls in your room being in the same corner at the same time.
This must be a very unlikely occurrence.
There are few ways to place all of the balls together in one corner, but lots of ways to spread them all around the room.
When all the balls are in the corner, the room has a low entropy.
If the balls are all over the place, the entropy is high.


The second law of thermodynamics says that the entropy in a closed system never decrease.
In other words; no matter what happens in a closed system, for example a room with no interaction with the outside world, there will be more and more chaos over time.
The system, or room, will eventually reach what we call a thermodynamic equilibrium.
If you want to keep the order in you room, you will need to supply it with energy.
That is partially why humans needs to eat.
Our cells use energy in chemical reactions, and we need to keep the entropy low in order for these chemical reactions to continue.
That is living beings has to be able to work against an increase in entropy, and also the reason this is one of the requirements we use to define life as we know it.


------------------------------------------------------------------------------------------

Entropy is in general a measurement of disorder in a system.
In thermodynamics, entropy is a measure of the number of specific ways in which a thermodynamic system way be arranged.
The entropy can never decrease in an isolated system, according to the second law of thermodynamics.
Eventually it will reach thermodynamic equilibrium, the configuration with maximum entropy. \cite{Wiki-entropy]


One can also think of entropy as the probability that a system will be in a certain configuration.
If you have box with 10 balls bouncing around, it is less likely that at any given time they will all be in the same corner.
Therefore the configuration where all the balls are close together in a corner has low entropy.


In thermodynamics the balls can be an analogy for gas particles in the air.
They will be bouncing around from each other, walls, floors and other objects.
The average speed of the particles in the air is proportional to what we call temperature.


Because the entropy in an isolated, or closed, system will always increase it needs a supply of for example energy to keep the entropy low.
This is one of the more intricate requirements for life.
A living organism typically uses metabolism to collect energy, which is transported to the cells of the organism to keep the entropy low and the chemical processes going.

\begin{thebibliography}{99}
	\bibitem{Wiki-entropy}
		Wikipedia,
		\emph{Entropy},
		March 11th 2015
	\bibitem{}
\end{thebibliography}