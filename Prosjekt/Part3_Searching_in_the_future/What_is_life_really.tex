\initial{W}e have just examined the specifics of life at micro scale, but could we be missing the forest for the trees?
There are some creative theories that view life on grander scales, and even predictions that a human invention will one day challenge our concept of life.

\subsection{Life in a computer}
\initial{T}he interest in Artificial Intelligence (AI) has risen in recent years, as business corporations see the opportunity to replace human manpower with computers to enable increased workloads.
Development in this field of study has of course lead to some interesting inventions.
In early 2014 Facebook announced that their face recognition software was as accurate as the human brain \cite{facebook}.
This example shows how inventions in AI have a rather practical approach these days, with single purposes, although the holy grail would be to create a more complex software and even with consciousness.
If algorithms like Facebook's were to be simplified, or when their patents are no longer valid, perhaps even more complex algorithms can be formed.

AI is based on human behavior, and is also connected to psychology.
The assumption that humans are the most intelligent beings, gives us the motivation to try to replicate ourselves.
Despite what many people believe, AI does not have to include fancy robotics and machinery, nor does it have to be an android in physical matter.
An agent, which is the smallest unit of AI, senses, reflects and actuates.
The sensors and actuators gives its physical constraints, and the reflection is core of the AI.
There are two levels of AI: weak AI and strong AI.
Weak AI is when an agent acts intelligent, which disables a Turing test to distinguish it from a human.
The Turing test is a test in which a human judge, holding a conversation with another entity, must deem whether or not that entity is human or machine. 
When it comes to an actual intelligent agent, it is a subject to strong AI.
However this is yet to be made.


\paragraph{The ghost in the machine}

In Philip K. Dicks emph{Do Androids Dream of Electric Sheep?}, which was made into the slightly-more-famed movie \emph{Blade Runner}, the protagonist Deckard works as a policeman specializing in catching runaway and criminal androids. In Deckard's world, artificial intelligence has advanced to such a level that telling the difference between a machine and a human being takes expert level knowledge and the application of a variant of the Turing test, a test in which a human judge, holding a conversation with another entity, must deem whether or not that entity is human or machine. 
While Deckard's machines aren't the end-all, be-all of artificial intelligence, they mark the crucial point in coming human history where human beings and machines become indistinguishable to anything but the trained eye. If we can reach a stage where having a conversation with a machine becomes indistinguishable from having a conversation with a regular human being, then can we say that the machine isn't alive? Can something mechanical be infused with a soul? Can something consisting of copper wires and transistors, designed in a lab and built in a factory, be just as living as something made of organic materials? Is it possible that we will \emph{build} life before we discover it?

\paragraph{Architects of our own demise}
The subject of science as a force for destruction is a well-known issue. Engineering has given us the tools to perform miracles, but also the tools to facilitate our own apocalypse. Recently, debates have grown regarding the future dangers of artificial intelligence - at its highest ideal, artificial intelligence will give birth to entities that will seem \emph{god-like} compared to a regular human being. Infinitely higher capacity for information storage and search, instant learning, physical capabilities and endurances so superiour to our own that we will be powerless to stop them \emph{should we make enemies of these machines}. This is the \emph{Terminator} endworld dystopian nightmare, in which human beings are in the process of being exterminated by the superior technology of the machine race. 
AI experts have split opinions on this potential future. Some believe that these future dangers are negligable; we will always have an element of control, or the machines will have no reason to abuse us, and the benefits of such god-like beings in our employ are beyond imagining. There is also a line of belief that says humanity will end itself before this kind of AI becomes global; the very respectable Hugo de Garis presents his theory on \emph{The Artilect War}\cite{artilect}, in which rival factions within the human race will destroy each other before powerful AI is made. The first of these two factions is convinced of the benefits of AI and wishes to continue to build them, whilst the other is so terrified of the machine future that they make war on the first faction. With such names as Stephen Hawking and Elon Musk (whom, in fairness, are not AI experts) speaking out against the development of AI, a careful eye has to be kept on the development of these technologies. 

\subsection{The Gaia theory}
\initial{C}onsidering what we now know of the unwitting coordination of the cells in our body, the Gaia theory is intriguing.
Originally concocted by a chemist named James Lovelock\cite{Lovelock} in the 60s, it is about how the very planet we live on can be a single organism consisting of trillions of lifeforms working together unwittingly.
Could it be that we are just cells in the body of the planet Earth?
Are we zooming around the Sun in a single organic spaceship?
Could we, in the future, as our space travel talents grow, encounter living planets as singular entities?