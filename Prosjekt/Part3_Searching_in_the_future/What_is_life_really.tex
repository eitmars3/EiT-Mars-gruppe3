\subsection*{What is life, \textit{really}?}

There has been theories that 
If we look at the criteria that defines life it could be that our surroundings is part of a life form.

\subsubsection{The Gaia theory}
One of the more interesting theories is Gaia theory \cite{Lovelock}, which relates to life on micro and macro scales. Now this is a theory, but it's an interesting one, and it's worth noting, especially with what we now know of the unwitting coordination of the cells in our body.
Gaia theory, originally concocted by a chemist named James Lovelock in the 60's, is about how the very planet we live on can be, just as we are, a single organism consisting of trillions of lifeforms working together, unwittingly.
Could it be that we are just cells in the body of the planet Earth?
Are we zooming around the sun in a single organic spaceship?
Could we, in the future, as our space travel talents grow, encounter living planets as singular entities?

\subsubsection{The universe is alive}

\subsubsection{Life in a computer}
The interest in artificial intelligence (AI) has risen in recent years, as business corporations see the opportunity to replace human manpower with computers to enable increased workloads.
Development in this field of study has of course lead to some interesting inventions.
In early 2014 Facebook announced that their face recognition software was as accurate as the human brain \cite{facebook}.
This example shows how inventions in AI have a rather practical approach these days, with single purposes, although the holy grail would be to create a software with consciousness.
If such algorithms were to be simplified or their patents are no longer valid, perhaps .