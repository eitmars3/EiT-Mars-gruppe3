If we look at the criteria that defines life it could be that our surroundings is part of a life form.
This has lead people to make different theories, and even predictions that humans one day will be able to produce intelligent beings.

\subsubsection*{The Gaia theory}
One of the more interesting theories is Gaia theory \cite{Lovelock}, which relates to life on micro and macro scales. Now this is a theory, but it's an interesting one, and it's worth noting, especially with what we now know of the unwitting coordination of the cells in our body.
Gaia theory, originally concocted by a chemist named James Lovelock in the 60's, is about how the very planet we live on can be, just as we are, a single organism consisting of trillions of lifeforms working together, unwittingly.
Could it be that we are just cells in the body of the planet Earth?
Are we zooming around the sun in a single organic spaceship?
Could we, in the future, as our space travel talents grow, encounter living planets as singular entities?

%\subsubsection*{The universe is alive}

\subsubsection*{Life in a computer}
The interest in Artificial Intelligence (AI) has risen in recent years, as business corporations see the opportunity to replace human manpower with computers to enable increased workloads.
Development in this field of study has of course lead to some interesting inventions.
In early 2014 Facebook announced that their face recognition software was as accurate as the human brain \cite{facebook}.
This example shows how inventions in AI have a rather practical approach these days, with single purposes, although the holy grail would be to create a more complex software and even with consciousness.
If algorithms like the one of Facebook's were to be simplified or when their patents are no longer valid, perhaps even more complex algorithms can be formed.

AI is based on human behaviour and is also connected to psychology.
The assumption that humans are the most intelligent beings gives us the motivation to try to replicate ourselves.
Despite what many people believe, AI does not have to include fancy robotics and machinery, nor does it have to be an android in physical matters.
An agent; which is the smallest unit of AI, senses, reflects and actuates, where the sensors and actuators gives its physical constraints, and the reflection is core of the AI.
There are two levels of AI, weak AI and strong AI. Weak AI is when an agent acts intelligent, which disables a Turing test to distinguish it from a human.
When it comes to an intelligent agent, it is a subject to strong AI, however this is yet to be made.