\initial{W}e have just examined the specifics of life at micro scale, but could we be missing the forest for the trees?
There are some creative theories that view life on grander scales, and even predictions that a human invention will one day challenge our concept of life.

\subsection{The Gaia theory}
Considering what we now know of the unwitting coordination of the cells in our body, the Gaia theory is intriguing.
Originally concocted by a chemist named James Lovelock\cite{Lovelock} in the 60s, it is about how the very planet we live on can be a single organism consisting of trillions of lifeforms working together unwittingly.
Could it be that we are just cells in the body of the planet Earth?
Are we zooming around the Sun in a single organic spaceship?
Could we, in the future, as our space travel talents grow, encounter living planets as singular entities?

\subsection{Life in a computer}
The interest in Artificial Intelligence (AI) has risen in recent years, as business corporations see the opportunity to replace human manpower with computers to enable increased workloads.
Development in this field of study has of course lead to some interesting inventions.
In early 2014 Facebook announced that their face recognition software was as accurate as the human brain \cite{facebook}.
This example shows how inventions in AI have a rather practical approach these days, with single purposes, although the holy grail would be to create a more complex software and even with consciousness.
If algorithms like Facebook's were to be simplified, or when their patents are no longer valid, perhaps even more complex algorithms can be formed.

AI is based on human behavior, and is also connected to psychology.
The assumption that humans are the most intelligent beings, gives us the motivation to try to replicate ourselves.
Despite what many people believe, AI does not have to include fancy robotics and machinery, nor does it have to be an android in physical matter.
An agent, which is the smallest unit of AI, senses, reflects and actuates.
The sensors and actuators gives its physical constraints, and the reflection is core of the AI.
There are two levels of AI: weak AI and strong AI.
Weak AI is when an agent acts intelligent, which disables a Turing test to distinguish it from a human.
The Turing test is a test in which a human judge, holding a conversation with another entity, must deem whether or not that entity is human or machine. 
When it comes to an actual intelligent agent, it is a subject to strong AI.
However this is yet to be made.