%kilde: http://www.telegraph.co.uk/news/science/space/10200818/Dangers-of-a-manned-mission-to-Mars.html

\begin{tcolorbox}[colback=green!5,colframe=green!40!black,title=5 dangers for humans to overcome at Mars]
\textbf{Atmosphere:} It is impossible to breath in, due to its low density, as well as its high complex of carbon dioxide.

\textbf{Radiation:} There is constant background radiation on Mars, but there is also a much stronger direct radiation from the Sun. With no significant protection from the Martian atmosphere or a planet wide magnetic field, radiation can reach intensities enough to ionise atoms and thus split chemical compounds, making the environment though to any living creature we know of.

\textbf{Climate:} Liquid water is not known to occur on Mars, however the poles have permafrost. The warmest climate, and thus the most friendly to humans, on Mars is to be found around its equator.

\textbf{Meteorites:} Mars is located near an asteroid belt, and with weak protection from the atmosphere, the Martian surface gives a clue on how great consequences these impacts may cause.

\textbf{Health:} Living in low gravity causes the human body to slowly fall apart. This is crucial when travelling to Mars, but could also be significant at the Martian surface. It is also expected that isolation so far from home could affect the traveller's mental health.
\end{tcolorbox}