So far, our pursuit for extraterrestrial life has been based on our knowledge from Earth.
As we already know, all known life on our planet is based on carbon compounds and water as a solvent \cite{OForm2}.
But as William Brains stated concerning these molecules in 2004: "this is not an inevitable conclusion from our knowledge of chemistry" \cite{OForm1}.
So if we open our minds, what could then exist out there?
Moreover, where could it exist?
And what could potentially replace these substances?

\subsection*{Non-carbon based life}

First, let us focus on carbon. Several prominent scientists are concerned that we might be blinded by our familiarity with carbon and Earth-like conditions.
It is not beyond the realm of feasibility that our first encounter with extraterrestrial life will not be a solely carbon-based \cite{OForm3}.
Several compounds might do the same trick.

Silicon is the most promising element to replace carbon as a basis for an alternative biochemical system. 
Located directly beneath carbon on the periodic table, silicon has similar chemical properties as carbon \cite{OForm4}.
Silicon has the same number of electrons in its outer shell, meaning it can form four bonds just like carbon.
Another shared property is that silicon can bond to itself making Si-Si bonds and therefore form large enough molecules to carry biological information \cite{OForm5}.
This might sound promising, after all C-C bonds are the basis for complex molecules on Earth \cite{OForm4}.
Another of carbon's vital features is that it can form chemical bonds with many other atoms creating organic functional groups.
The chemical versatility required to conduct the reactions of biological metabolism and propagation is caused by carbons ability to bond with hydrogen, oxygen, nitrogen, phosphorus, sulfur, as well as metals such as iron, magnesium and zinc.
Silicon, in contrast, interacts with only a few other atoms making it monotonous compared with carbon \cite{OForm5}.

Some of the most common carbon molecules on Earth, such as carbon dioxide CO2 and methane CH4 do have silicon derivatives (silicon dioxide SiO\textsubscript{2} and silane SiH\textsubscript{4}).
Although SiO\textsubscript{2} is common on Earth (quartz), it is, in contrast to CO\textsubscript{2} a solid at temperature below 2000\degree C.

So far, we have compared silicon and carbon within the context of earth-like conditions.
This does not have to be the case for other planets.
Can you imagine a silicon-based organism, living in a 3000 degrees atmosphere?
It may sound unlikely, but chemistry able to support life does not have to be within the temperature ranges we find comfortable.
Even here on earth one specie's sweet spot could be another specie's worst nightmare.

Biochemists also speculate whether there are other substances that could replace carbon.
In theory, many elements could work.
Even counter-intuitive elements such as arsenic may be capable of supporting life under the right conditions.
Actually, some marine algae is reported to incorporate arsenic into complex organic molecules \cite{OForm3}.
Chlorine, nitrogen, phosphorus and sulfur are also possible elements to replace carbon.
Sulfur and phosphorus are capable of forming long-chain molecules much like carbon and some bacteria have been discovered to survive on sulfur rather than oxygen \cite{OForm3}.

\subsection*{Non-water solvent}

Water is considered a key factor when searching for extraterrestrial life.
Prominent scientists, such as Carl Sagan, believe carbon to be difficult to replace, but that water is less essential \cite{OForm2} \cite{OForm3}.

Ammonia, for example, has many of the same properties as water.
It can dissolve most organics as well as or better than water, and it has the unprecedented ability to dissolve many elemental metals such as sodium, magnesium and aluminum.
Several other elements such as iodine, sulfur, selenium and phosphorus are also somewhat soluble in ammonia with minimal reaction.
All these elements are important in biochemistry capable of supporting life.
The vital solvent for a living organism should be capable to permit acid-base reactions.
By using ammonia as a solvent the acids and bases are quite different than when using water because acidity and basicity are defined relative to the medium in which they are dissolved. Therefore, water and ammonia are not chemically identical but just simply analogous.
However, on the down side, the temperature range in which ammonia stays liquid is rather small compared with water (at one atm.) and the surface tensions are only one third of as much.
The hydrogen bonds that exist between ammonia molecules are also much weaker than those in water \cite{OForm7}.

Ammonia or an ammonia-water mixture has a substantial lower freezing temperature \cite{OForm3}, and this brings us back to Titan, one of Saturn's moons.
Rich on complex organic chemistry but way too cold to hold liquid water it is considered unsuitable to support life.
Considering alternative biochemistry by replacing water with ammonia Titan would be highly capable to support extraterrestrial life.

Stephen Hawking has several exciting ideas about extraterrestrial life.
He imagines lifeforms independent of water as a solvent.
Nitrogen exists as a gas at normal conditions, but at very low temperatures, -196\degree C or lower, it becomes a liquid resembling water \cite{OForm6}.
Does a world exist, far beyond the habitable zone, holding large oceans of liquid nitrogen capable of supporting life?
This idea is as intriguing as it is unlikely.
Nitrogen and water have quite different chemical properties.
However, as Stephen Hawking underlines, with a radically different environment, comes radically different chemical reactions.
Water and carbon might be the very last things capable of supporting life in some extreme planetary conditions \cite{OForm3}.

All of these ideas might sound just a bit too incredible and unlikely, but actually, it has some emphasis. 
Discoveries over the past decades revealed extreme lifeforms thriving at highly unlikely places, such as superheated walls of ocean-volcanic vents and the interior of the planets crust.
The NASA sponsored report recommends that the search for life should be widened to include the possibility of "weird" life.
It concludes: "Nothing would be more tragic in the American exploration of space than to encounter alien life and fail to recognize it" \cite{OForm3}.

If we open up for these possibilities, life seems suddenly able to exist everywhere, just awaiting to be discovered.