KOMPETANSETREKANT

Allerede andre landsbydag fikk gruppen i oppgave å lage en kompetansetrekant. Kompetansen ble evaluert innenfor tre kategorier: teoretisk kunnskap innenfor sitt fagfelt, praktisk erfaring og personlige trekk. 
I første omgang laget hvert medlem en individuell trekant, for så å presentere sine evner til gruppen. 
Til slutt samarbeidet gruppen om å lage en trekant som representerte hele gruppens kompetanse.
Hensikten med oppgaven var å bevisstgjøre gruppemedlemmene om sin individuelle kompetanse og gi innsikt i de andre medlemmenes kompetanse. 
På dette tidlige tidspunktet kjente ikke gruppen til hver enkelts kompetanse og var heller ikke vant til å evaluere sine egne kunnskaper.

Det mest slående med gruppens endelige kompetansetrekant var spredningen i fagfelt.
En heterogen gruppe er ofte en fordel.
Personer med ulik bakgrunn vil ha ulike løsninger, tanker og ideer om samme problemstilling. 
På denne måten oppnår man et høyere nivå av forståelse, blir introdusert for nye innsynsvinkler og kan besvare problemstillingen grundigere.
Mangfoldet gir grunnlag for en dynamisk diskusjon der et individs opprinnelige ide vil bli utviklet videre av hele gruppen.
Resultatet er at ideen får et forbedret og bredere innhold enn det et enkeltmedlem alene har kompetanse til å besvare.
Et slikt samarbeid krever at medlemmene klarer å formidle sine ideer til fellesskapet.
Når gruppen er heterogen er det viktig at ideer blir formidlet slik at alle forstår budskapet.
Like viktig som å være en god formidler, er det å kunne lytte til andres ideer. 


Den og mangelen på kunnskap innen biokjemi.
