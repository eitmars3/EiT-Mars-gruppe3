%Effektivitet (sakte progresjon, brukt tid pÅ Å bli ferdig, grundig vs tidskrevende)
%Hovedansvar: Simen & Ingelin

%Vil se pÅ hvordan gruppa har jobbet effektivt med prosjektrapporten og prosessoppgavene. Starte med Å se pÅ hvordan vi laget problemstilling og disposisjon, fordelte arbeidsoppgaver og konkret jobbet med dem. Under prosessoppgavene kan det vÆre viktig Å ta opp kommunikasjonsmØnstre i tillegg.

%%% FERDIG %%%

%Konkret: Samarbeidskontrakt (spesifikk respect for andres forslag)

%Konkret: Rullering av gruppeleder/referent, hvilke innsikter dette har gitt og hvordan dette har pÅvirket gruppen

%Lagt mye fokus pÅ konsensus og at vi skal ha en felles forstÅelse for arbeidet (SITRA kan nevnes som eksempel)

%%% GJENSTÅR %%%

%Konkret: Se pÅ trekanten: Trivsel, leveranse, lÆring, prØve og plassere oss. Gjenspeiles i samarbeidsavtalen
%Konkret: Utviklingen av samarbeidsindikatoren
%Abstrakt: Hvordan utspiller det seg nÅr gruppen jobber
%Abstrakt: Arbeidsfordeling, hvordan skal folk fÅ brukt kunnskapen sin
%FravÆr av aktivt system for “to-do” (f eks Trello)
%AngÅende prosessrapport: FØrste gang vi skriver en sÅnn rapport, sÅ det er vanskelig Å komme i gang
%Abstrakt: Pros and cons av jobbing for et mÅl - milepÆler, deadlines, disposisjon med tidsfrister

På engelsk eksisterer det to begrep som måler grad av suksess ved et resultat. 
I $Oxford Dictionary of English$ er $effectivity$ definert som i hvilken grad et produsert resultat gjenspeiler det ønskede målet, mens $efficiency$ omhandler hvor mye ressurser som har blitt brukt på å skape det produserte resultatet. 

Å utvikle et effektivt gruppearbeid er ansett som et hovedmÅl i karakterisering av en velfungerende gruppe. 
Som inngÅr i klassifiseringen av en gruppe, mÅ den arbeide mot et felles mÅl. 
Oftest vil dette mÅlet ogsÅ vÆre tidsfestet slik at en viss progresjon er krevd for hver gang gruppen mØtes. 
I det gruppearbeidet som omtales i denne rapporten var oppgaven, som nevnt tidligere, Å forfatte en populÆrvitenskapelig artikkel med tema $Biologisk skattejakt pÅ planeten Mars$, gjennomfØre en rekke oppgaver innen teambuilding samt reflektere over disse og til slutt presentere gruppeprosessen i en egen rapport. 
Dette arbeidet skulle gjennomfØres i lØpet av 15 uker hvor gruppen var planlagt og mØtes hver onsdag fra 08.30\endash16.00. 
Allerede 2. landsbydag ble det enighet om at arbeidet med EiT skulle finne sted de gitte onsdagene, og ikke ellers, sÅ langt det lot seg gjØre. 
Å gjennomfØre et effektivt gruppearbeid ble derfor fra starten av et viktig fokus. 

%Samarbeidskontrakt
En samarbeidskontrakt er en skriftlig avtale mellom medlemmene i en gruppe som stadfester punkter vedrØrende bÅde praktisk og sosialt samarbeid og ambisjonsnivÅ. 
Hensikten med avtalen er at gruppen som helhet fÅr diskutert og kommet til enighet om rammene og mÅlet med samarbeidet, og ikke minst at det finnes noe konkret Å henvise til ved eventuelle konflikter og brudd pÅ kontraktspunkter. 

Rammeverket for gruppearbeidet ble utarbeidet sÅ tidlig som 2. landsbydag med mulighet for revisjon gjennom hele arbeidsperioden. 
Hovedemnene var leveranse, lÆring og trivsel.
Ambisjonen om Å gjennomfØre et effektivt gruppearbeid ble gjenspeilet i flere punkter. 

En forutsetning for Å kunne samarbeide som gruppe er at gruppen faktisk mØtes. 
Prosedyre ved fravÆr ble dermed behandlet i flere punkter. 
Gruppen var enig om at bÅde forsinkelse og fravÆr skulle varsles, men at det ikke var nØdvendig med kontraktsbestemt sanksjon hvis slikt inntraff. 
Ved planlagt fravÆr skulle vedkommende bli tildelt en oppgave for Å jobbe inn tapt tid. 
Å behandle fravÆr som kontraktsfestede punkt bidro til at medlemmene hadde dem i bakhodet selv om de ikke ble referert til aktivt.
FravÆr og forsinkelser ble i stor grad meldt fra om, og gruppen var klar over at bÅde Karsten og Simen ville ha planlagt fravÆr i forbindelse med pÅskeferien. 
PÅ den mÅten kunne gruppen pÅ forhÅnd anmode om Å fÅ utsatt prosesssamtalen med lÆringsassistenter og landsbyleder til en onsdag hvor alle gruppemedlemmene var til stede.
FravÆr og forsinkelser ble aldri et stort problem for gruppen, og det ble ikke ansett som nØdvendig Å innfØre strengere rammer for fravÆr under revisjonen av kontrakten midtveis i arbeidsperioden. 

Neste punkt lØd enkelt og konkret at alle skal gjØre en Ærlig innsats i lØpet av landsbydagen.
Enda verre enn at gruppemedlemmer er fravÆrende, er passive og kanskje til og med motarbeidende deltakere.
PÅ det tidspunktet kontrakten ble forfattet hadde gruppemedlemmene enda ikke fÅtt innsikt i hverandres arbeidsmentalitet og arbeidsinnstilling.
Dette punktet, som direkte omtalte Ønsket om effektivitet, var derfor i stØrre grad ment som motivasjon og oppfordring, enn som aksjon pÅ en inntruffet situasjon.
I ettertid har gruppen reflektert over at alle medlemmenes innstilling og pÅgangsmot for Å gjØre godt arbeid har vÆrt utelukkende av den positive sort.
Flere punkter som slÅr ned pÅ ikke-bidragende deltagelse har derfor aldri vÆrt aktuelt.

I grupperefleksjonen fra 2. landsbydag stÅr det at det ble en del drØfting rundt semantikken i samarbeidsavtalen. 
Diskusjon i plenum rundt sprÅklig formulering av setninger fikk unØdvendig mye oppmerksomhet.
Gruppen hadde fÅtt sitt fØrste mØte med temaet ineffektiv diskusjon.
For Å forhindre tap av tid pÅ nÆringsfattige ting ble gruppen enig om Å innfØre en mØteleder og en referent fra og med neste landsbydag.
MØtelederen skulle ha hovedansvar for Å holde arbeidsplanen samt vÆre ordstyrer i diskusjoner.
Referenten skulle notere fra innsjekk, utsjekk, oppgaver som omhandlet teambuilding og den avsluttende grupperefleksjonen.
Det ble vedtatt Å innfØre en ordning slik at de to rollene gikk pÅ omgang mellom medlemmene.
I ettertid kan det stilles spØrsmÅl ved om en mer effektiv ordning ville ha vÆrt Å ha en fast ordstyrer og en fast referent.
De aktuelle personene ville kunne ha lÆrt av sine feil og gjort justeringer fra gang til gang i motsetning til Å hoppe inn i rollen hver 6. uke.
Valget om rullering stod ved like fordi tapet av effektivitet ble vurdert til lavt i forhold til den verdien hvert medlem ville fÅ ut av Å fÅ prØvd seg i de to rollene.
InnfØringen av ordstyrer og referent var et valg gruppen aldri diskuterte Å endre pÅ. 
Verdien av rollene gjorde seg mest gjeldende nÅr gruppen utfØrte oppgaver som krevde diskusjon, refleksjon og deling av synspunkt.
Gruppen som diskusjonsforum hadde sjeldent problemer med at medlemmer ikke lot andre fÅ ordet, men Å inkludere alle og ikke minst bryte gjennom ved digresjoner var en viktig jobb for ordstyreren. 
Spesielt som tiden gikk ble stemningen naturligvis mer og mer kameratslig slik at tulling og tØysing kom lettere.
Dermed fikk ordstyrer ogsÅ bryne seg pÅ oppgaven i Å skille mellom relevant og urelevant utvikling av et diskusjonstema.

Å skille mellom når en diskusjon spinner inn i en relevant eller urelevant retning er sjelden en lett oppgave.
En av fordelene med å arbeide i en tverrfaglig heterogen gruppe er at personer med ulik erfaring og bakgrunn tenker ulikt når en ide blir presentert.
Gruppen var fra starten klar på at dette mangfoldet skulle verdsettes og at terskelen for å presentere en ufullstendig ide skulle være lav.
Ønsket var at resten av gruppen skulle kunne bygge videre på en grunnide og åpne for nye og interessante vinklinger. 
I samarbeidsavtalen het det: $Alle i gruppen skal vise hverandre respekt. Dette gjennom å lytte, bidra med egne meninger, gi konstruktiv kritikk og bygge videre på andres ideer.$.
For å realisere dette punktet diskuterte gruppen ofte ved å la hvert medlem legge frem sitt synspunkt etter tur.
Å ta en slik runde rundt bordet var sjelden den mest effisiente måten å oppnå et resultat på.
Grundig diskusjon krever mye tid. 
I starten, da det ble gjennomført tidsbegrensede oppgaver, merket gruppen at denne ressursen raskt ble brukt opp.
Likevel kan det argumenteres for at metoden bidro til å oppnå effektivt arbeid ved at slutningene fra en slik diskusjon ga det mest suksessfulle forholdet mellom oppnådd og ønsket resultat.
Med denne synsvinkelen kan det tenkes at effisient og effektivt arbeid må behandles som et kompromiss der begge umulig kan være på topp samtidig.



%%% Effektivitet, trivsel, motivasjon, kvalitet, ineffektivitet

%Kommentere pÅ effektivitet vs trivsel. Se pÅ hvordan balansegangen i trivsel pÅvirker effektivitet og motivasjon for Å arbeide. 
%Effektivitet vs. kvalitet (tidspress, ulike tanker om hva som er viktig, noen i gruppen mere opptatt av tid enn andre)
%Abstrakt: Ineffiktivitet som tema (bÅde faglig og sosial)
%Runde rundt bordet for Å holde alle oppdatert
%Pros and cons ved grundig planlegging
%Hatt mye fokus pÅ samarbeid, at man ikke skal arbeide alene og at alle skulle ha god kontroll pÅ store deler av prosjektet til enhver tid
%For mye sosialt, mindre fokus pÅ arbeidsoppgaver som skal lØses?

Tidspress var et tema gruppen tidlig tok opp i plenum og diskuterte grundig.
Alle medlemmene var svært fornøyd med måten ting ble diskutert på, og likte fokuset på at ting ble gjort grundig.
Under grupperefleksjonen på landsbydag 3 ytrer flere at de er bekymret for om gruppen arbeider for treigt.
Det ble også påpekt at ordningen med ordstyrer og referent fungerer bra, og at det føles som at det har hjulpet på effektiviteten.
Samme dag påpekte Jonas at han foretrekker grundig arbeid selv om det typisk er treigere.
Hele gruppen var i grunn enig i dette, men det måtte selvsagt eksistere en balanse mellom grundighet og progresjon.
I etterkant kan det påpekes at det har vært en fordel for gruppen å ha medlemmer som verdsetter å bruke god tid og medlemmer som husker å iverksette tidsdrister.
 
En forhastet beslutning som gjerne kun er basert på en enkeltpersons ide fremkaller sjeldent tilhørighet og eierskap hos en hel gruppe.
Å føle eierskap til en ide, og føle tilhørighet til en gruppe og det den skal produsere, er alfa og omega for god motivasjon.
At alle medlemmer i en gruppe har et ønske om og er villig til å arbeide for å nå gruppens mål vil i lengden føre til både effektivt og effisient arbeid.
Arbeid uten tilstrekkelig entusiasme er hverken trivelig eller av god kvalitet.





Metoden med å ta runden rundt bordet ble også benyttet for å raskt presentere hva hver enkelt arbeidet med, hvordan det gikk og planen videre.
På den måten var alle gruppens medlemmer oppdatert på hverandres arbeid til en hver tid, og alle følte seg ajour kunnskapsmessig.



%Kommentere disposisjon med deadline
%


























