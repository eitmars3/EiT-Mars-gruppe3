\documentclass[5p]{elsarticle}
\journal{Veileder}	
\usepackage[utf8]{inputenc}
\usepackage[T1]{fontenc} 				
\usepackage[norsk]{babel}				
\usepackage{graphicx}       				
\usepackage{amsmath,amssymb} 				
\usepackage{siunitx}					
	\sisetup{exponent-product = \cdot}      	
 	\sisetup{output-decimal-marker  =  {,}} 	
 	\sisetup{separate-uncertainty = true}   	
\usepackage{booktabs}                     		
\usepackage[font=small,labelfont=bf]{caption}		
\usepackage{minitoc}

\makeatletter
\def\ps@pprintTitle{%
  \let\@oddhead\@empty
  \let\@evenhead\@empty
  \let\@oddfoot\@empty
  \let\@evenfoot\@oddfoot
}
\makeatother

\makeatletter
\setlength{\@fptop}{0pt}
\makeatother

\renewenvironment{abstract}{\global\setbox\absbox=\vbox\bgroup
\hsize=\textwidth\def\baselinestretch{1}%
\noindent\unskip\textbf{Introduksjon}
\par\medskip\noindent\unskip\ignorespaces}
{\egroup}



\setcounter{totalnumber}{5}
\renewcommand{\textfraction}{0.05}
\renewcommand{\topfraction}{0.95}
\renewcommand{\bottomfraction}{0.95}
\renewcommand{\floatpagefraction}{0.35}


\begin{document}

\begin{frontmatter}


\title{Referat - 18.02.2015}

\author[]{Ingelin Garmann, Simen L. Hegge, Karten Olav Kjensmo, \\ Jonas Sandøy Misund, Martin Nordal \& Anna Solveig Julia Testani\`{e}re}
\address{Norges Teknisk-Naturvitenskapelige Universitet, N-7491 Trondheim, Norway}

\begin{abstract}
Referat fra innsjekk, gruppeoppgaver, grupperefleksjon og utsjekk.
\end{abstract}

\end{frontmatter}

\section*{Innsjekk 18. februar 2015}
\subsection*{Karsten}
Skal på tur til København til helga og besøke søster + kompiser. Søstra var på togstasjonen med skytinga, så litt dramatisk. Alt gikk bra!

\subsection*{Anna}
9 dager uten sigaretter. Går bra enn så lenge, men utrolig vanskelig!! Hatt mor på besøk og vært på restaurant. Var på svømmetrening i går :)

\subsection*{Ingelin}
Har det bra. Badet i sjøen i går. Faktisk ikke så kaldt som man skulle tro. Hoppet i badstue etterpå. Har hørt mye på Susanne Sundfør sitt nye album.

\subsection*{Jonas}
Jeg har jobbet mye på Samf. Vinkurs i går. Forskjøv fastelaven til i går siden ingen hjemme hadde tid i helga. Begynt å bli litt uggen i halsen.

\subsection*{Martin}
Måtte ha vinduet åpent når han kom hjem på mandagen, så begynt å bli litt sjuk nå. Var også i Paris i helga. Like i sentrum. Alt i butikkene var dyrt, og det var lite næring i maten (les: kake). Var også i Louvre, skikkelig kult!

\section*{Gruppearbeid om liv \& roveren}
Alle i gruppa jobber på egenhånd og leser om hvilken type liv NASA leter etter på Mars og hvordan de leter der med Roveren. 

Klokka 11 samles gruppa for å snakke om hva vi har funnet.

\subsection*{Liv}
Jonas, Martin og Ingelin leser artikler om liv.

\subsubsection*{Martin \& Ingelin}
Flere artikler om introdukjson og teorier om hvordan liv kan ha oppstått. Mye som er ''enkelt''. Blant annet:

\begin{itemize}
\item Elektriske gnister i gasser/væsker
\item Leireteori
\item Ved undervannsvulkaner
\item En tidligere jord med mindre sollys og liv som oppstår under isen
\item RNA-verden. Man trenger både proteiner og DNA samtidig, men RNA kan ha startet det hele siden det kan ta oppgaven til både proteiner og RNA
\item Molekyler som ''baller på seg''
\item Panspermia
\item Termodynamisk evolusjon
\begin{itemize}
\item Fysiker som ser på livets opprinnelse fra et fysisk perspektiv. Tenker på termodynamikkens 2. lov og at uorden tar overhånd over tid. Har sett på at molekyler som blir samlet og utsatt for stråling kan gi mer energi til omgivelsene enn ellers. Darvinistisk teori er et spesialtilfelle av denne teorien.
\item Mange som har fått øynene opp for denne teorien, kommet litt i vinden i det siste, basert på noen andre teorier fra før, men videreutviklet.
\item De samme fysiske lovene ser ut til å også forklare hvordan virvler i turbulens kopierer seg. Liv lever av negativ entropi
\end{itemize}
\item Har funnet en artikkel om en som ønsker å bruke spillteori til å definere liv, og hvordan økosystemer utvikler seg.
\end{itemize}

\subsubsection*{Jonas}
Sett på NASA sin definisjon av liv.

Hvilke kriterier man bruker når man ser på liv og tanker om hvordan disse begrenser hva man leter etter.

Trenger man en mer grunnleggende teori, hvor liv er en subkategori eller et spesialtilfelle.

\subsection*{Rover}
Karsten og Anna finner artikler om Roveren og leser om den.

Sett på funfacts om Roveren.

\begin{itemize}
\item Reise fra Jorden - 9 mnd
\item Landing - at Roveren landet alene siden det ikke er mulig å forta styringen fra Jorden pga. at kommunikasjon tar tid.
\item Rene fakta
\begin{itemize}
\item Vekt
\item Tilbakelagt distanse
\item Folk som analyserer data
\item Pris - 2.5 mlrd dollar
\end{itemize}
\item Kommunikasjon - tre satelitter rundt både Mars og Jorden. Det er alltid mulighet for kommunikasjon mellom Mars og Jorden
\end{itemize}

Det har vært tre store oppdagelser på Mars i løpet av kort tid
\begin{itemize}
\item Man har funnet spor etter flytende vann
\item Atmosfæren på Mars endrer seg over tid, ganske mye
\item Har funnet forhold på Mars som gir grunnlag for at det kan leve noe der
\end{itemize}

Det er noen deler av Roveren som er aktuelle for oss, noen som ikke er det
\begin{itemize}
\item Laser og mikroskop som kan ta analyser på lang avstånd
\item Alfapartikler som skytes og kan analysere prøver
\item SAM-lab
\item Leting etter vann
\item Nøytronleting etter hydrogen/vann opp til 60 cm ned i bakken
\end{itemize}

Funnet et godt dokument som kan brukes

Framtidige planer
\begin{itemize}
\item Europa - 2018
\item Kina - 2020
\item USA - 2020
\end{itemize}


\section*{Grunnregler til gruppearbeid}

\subsection*{Personlige utfordringer}

\subsubsection*{Karsten}
Min utfordring i forhold til disse grunnreglene er at han en tendens til å avbryte, ta ordet, trekke egne konklusjoner på andres innspill. Har også en tendens til å være utålmodig når andre bruker lang tid til å komme til saken.

Spacer litt ut dersom man tror man har fått med seg det de andre har å si. Går litt videre i eget hode.

\subsubsection*{Ingelin}
Min utfordring i forhold til disse grunnreglene er å bruke eksempler og være konkret. Er kanskje ikke så flink til å tenke over andre mulige tolkninger av det hun selv sier.

Ikke alltid hun er like flink til å høre hele resonementet, blir litt utålmodig før folk har kommet med alle synspunktene de har.

\subsubsection*{Anna}
Min utfordring i forhold til disse grunnreglene er å legge egne tolkninger i andres ord. Har en tendens til å ta generelle utsagn personlig, selv om andre ikke hadde tiltenkt dette. Også litt problemer med språkbarriere når man skal være veldig presis. Kan også hoppe på første og beste løsning, samt ikke forklare helt hvorfor egne forslag er gode. Vil gjerne bli flinkere på å bruke konkrete eksempler i diskusjon med andre.

\subsubsection*{Jonas}
Min utfordring i forhold til disse grunnreglene er at jeg føler jeg har en altfor lang lunte og synes det er vanskelig å ta opp ting som er ubehagelig i plenum. Kan virke selvgod og bedreviter, så det fører til at jeg er ekstra forsiktig med å ta opp konflikter meg andre. Har også en tendens til å gå fort over på å finne direkte løsninger på problemer i stedet for å lage et feller grunnlag først. Det blir viktigere å finne flere mange løsninger, for så å velge ut \'e n enn å skape et felles fundament og bygge en løsning derfra.

\subsubsection*{Martin}
Min utfordring i forhold til disse grunnreglene er å passe på at alle er på samme grunnlag før man kommer til en løsning. Folk kan ha forskjellige betydninger om det samme, så det er viktig å finne ut hva alle mener om tema og hva de kan fra før av.

\subsubsection*{Gruppa}
Gruppa sin utfordring i forhold til disse grunnreglene er at det er mange generelle tanker som er åpning for tolkning. Kan heller være mer konkret med bruke små ord og snakke med litt forbehold. Vi er også veldig påpasselig på at alle skal få viljen sin på samme tidspunkt, at vi er for generelle og vage. Vi burde bli flinke til å ta opp ubehagelige ting slik at det ikke skurrer skikkelig seinere. Gruppa er i sin helhet ganske konfliktsky.

Vi burde også bli flinkere på å ta beslutninger. Være mer effektive i prosessen fra id\'e til beslutning. Kan også bli flinkere til å ha en strukturert prosess rundt å legge fram egen id\'e, hvorfor vi mener det og om andre har innspill.

Burde klare å gå et skritt tilbake fra problemet og løsningen vår for å kunne klare å se om vi har noen flere synspunkter som er usagt.


\section*{Tilbakemelding på grupperefleksjon}
Mange som hadde kommentert på effektiviteten i gruppa.
\begin{itemize}
\item Effektiviteten i gruppa forrige gang var ikke helt på topp for alle sammen. Litt fordi vi har endret arbeidsmetode fra å jobbe mye med tidsfrister til at vi plutselig har frie tøyler. I dag har vi jobbet litt mer konkret med problemstillingen (lest artikler etc.). Brukte litt lang tid på å komme i gang der vi slapp, og var ikke tydelig nok på konkrete mål for dagen.
\end{itemize}

Uenighet om tidspunkt for tilbakemeldinger fra ordstyrer?
\begin{itemize}
\item Mye av det vi sier i refleksjonene våre er forslag, og ikke nødvendigvis personlige meninger og påstander. Gruppa har kommet til enighet om at tilbakemeldinger fra ordstyrer er opp til den personen, og burde komme underveis i arbeidet. Gjerne også i sammenheng med en pause for å ikke bruke opp mye ar arbeidstiden.
\end{itemize}

Hva tenker gruppa om å ha en ordstyrer som har hovedansvar for kontroll.
\begin{itemize}
\item Gruppa synes det er bra å ha en ordstyrer som har hovedansvar for å overholde tidsskjema og dagens arbeidsmål. Mange er ganske nye i dette, så det er naturlig at det ikke går helt perfekt til å begynne med. Det er jo noe vi trenger å lære dette også.
\end{itemize}

\section*{Grupperefleksjon}
\subsubsection*{Jonas}
Sa jeg var uenig med Anna om hva som var viktig å forbedre på gruppa (spørre om andre synsvinkler når vi har en problemstilling). Jeg lurer på hvordan hun tolket det. Hun syntes med en gang at det var ubehagelig fordi hun tenkte at jeg var uenig med alt hun sa, men forstod ut fra de eksemplene jeg hadde at vi stort sett var enig. Jonas burde med andre ord vært tydeligere på hva han var uenig på, og forklart det godt siden det skapte misforståelse.

Til Karsten og Anna: Dere hoppet inn når Ingelin skulle svare på hva hun mente om å ha en ordstyrer. Det var ikke helt heldig.

Gruppa: Synes vi var flinke til å jobbe med de 9 grunnreglene.

Om fesiliteringen: Synes fortsatt at det er veldig mye tidsfrister, og kanskje ikke helt tilrettelagt med tanke på tidsbruk. Føler vi blir pushet videre til litt forhastede beslutninger og diskusjoner.

\subsubsection*{Martin}
Synes de 9 grunnreglene var nyttige.

Om seg selv som leder: Kanskje litt ustrukturert, dårlig tidsperspektiv. Man skal kanskje gi litt fortløpende tilbakemeldinger dersom man ikke har en helt strukturert leder. På fordelingen av arbeidsoppgaver tenke han at det var fint om man tok førstemann til mølla. Synes det er litt vanskelig å balansere det å være ordstyrer og det å delta i samtalen.

\subsubsection*{Karsten}
Om seg selv: Prøver kanskje å beskytte andre som settes under press, og svare for dem. Det er noe han ønsker å slutte med.

Til Martin: Ikke en dårlig gruppeleder, men synes kanskje at han glemte å være det til tider.

Til Jonas: Også mye mer stille enn vanlig. Logisk synes han var referent.

Gruppa: Synes det var kjekt å høre om hva folk hadde å si om det de fant etter arbeidsøkten. Likte denne måten å jobbe på og tror den er bra å ta med videre.

Ordstyrer: Ved beslutninger kan man si at: "Nå gjør jeg som dette, fordi dette, har noen et innspill til avgjørelsen?"

\subsubsection*{Ingelin}
Seg selv: Kanskje litt passiv. Spesielt under prosessarbeidet. Spesielt når vi hadde fasilitator. Antageligvis derfor hun også ble spurt spesielt når fasilitator var der. Bruker litt tid på å tenke seg om å komme på et bra svar på ting. Sikkert derfor Anna og Karsten hoppet inn og ville hjelpe til litt.

Til Martin: Virker som om han tok det til seg på begynnelsen at han hadde en oppgave i løpet av dagen. Dette forsvart ut over dagen.

Til Jonas: Merket også at du var litt mer stille i løpet av dagen. Synes det var greit at han tok opp noen av tingene han hadde skrevet opp for å sjekke om alle var enige i det som stod.

Gruppa: Om prosessoppgaven tok vi det kanskje litt raskt på oppgavefordelingen. Det hadde kanskje litt å si at Martin tok en litt passiv rolle der.

\subsubsection*{Anna}
Om seg selv: Ønsket å få en oversikt over Roveren. Endte opp med å printe ut et veldig godt dokument sammen med Karsten og brukte det. Lurer litt på om hun var effektiv nok, eller ikke. Hadde satt seg mål for dagen.

Om ordstyrer generelt: Synes alle utenom Ingelin har vært litt for tilbakeholden. Burde klare å stå litt mer fram og ta gode avgjørelser. Må klare å sette opp en tydelig og strukturert plan for dagen. Synes at ordstyrer skal kunne ta litt mer beslutninger i løpet av dagen, så burde heller andre bare si ifra dersom de er uenige.

Om Martin: Var flink til å sjekke hvordan det lå an med de ulike under arbeidsøkten. Var også til tider flink til å dra igang gruppa. Har en avslappende og behagelig personlighet som ikke stresser gruppa, det er fint. Det gjorde at hun selv ikke følte noe press underveis.

Om referering: Synes det er en litt stor jobb, og synes at hun mister de som refererer i diskusjonene.

Om Jonas: Skulle heller hatt meningene inn med en gang, i stedet for å sitte bak skjermen og referere.

Om gruppa: Syntes det var fint at hun og Karsten gikk sammen og tok opp det de hadde jobbet med. Lurer litt på hvorfor resten ikke gikk sammen. Hva er det beste? Hva er mest effektivt?

\subsubsection*{Gruppe}
Like før vi gikk sammen hele gruppa for å legge fram arbeidet med prosjektet gikk Anna og Karsten sammen og tok en felles gjennomgang av sitt tema (Rover). Ingelin, Martin og Jonas som hadde et annet tema (liv) gjorde ikke dette. Karten og Anna sin presentasjon var tydeligere og det virket som om de hadde en bedre oversikt enn de andre hadde for sitt tema. Gruppa er enige i at vi burde legge en bedra plan for slike arbeidsøkter i framtiden og at det kan være lurt å jobbe sammen så mye som mulig, slik som vi ble enige om når vi utarbeidet samarbeidskontrakten. Neste arbeidsøkt skal vi prøve å ha en tydeligere struktur, ha lagt klare mål for dagen og følge disse opp underveis.

I løpet av dagen har Jonas skrevet referat fra det som har blitt sagt i plenum (innsjekk, presentasjon av arbeidsrunde, prosessarbeid om grunnregler, grupperefleksjon og utsjekk). Anna synes referenten skriver for mye i løpet av dagen, og faller litt ut av gruppa (kanskje spesielt Jonas som bruker å delta aktivt i diskusjoner). Hun synes det også er litt bortkastet tid å referere så mye. Jonas mener han selv får brukt referentrollen til å holde seg litt tilbake og ikke avbryte så ofte som vanlig. Karsten sier han er uenig med at referatene er for lange og mener det er vitkig med gode referater til seinere. Anna presiserer at hun mener det hovedsaklig er prosjektarbeidet som ikke trenger å refereres. Dette kommer av at de som jobber med en del av prosjektet allerede har notater og kilder. Det blir rett og slett unødvendig. Jonas og Karsten sier de er enige i det Anna sier, og trekker tilbake at de mener referent skal skrive så mye som til nå. Anna sier hun gjerne skulle vært litt mer presis når hun la fram hva hun mente. Dette er en av de tingene gruppa er enige om at de må jobbe med; klarhet i argumenter, og gode forklaringer. En tydelig og strukturert ordstyrer kommer nok til å kunne hjelpe med dette, samt at alle i gruppa har fokus på de områdene vi ønker å forbedre (bruke spesifikke eksempler, forsklare bakgrunnen for arguemntene og intesjonen med løsningene). Vi kommer også til enighet om at det ikke er nødvendig å referere fra prosjektarbeidet, av den grunnen Anna presenterte.
\end{document}