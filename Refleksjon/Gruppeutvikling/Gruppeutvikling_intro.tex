Gruppens utvikling er delt i tre hovedtrender: kommunikasjon, effektivitet og konflikter.
Ved gjennomgang av grupperefleksjoner er det spesielt disse tre aspektene som går igjen i situasjonene som har oppstått.
\begin{itemize}
\item Kommunikasjon på grunn av at gruppen har diskutert en del om konsekvensene av at enkelte medlemmer prater mer/mindre enn andre, samt diskusjon rundt definering av ord og uttrykk.
\item Effektivitet siden gruppen har slitt litt med å nå tidsfrister.
\item Konfilkt fordi gruppen har hatt få til ingen konflikter.
\end{itemize}

Før disse trendene diskuteres grundigere må begrepet gruppe defineres. 
\emph{"En gruppe er definert som en samling av personer, gjenstander som ved et tilfelle eller i en eller annen hensikt står, befinner seg, er kommet i en slik nærhet av hverandre at de i forhold til omgivelsene danner et slags hele"} \cite{prosjekteringsledelse}.
For å kunne snakke om en sosial gruppe må følgende krav være oppfylt \cite{orgorg}:
\begin{itemize}
\item Gruppen forener medlemmene i et nettverk av sosiale relasjoner
\item Gruppemedlemskapet kan angis
\item Gruppen må ha minst ett mål
\item Alle i gruppen må oppfatte seg som medlem
\end{itemize}
I gruppen eksisterer normer som foreskriver visse handlinger og forbyr andre.
Et team er en unik form for gruppering, det er likevel viktig å påpeke at grensene mellom gode arbeidsgrupper og effektive team er flytende \cite{orgorg}.
Assmann (2008) har følgende definisjon på team: \emph{"Team er en liten, flerfaglig sammensatt gruppe med et felles formål der medlemmene opplever felles ansvar for at de oppnår resultater"}.
Det som ifølge Assmann i særdeleshet kjennetegner et team, er felleskap og samspill mot felles mål, mens i arbeidsgrupper kan de sosiale koblingene innad være løsere og mer generelt fokusert på kommunikasjon og samordning.
Med andre ord er intern samhørighet, kohesjon og gruppefølelse mer viktig for et team enn for andre former for arbeidsgrupper.
Kollektive prestasjoner teller mer enn individuelle, og felles skjebne rangerer over personlige agendaer \cite{orgorg}.