Før gruppas utvikling kan beskrives må begrepet gruppe defineres. En gruppe er definert som en samling av personer, gjenstander som ved et tilfelle eller i en eller annen hensikt står, befinner seg, er kommet i en slik nærhet av hverandre at de i forhold til omgivelsene danner et slags hele [1]. 
For å kunne snakke om en sosial gruppe må følgende krav være oppfylt [2]:
\begin{itemize}
\item Gruppen forener medlemmene i et nettverk av sosiale relasjoner
\item Gruppemedlemskapet kan angis
\item Gruppen må ha minst et mål
\item Alle i gruppen må oppfatte seg som medlem
\end{itemize}
I gruppen eksisterer normer som foreskriver visse handlinger og forbyr andre
Et team er en unik form for gruppering, det er likevel viktig å påpeke at grensene mellom gode arbeidsgrupper og effektive team er flytende [2].
Assmann (2008) har følgende definisjon på team: «Team er en liten, flerfaglig sammensatt gruppe med et felles formål der medlemmene opplever felles ansvar for at de oppnår resultater».
Det som ifølge Assmann i særdeleshet kjennetegner et team, er felleskap og samspill mot felles mål, mens i arbeidsgrupper kan de sosiale koblingen innad være løsere og mer generelt fokusert på kommunikasjon og samordning.
Med andre ord er intern samhørighet, kohesjon og gruppefølelse mer viktig for et team enn for andre former for arbeidsgrupper.
Kollektive prestasjoner teller mer enn individuelle, og felles skjebne rangerer over personlige agendaer [2].