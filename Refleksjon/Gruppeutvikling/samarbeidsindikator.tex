
Hovedtemaet i denne sekjsonen er å undersøke det sosiale samværet gruppen har hatt, samt diskutere hvilke effekter dette har hatt på gruppens produktive kapasitet. Her vil gruppen støtte seg til faglitteratur, og se på forhold som særpreger effektive grupper[simens kilde]

[Kilde fra organisasjon og organisering, se Simens kilde]

[kilden] lister opp en rekke sosiale kvaliteter [cite denne listen som simen har i sin del] som er gavnlige for gruppesamarbeid:  åpen kommunikasjon, gjensidig tillit, sosial støtte, utnyttelse av individuelle forskjeller
Alle disse punktene kan kommenteres på i gruppens sosiale utvikling.
Enighet om mål ble etablert tidlig i form av samarbeidskontrakten, som sikret felles enighet om spesifikke mål. Ønsket om en A har preget gruppens innsatser, og strikse regler om oppmøte og leveringsfrister har påvirket profesjonalitet innad i gruppen. 
Åpen kommunikasjon er noe gruppen scoret lavt på tidlig i øvelsen[siter Samarbeidsindikatoren]. 


Individ forskjeller: En god sosial miljø gjorde at medlemmer ble mer klar over hvilke personlighetstyper som finnes i gruppen, informasjon som er svært viktig i et samarbeid



Prosessrapport: Sosiale inntrykk

En god sosial kommunikasjon og en god tone i gruppen hjelper å skape et positiv og trygt arbeidsmiljø. Medlemmer får lyst å samarbeid og føler seg mer komfortabel å jobbe sammen. En god sosial miljø gjorde at medlemmer ble mer klar over hvilke personlighetstyper som finnes i gruppen, informasjon som er svært viktig i et samarbeid. Fra begynnelsen av prossessen er hver enkle medlemmer av gruppen vært inkluderende og dedikerte gruppe medlemmer. 

-Derfor ble for eksempel, innsjekk og utsjekk alltid tatt på alvor. I disse “øyblikket”, har individer fått mulighet til å vise tillit til hverandre og inkludere resten av gruppen i livet sitt. 
Det ble etterhvert en slags tradisjonen, hvor vi følger hverandres prosjekt og ikke minst utvikling i andre området av livet. Crossfit, barn, maraton, sigarett, Bahamas, mat og Samfundet er noen eksempler vi kan ta som viser det punktet. 

-At gruppen har hatt så god sosial kontakt er medfører til både positiv og negativ konsekvenser. Å konsentrere seg på det positiv gjøre at medlemmer av gruppen ble motivert fra begynnelse til å gjøre en god jobb, og ikke skuffet andre medlemmer av gruppen siden det kunne ha skapet konflikt og ødelegge den god tone. En naturlig konsekvens av det er medlemmer tør kanskje ikke å ødelegge den positiv sosial kommunikasjon ved å gi hverandre tilbakemeldinger eller kritikk i sammenheng til arbeidet. Gruppen la merke til at få konflikter har skjedd i løpet av semester og at det kan være en av grunnene. Viktig å legge merke til at gruppen fikk ikke noe bra resultat fra første “gruppe test” på kategorien “ærlig og direkte “. Den god tone gjorde også at gruppen i begynnelse (⅔ første landsbydag)  har slittet med tid og struktur og tapte derfor på effektivitet. 

-Det er ingen tvil om at sosial god kommunikasjon går hånd i hånd med hvordan medlemmer jobber. Siden alle har vært svært dediktert i jobben sin og fått etterhvert tillit av gruppen gjorde at gruppen hadde en mer avslappende forhold til forsinkelse (Jonas), fravær (Karsten) og frist forsinkelse (Simen,..). 

-Inkluderende medlemmer og positiv forhold til faget gjorde at medlemmer føles eierskap og ansvar i samarbeidet. Alle medlemmer blir sett og ingen har fått mulighet til å melde seg ut. Gruppen er for eksempel  bestemt seg å  dra på restaurant for å ferie sammen når rapporter blir leverte. Det gjøre at medlemmer har knyttet sammen og dele ansvar.  Å fokusere på at alt skulle gjøres sammen i sosial sammenheng har hatt positiv konsekvens i vårt faglig samarbeidet. (Eksempel av Jonas i konferans [forelesningen?] som satt alene, Karsten og Anna gjorde plass sånn at de kunne alle 3 sitte sammen, Crossfit dag, vi slette å ta ut en medlem av gruppe )









<Kort beskrivelse av den sosiale atmosfæren: Avslappet, tendens til vitsing, tendens til distractions under gruppemøter, tendens til å kalle til fokus når det trengs>
etc
Spørsmål:
Har vi hatt positiv innflytelse av at alle var villige til å gjøre en grundig jobb (gå etter A’en, samarbeidskontrakten)?
Var alle det fra start av, eller kom viljen til å gjøre en grundig jobb som følge av god teamfølelse?
“Veldig avslappet” pros and cons
Pros: Har dette gjort at folk ikke blir arrige over forseintkomminger og fristoverskridelser, i og med at alle på teamet føler seg trygge på at en innsatts blir gjort? 
Pros: Har dette gjort det lettere å håndtere uenigheter, siden folk har hatt god vilje og “likt” hverandre?
Har det vært nyttig som ‘sosial judo’ i å kunne snakke om konflikter i tidlig fase, lenge før kokepunkt?
Cons: Har det kostet oss verdiful tid, spesielt i den første halvdelen av faget? Var progresjon i arbeidet jevnlig handicappet? Gikk det ofte veldig sakte fremover? Brukte gruppen mye tid på unødvendig sosial prat? Kunne dette føre til frustrasjon blandt noen medlemmer på gruppen, og mindre hos andre, derved skape mulighet for konflikt (noen jobber hardt, andre prater bare tull)
Innflytelse av god sosial stemning på innsjekk/utsjekk: bryr folk seg om hverandre (referer til fagets beskrivelse av forskjellige team-egenskaper). F eks teamet bryr seg om de individuelle og hvilken overføringsverdi dette har til arbeidet.
 Anna røykestopp/maraton, Simen + sønn Odin, Ingelin CF progresjon, Martin mat (:))) /DJing/tech ansvar på singsaker studenthjem), Karsten mountainbike/Bahamastur, Jonas sine eventyr på samfundet 

<mangler spesifikke referanser til referat; vi må få samlet disse et sted>
