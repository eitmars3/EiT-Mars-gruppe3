En samarbeidsavtale ble skrevet i starten av prosessen for å oppnå en felles oppfatning innad i gruppen om hvordan gruppearbeidet skulle utføres, og for å bestemme hvilke krav gruppens medlemmer kunne stille til hverandre. Gruppen var nå over i Tuckmans andre fase i gruppeutviklingen; utprøvingsfasen (storming). Det viktige i denne fasen var å etablere et system for gruppen samt felles mål og forventninger. 

Bush et al. (2012) peker på flere punkt som kjennetegner effektive grupper [2]:

\begin{itemize}
	\item Gruppens mål. Gruppens mål og de oppgavene som må utføres er oppfattet og akseptert av gruppemedlemmene.
	\item Åpen kommunikasjon. Deltakerne gir fritt uttrykk for følelser og meninger, og alle lytter aktivt. Alle er innstilt på å endre oppfatning på grunnlag av fakta og meninger som de andre gruppemedlemmene legger fram.
	\item Gjensidig tillit. Gjensidig tillit og åpen kommunikasjon henger nøye sammen. Medlemmene tør være seg selv og spiller ikke kunstige roller.
	\item Sosial støtte. Alle gir hverandre støtte, oppmuntrer, forståelse, oppmerksomhet og anerkjennelse.
	\item Utnyttelse av individuelle forskjeller. Medlemmenes særpreg og særpregede kompetanse må komme til sin rett, og dette må avbalanseres mot hensynet til samhørighet og felles måloppnåelse.
	\item Fleksibelt lederskap. Gruppen må selv bli enige om hvilket lederskap som er ønskelig.
\end{itemize}

Samarbeidsavtalen inneholder flere viktige punkt som legger til rette for effektiv gruppearbeid i henhold til faktorene overnfor. 

Punkt nummer 6 i samarbeidsavtalen (se vedlegg xxx) sier som følger: “Alle har et medansvar for å jobbe mot en prosess- og prosjektrapport som tilfredsstiller kravene til en A”. Dette viser at gruppen har satt egen mål som er oppfattet og akseptert av alle medlemmene.

Samarbeidsavtale punkt 10 sier som følger: “Gruppen skal ha rullerende møteleder og sekretær for hver landsbydag. Møteleder har hovedansvar for å overse arbeidsplan og trivsel samt å være ordstyrer. Sekretær refererer fra innsjekk og utsjekk.” Som det presiseres av Busch et al. (2012) bør det være et fleksibelt lederskap som gruppen selv har blitt enig om for å legge til rette for effektivt gruppearbeid.

Samarbeidsavtale punkt 7 og 8 sier henholdsvis: “Gruppa skal sørge for at alle fagfelt er representert i prosjektet” og “Gruppen skal drive kunnskapsutveksling gjennom diskusjon og drøfting av faglig innhold, samt utfordre hverandre til å gå utenfor den faglige komfortsonen.” Som det fremgår av disse punkene legges det til rette for utnyttelse av individuelle forskjeller som videre legger til rette for effektiv gruppearbeid.

Samarbeidsavtale punkt nummer 11 sier som følger: “Alle i gruppen skal vise hverandre respekt. Dette gjennom å lytte, bidra med egne meninger, gi konstruktiv kritikk og bygge videre på andres idéer.” Dette er et svært viktig punkt. Som det fremgår av Busch et al. sine punkter for effektiv gruppearbeid legges det her til rette åpen kommunikasjon, gjensidig tillit og sosial støtte. Det er også mulig å se på dette et prestasjonsmotivasjonsperspektiv. Busch et al. (2012) definerer fire forskjellige type personer som ut fra ønsket om suksess og frykten for nederlag.

\begin{center}
	\begin{tabular}{|l|c|c|c|c|}
		\hline
		Persontyper & A & B & C & D \\ \hline
		Ønsket suksess & X & X & - & - \\ \hline
		Frykt for nederlag & X & - & X & - \\ \hline
	\end{tabular}
\end{center}

Ved å studere de fire forskjellige personene i tabellen (person A til D) og hvordan kombinasjonen av de forskjellige motivene påvirker prestasjonen. A ønsker sterk fremgag, men har samtidig sterk prestasjonsangst. B har samme sterke tendens til å gå løs på oppgaven, men er klart mindre redd for konsekvensene dersom han eller hun ikke lykkes. Både C og D mangler det sterke ønske om å lykkes. C er i tillegg redd for å dumme seg ut, mens D stiller seg nokså likegyldig til det hele. Det kommer derfor klart fram av tabellen at person B klarer seg best i en prestasjonssituasjon [2]. Punkt 11 i samarbeidsavtalen har som formål å fjerne frykten for nederlag slik at alle gruppemedlemmene skal ha mulighet å være person B (ønsket suksess må selvfølgelig også være tilstede).

Som det diskuteres her ble grunnlaget for et effektivt gruppearbeid lagt tidlig i prosjektet. Ved å lage samarbeidsavtalen ble flere viktige faktorer fastslått og akseptert av alle gruppedeltakere. Det er viktig å understreke at gruppemedlemmene ikke nødvnedigvis hadde gjensidig tillit på dette tidspunktet, men grunnlaget er lagt for å oppnå dette. Samme med frykten for nederlag, den er ikke nødvendigvis fjernet ved dette tidspunktet, men grunnlaget for å oppnå dette er lagt.
