\subsubsection{Konfliktskyhet}

%% OBS ADD TEORI %%

\paragraph{Konflikter i gruppesituasjonen} Konfliktskyhet er en naturlig egenskap å ha, spesielt i kulturen som hersker i det moderne Norge.
Konflikter er ubehagelige, stressende, krever store mengder energi og resulterer ofte i mellommenneskelig skade.
Forhold som har vært gode kan bli dårlige og forhold som har vært dårlige kan bli helt ødelagt.
Det er alltid noe på spill utover det diskuterte når en konflikt skjer, og spesielt i gruppeatmosfærer kan en enkelt konflikt ha varige skader lenge etter konflikten fant sted. 
En konflikt kan bygges opp sakte eller skje spontant, og konfliktskyhet fremkommer i begge tilfeller. 
De fleste velger konfliktene sine med omhu. Noen ganger må en konfrontasjon finne sted for at man skal føle seg respektert, eller for å behandle en situasjon som oppfattes som er urettferdig. 
Noen situasjoner er ikke verd de potensielle negative følger, og folk finner derfor andre måter og forholde seg til konflikten på. 

\paragraph{Konflikter i gode sosiale miljø}
Dette kan man observere i EiT, hvor konfliktskyhet kan ha mange årsaker, og resultere i mange forskjellige utfall.
Den gode sosiale tonen innad i gruppen kan ha vært en barriere for direkte kommunikasjon - noe som kommenteres på i forbindelse med den første samarbeidsindikatoren som finnes på side \pageref{samarbeidsindikator1}. 
Her er scoren for gruppens evne til å være ærlig og direkte var meget lav.
Dette endret seg over tid, men er allikevel indikativt på at overdiplomatiske ordvalg har begrenset kommunikasjon.
Det var, tidlig i prosjektet, en stor frykt for å skape unødvendig dårlig sosial trivsel. 
Kan gruppen ha blitt enda mer konfliktsky av den gradvis økende sosiale atmosfæren, eller har dette gjort det enklere å snakke direkte?
Gruppen har lurt på om frykten for sosial friksjon har vedvart, eller om den har blitt gradvis mitigert av et styrket sosialt bånd. Dette viser samarbeidsindikator 2 og 3 viser til, men det kan fortsatt være en blanding av faktorene. 

\paragraph{Tid som hjelpende faktor}

%% MÅ HA BEVIS %%

Det gikk en uke mellom hver landsbydag, og en uke er god tid til å fordøye inntrykk, tenke over saker og la emosjoner falle til ro. 
Til sammenligning med EiT intensiv, hvor arbeidet foregår over meget kort tid, er en ukes pusterom mellom hver hele landsbydag god tid til å fatte ro.
Gruppen mener det har hatt en positiv effekt på konflikthåndteringen. 
Det har skjedd flere ganger at et medlem har gått fra landsbydagen med et negativt inntrykk av en situasjon, hvor angrep og forsvar har blitt vekslet jevnt, og at dette negative inntrykket mitigeres når medlemmet får tid til ro.
Ved neste ukes møte er den negative situasjonen behandlet, og begge parter er roligere og vennligere innstilt.

\paragraph{Innordning vs. underkastelse}
En av de meste effektive våpnene i konfliktprevensjon er å være komfortabel med å innordne seg når situasjonen nå krever det. 
Det er en forskjell mellom innordning og underkastelse. 
Ved underkastelse forstås en mishandling: et medlem føler seg såpass intimidert eller urespektert at en sterk og legitim mening blir holdt tilbake. 
Underkastelse er skadelig, og tilsier at det er virkemidler på spill som ikke baserer seg på rolig og resonnerende debatt. Ved innordning forstås en diplomatisk handling: den innordnende har presentert sine argumenter og disse overbeviser ikke.
Den innordnede har forstått at h*n er i mindretall og at en kontinuasjon av debatten ikke vil tjene noe godt formål.
Den innordnede har altså gode følere for både hvor viktig diskusjonen er, og for resten av gruppens attitude til diskusjonen.
Det har hersket mye diskusjon på gruppen, som i generelle trekk har valgt å fremgå mere metodisk og grundig i alle avgjørelser. Ved å diskutere jevnlig, har enkelte medlemmer vært nødt til å innordne seg flertallets meninger. 

\paragraph{Preventiv vs. reaktiv konfliktshåndtering}
Konfliktløsning er en av de sentrale konseptene i EiT.
Mangeler på større konflikter i gruppen har vært en bekymring for fagets læringsmål og for karakteren. 
Faglig skulle det helst ha forekommet en større konflikt eller to, slik at gruppens medlemmer fikk prøvd seg på reaktiv konfliktløsning. 
Ved å forholde seg diplomatisk mitt i en ubehagelig gruppesituasjon og teste ut virkemidlene beskrevet forskjellige steder i pensum. 
Dette fikk gruppen ikke gjort på stor skala.
Derimot har det blitt demonstrert kapasitet for prevantivt konfliktshåndtering. 
Konflikter har blitt behandlet før de fikk utartet seg til noe større. 
Gruppens samarbeid har vært rettet mot en preventiv holdning, og denne kan sies å ha utspillet seg med suksess. 

