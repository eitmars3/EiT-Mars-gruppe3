\subsubsection{Konfliktskyhet}

\paragraph{Konflikter i gruppesituasjonen}. Konfliktskyhet er en ganske naturlig egenskap å ha, spesielt i kulturen som hersker i det moderne Norge. Konflikter er ubehagelige, stressende, krever store mengder energi og resulterer ofte i mellommenneskelig skade. Forhold som har vært gode kan bli dårlige, forhold som har vært dårlige kan bli helt ødelagt. Det er altid noe på spill utover det diskuterte når en konflikt skjer, og spesielt i gruppeatmosfærer kan en enkelt konflikt ha varige skader lenge etter konflikten fant sted. En konflikt kan bygges opp sakte eller skje spontant, og konfliktskyhet fremkommer i begge tilfeller. De fleste velger konfliktene sine med nogenlunde omhu - noen ganger må en konfrontasjon finne sted for at man skal føle seg respektert, eller for å behandle en situasjon man mener er urettferdig. Noen situasjoner, derimot, er ofte ikke værd de potensielle negative følger, og folk finner derfor andre måter og forholde seg til konflikten på. 

\paragraph{Konflikter i gode sosiale miljø}
Dette kan man også observere i EiT, hvor konfliktskyhet kan ha mange årsaker, og resultere i mange forskjellige utfall. Den gode sosiale tonen innad i gruppen kan ha vært en barriære for mere direkte kommunikasjon - noe som kommenteres på i forbindelse med den første samarbeidsindikatoren\ref{samarbeidsindikator1}, hvor scoren for gruppens evne til å være ærlig og direkte var meget lav. Dette endret seg over tid, men er allikevel indikativt på at overdiplomatiske ordvalg har begrenset kommunikasjon. Det var, tidlig i prosjektet, en stor frykt for å skape unødvendig dårlig sosial tone. Kan gruppen ha blitt ennå mere konfliktsky av den gradvist økende sosiale atmosfæren, eller har dette gjort det enklere å snakke direkte? Gruppen har lurt på om frykten for sosial friksjon har vedvart, eller om den har blitt gradvist mitigert av et styrket sosialt bånd, som samarbeidsindikator 2 og 3 viser til, eller om det er en blanding av disse to faktorene. 

\paragraph{Tid som hjelpende faktor}
En faktor som er værd og nevne er tid. Det gikk en uke mellom hver landsbydag, og en uke er god tid til å fordøye inntrykk, tenke over saker og la emosjoner falle til ro. Til sammenligning med EiT intensiv, hvor arbeidet foregår over meget kort tid i høy intensitet, er en ukes pusterom mellom hver hele arbeidsdag god tid til å fatte ro. Gruppen mener det har hatt en positiv effekt på konflikthåndteringen. Det har skjedd flere ganger at et medlem har gått fra landsbydagen med et negativt inntrykk av en situasjon, hvor angrep og forsvar har blitt vekslet jevnt, og at dette negative inntrykket mitigeres når medlemmet får tid til ro. Ved neste ukes møte er den negative situasjonen behandlet, og begge parter er roligere og vennligere innstilt enn de var ved slutt av siste dag.

\paragraph{Innordning vs. underkastelse}
En av de meste effektive våpnene i konfliktprevansjon er å være komfortabel med å innordne seg når situasjonen nå kaller på det. Det er en forskjell mellom innordning og underkastelse som må defineres. Ved underkastelse forstås en mishandling: et medlem føler seg såpass intimidert eller urespektert at en sterk og muligvis legitim mening blir holdt tilbake. Underkastelse er skadelig, og tilsier at det er virkemidler på spill som ikke baserer seg på rolig og resonnerende debatt. Ved innordning forstås en diplomatisk handling: den innordnende har presentert sine argumenter og disse overbeviser ikke; den innordnede har forstått at h*n er i mindretall og at en kontinuasjon av debatten ikke vil tjene noe godt formål; den innordnede har altså gode følere for både hvor viktig diskusjonen er, samt gode følere for resten av gruppens attitude til diskusjonen. Det har hersket mye diskusjon på gruppen - som, i generelle trekk har valgt å fremgå mere metodisk og grundig i alle avgjørelser, og derfor ofte hadde mye å diskutere - og jevnlig har enkelte medlemmer innordnet seg flertallets meninger eller overbevisende argumenter. 

\paragraph{Prevantiv vs. reaktiv konflikthåndtering}
Konfliktløsning er en av de sentrale konseptene i EiT. Mangler på større konflikter på gruppen har vært en bekymring for fagets læringsmål og karakteren. Faglig skulle det helst ha forekommet en større konflikt eller to, slik at gruppens medlemmer fikk prøvet seg på reaktiv konfliktløsing; på å forholde seg diplomatisk mitt i en ubehagelig gruppesituasjon og teste ut virkemidlene beskrevet forskjellige steder i pensum. 
Dette fikk gruppen ikke gjort på stor skala. Derimot mener vi at vi har demonstrert stor kapasitet for prevantivt konflikthåndtering - en slags sosial judo for å stoppe konflikter før de fikk utartet seg til noe større. Gruppens samarbeid har vært rettet mot en prevantiv holdning, og denne kan sies å ha utspillet seg med suksess. 

