\subsection{Konflikter}

En av hovedmålene til EiT er at gruppen skal utvikle sin samspillkompetanse\cite{eitlaeringsmaal}, og dette tar to former.
For det første må gruppen kunne ta hverandres ideer videre, og for det andre må gruppen kunne håndtere de konflikter og uenigheter som kan oppstå underveis.
Når man ikke har et etablert faghierarki eller en arbeidsstruktur fastlagt på forhånd med sjefer og ansatte kan det bli grobunn for konflikt. Det finnes ingen ekstern myndighet - alle starter på lik plan, og derfra må gruppen både lære å jobbe sammen og faktisk jobbe sammen mot målet. Denne sekjsonen av rapporten vil omhandle konflikter i gruppen, eller manglen på dem, og prøve og analysere hvorfor det gikk som det gikk på gruppen. Her vil vi se nermere på begrepet konflikt som gruppen har forstått det; undersøke de elementære faktorene som inngikk i gruppeligningen, og samtidig stille spørsmålstegn ved faktorene for å prøve og se hvor stor en grad de hadde effekt. Hvordan har konflikter utspilt seg? Hvordan kan konflikter i gruppen karakteriseres? Har manglen på konfrontasjonelle konflikter betydd at gruppen har vært heldig med personligheter, motivert og flink til å samarbeide, konfliktsky, eller en blanding av disse?
