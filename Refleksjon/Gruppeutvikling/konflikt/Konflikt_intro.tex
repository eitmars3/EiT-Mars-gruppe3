\subsection{Konflikter}

Et av hovedmålene til EiT er at gruppen skal utvikle sin samarbeidskompetanse\cite{eitlaeringsmaal}, og dette tar to former.
For det første må gruppen kunne ta hverandres id\'{e}er videre, og for det andre må gruppen kunne håndtere de konflikter og uenigheter som kan oppstå underveis.
Når man ikke har et etablert faghierarki eller en arbeidsstruktur fastlagt på forhånd med administrasjon, kan det bli grobunn for konflikt. 
I dette gruppearbeidet finnes det ingen ekstern myndighet - alle starter på likt plan, og derfra må gruppen både lære å arbeide sammen og jobbe mot målet. 
Denne seksjonen av rapporten vil omhandle konflikter i gruppen - eller som vi skal se - mangelen på dem, og prøve og analysere hvorfor det gikk som det gikk.
Her vil vi se nærmere på begrepet konflikt slik som gruppen har forstått det.
Det vil si, undersøke de elementære faktorene som inngikk i gruppearbeidet, og samtidig stille spørsmålstegn ved i hvor stor grad de hadde effekt.
Hvordan har konflikter utspilt seg?
Hvordan kan konflikter i gruppen karakteriseres? 
Har mangelen på konfrontasjonelle konflikter betydd at medlemmene i gruppen har vært konfliktsky? Eller har vi vært motiverte og flinke til å samarbeide? Eller har vi vært heldige med personligheter, eller er det en blanding av alle disse?
\\