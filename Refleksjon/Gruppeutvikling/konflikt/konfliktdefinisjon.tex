\subsubsection{Konfliktdefinisjon}

\paragraph{Begrepsforståelse}
\emph{Konflikt} er et ord som kan dekke alt fra en uenighet om hvor mye sukker som skal i kaffen, til moderne krig i stor skala. Konflikt tildeles muligvis en mere aggressiv og intens betydning i folketale, men i EiT rammeværket vil vi være gankse presise i vår forståelse av begrepet. Vi mener det er en forskjell mellom uenighet og konflikt, og at de ligger en følelsesladet forskjell mellom en redelig debatt og en kamplignende konflikt. Saklig uenighet bidrar til nytenkning og fremdrift, mens konflikt fører til stagnasjon, nedsatt produktivitet og mistrivsel\cite{ledernytt}. Partene i en konflikt går i hver sine skyttergraver, og det blir slutt på utforskning, dialog og konstruktivt samarbeid. I denne seksjonen vil vi henvise oss til faglitteraturen innen ledelse og administrasjon, og undersøke årsaker som bidrar til konflikter på gruppen. Dette vil stille gruppens konfliktsituasjon i et klarere lys. 


\paragraph{Hvorfor går det galt?}




Liste av årsaker til konfliktfyllte grupper\cite[p.~253]{orgorg}

\begin{itemize}

  \item Gruppemedlemmene tar stilling og nekter å gå inn på kompromiss.
  \item Gruppemedlemmene viser utålmodighet overfor hverandre.  
  \item Ideer blir angrepet før de er blitt ferdig fremlagt.
  \item Gruppemedlemmene angriper hverandre personlig.
  \item Gruppemedlemmene anklager hverandre for ikke å forstå hva som er viktig.
  \item Gruppemedlemmene oppfatter bare fragmenter av det de andre sier.  

\ldots
\end{itemize}