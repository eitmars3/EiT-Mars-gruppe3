\subsubsection{Konfliktdefinisjon}

\paragraph{Begrepsforståelse}
\emph{Konflikt} er et ord som kan dekke alt fra en uenighet om hvor mye sukker som skal i kaffen, til moderne krig i stor skala. 
Konflikt tildeles en aggressiv og intens betydning i folketale. I EiT rammeverket er forståelsen av begrepet presist. Det er en forskjell mellom uenighet og konflikt, og det ligger en følelsesladet forskjell mellom en redelig debatt og en kamplignende konflikt. Saklig uenighet bidrar til nytenkning og fremdrift, mens konflikt fører til stagnasjon, nedsatt produktivitet og mistrivsel\cite{ledernytt}. Partene i en konflikt går i hver sine skyttergraver, og det blir slutt på utforskning, dialog og konstruktivt samarbeid. I denne seksjonen henvises det til faglitteraturen innen ledelse og administrasjon, og årsaker som bidrar til konflikter undersøkes. Dette vil stille gruppens konfliktsituasjon i et klarere lys. 

\paragraph{Hvorfor går det galt?}

Faglitteraturen viser til en liste av årsaker til konfliktfylte grupper\cite[p.~253]{orgorg}

\begin{itemize}

  \item Gruppemedlemmene tar en stilling og nekter å gå inn på kompromiss.
  \item Gruppemedlemmene viser utålmodighet overfor hverandre.  
  \item Ideer blir angrepet før de er blitt ferdig fremlagt.
  \item Gruppemedlemmene angriper hverandre personlig.
  \item Gruppemedlemmene anklager hverandre for ikke å forstå hva som er viktig.
  \item Gruppemedlemmene oppfatter bare fragmenter av det de andre sier.  
\end{itemize}

\paragraph{Konflikter - en god ting?}

Konflikter er en naturlig del av samspillet mellom mennesker. De er en følge av at mennesker ønsker å stå for det de mener, og at det er spillerom for å gi uttrykk for ulike synspunkter. De kan være nedbrytende og lammende, men kan også medføre engasjement, vekst og forbedringer. Konflikter er derfor ikke gode eller dårlige i seg selv. Spørsmålet er hvordan man forholder seg til dem, og om de håndteres på en destruktiv eller konstruktiv måte\cite{helsekompetanse}. Det kan også forekomme en sakte oppbygging av småirritasjoner, som kan ta utløp i en større konflikt. Denne større konflikten kan resultere i motgang, ingenting, eller fremgang - og den siste av disse er ofte også fulgt av et utslipp av negativitet, som kan ha en beroligende effekt videre. 

\paragraph{Faglig uenighet}
Faglig uenighet skjedde regelmessig, og gruppen har hatt engasjerende diskusjoner rundt bordet. Målet i disse diskusjonene har altid vært å gjøre oppgaven eller samarbeidet bedre. Det ble ikke noe konflikt på et personlig nivå. Disse uenigheter, uansett hvor engasjerte de kan være, er mye lettere å takle for medlemmer enn en konflikt. Det finnes en visshet om at det skjer for gruppens beste interesser; tonen er fortsatt på ett profesjonell nivå, og ingen personlige angrep eller usaklige argumenter har blitt fremført. Dette kan kalles en konflikt etter definisjonsrammene, men det er likevel viktig å undersøke kontrasten mellom en følelsesladet konflikt, hvor stolthet og ære er i sentrum, kontra en objektiv, redegørende konflikt hvor utfall og logikk er i sentrum. 






