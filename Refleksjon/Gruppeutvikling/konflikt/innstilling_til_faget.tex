\subsubsection{Positiv innstilling til faget}
 
Fra begynnelsen av prossessen har hvert enkelt medlem av gruppen vært positivt innstilt og dedikert, noe som gjorde at en hel avdeling av potensielle konflikter ikke oppstod.

Gruppen har ikke opplevd motstand om faget fra enkelte individ, hverken på projekt eller prosess delen. Medlemmene delte at de hadde en del forventninger til faget, og at de hadde lyst å lære. Uansett hvor ukomfortabelt eller vanskelig det var å skrive personlig refleksjon i den første fasen, gjorde gruppen en god insats og prøvde å få noe ute av det. Denne attituden, rettet mot fagets øvelser og rammeværk som helhet, og at slike typer oppgaver ble tatt seriøst med en gang fra alle medlemmene, bidro til at gruppen unngikk noen form for motivasjonskonflikt.

Gruppen ble oppmerksom allerede når kompetanstrekanten[ADD REF NÅR INGELNI/SIMEN HAR ADDET TREKANT] ble laget at ingen hadde ekspertise innenfor hverken biologi (prosjekt) eller humaniora (prosess). Dette gjorde at gruppen ble bekymret for å stille ganske svakt på faglig kunnskap innenfor landsbyens hovedtema. Gruppen har i etterkant forstått at dette har også hjulpet i samarbeidet, da alle har startet fra samme sted rent faglig. Dette gjorde at det ikke ble noe faglig dominans fra et individ, noe som kunne ha skapet konflikt eller forstyrret balansen i arbeidsfordeling. Som konsekvens av dette var medlemene ydmyke, åpne til forskjellige løsninger, og alle fikk mulighet til å bidra likt på oppgavene. Det som var sett som en svakhet i begynnelsen, ble en av grunnene til manglen på større konflikter på gruppen.

Gruppen har også gjort en god jobb rundt konfliktprevansjon i oppstartsfasen. Medlemmene satte struktur og regler med virkemidler som arbeidskontrakten og individuelle mål.  Medlemmene har vist respekt for målet med EiT og har vært flinke til å definere oppgaver, ansvar og mål. Gruppen ble for eksempel enig om å ha som mål å jobbe mot en A; formuleringen i arbeidskontrakten er at alle skal levere en \emph{innsats} tilstrekkelig for toppkarakter. At forventninger til både faget og hverandre ble avklart tidlig gjorde at konflikt rundt selve strukturen ikke har oppstått. Gruppemedlemmene kunne ikke anklage hverandre for ikke å forstå oppgavene eller målet. Det har aldri vært et problem å møte utenom onsdager, til privat jobbing der dette trengtes. Inkluderende medlemmer og et positivt forhold til faget gjorde at medlemmer følte eierskap og ansvar i samarbeidet. Alle medlemmer blir sett og ingen har fått mulighet til å melde seg ut. Som konsekvens av disse faktorer gruppen ikke har hatt noen interessekonflikter.
