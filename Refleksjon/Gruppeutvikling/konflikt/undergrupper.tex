\subsubsection{Undergrupper}

Gruppen ville undersøke om det hadde formet seg noen undergrupper/klikker underveis i arbeidet, og om disse kunne ha innflytelse på arbeidet.
Ved starten var det ingen som kjente til hverandre, så det ble en egalitær start.
Dette ble bygget videre på med en inkluderende gruppekultur. Det ble også lagt mye fokus på at alle skulle få sagt det de ville i innsjekk- og utsjekksrutinen. 
Disse faktorene har medført mindre muligheter for undergrupper. 
Det vil ikke kunne sies at det ble dannet undergrupper. 
Enkelte gruppemedlemmer interagerte bedre med hverandre enn andre, for eksempel Karsten og Martin som både har felles fagbakgrunn og skolevei ble godt sammensveiset. Anna og Ingelin hadde lignende arbeidsstil og jobbet mer sammen i par enn andre. 
Det kan spørres om gruppen hadde dannet klikker og undergrupper om prosjektet hadde foregått over enda lengere tid.
Var dette en toårskontrakt og ikke et semesterfag, ville det kanskje ha oppstått sterkere kliker som kunne ha påvirket arbeidet. 
\\