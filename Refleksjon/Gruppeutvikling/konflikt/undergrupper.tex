\subsubsection{Undergrupper}

Gruppen ville undersøke om det hadde formet seg noen undergrupper/kliker underveis i arbeidet, og om disse kunne ha innflytelse på arbeidet. Ved starten var det ingen som kjente til hverandre, så det ble en meget egalitær start. Dette ble bygget videre på med en meget inkluderende gruppekultur, samt mye fokus på at alle skulle få sagt det de ville og den godt benyttede innsjekk- og utsjekksrutinen som er beskrevet over. Disse faktorene har medført mindre muligheter for undergrupper. Det ble allikevel tydelig etter tid at hvisse medlemmer hadde en mere naturlig samtale- og arbeidsflyt underveis. Andre faktorer, som for eksempel at Karsten og Martin både har felles fagbakgrunn og skolevei, gjorde at noen medlemmer ble litt mer sammensveiset sosialt enn andre. Ingelin og Anna hadde ligendne arbeidsstil, og jobbet kanskje mere sammen enn andre par. 
Det vil allikevel ikke kunne sies at det ble dannet undergrupper; dette er mere overfladiske kommentarer på hvordan enkelte gruppemedlemmer interagerte bedre med hverandre enn andre. Det kan spørres om gruppen hadde dannet kliker og undergrupper om prosjektet foregikk over lenger tid - var dette en toårskontrakt og ikke et semesterfag, ville det kanskje ha oppstått sterkere kliker som kunne ha påvirket arbeidet. 