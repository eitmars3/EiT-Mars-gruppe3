\subsection{Effektivitet}
På engelsk eksisterer det to begrep som måler grad av suksess ved et resultat. 
I \textit{Oxford Dictionary of English} er \textit{effectivity (effektivitet)} definert som i hvilken grad et produsert resultat gjenspeiler det ønskede målet, mens \textit{efficient work (effisient arbeid)} omhandler hvor mye ressurser som har blitt brukt på å skape det produserte resultatet. 

Å utvikle et effektivt gruppearbeid er ansett som et hovedmål i karakterisering av en velfungerende gruppe. 
Som det inngår i klassifiseringen av en gruppe, må den arbeide mot et felles mål. 
Oftest vil dette målet også være tidsfestet slik at en viss progresjon er krevd for hver gang gruppen møtes.

Allerede 2. landsbydag ble det enighet om at arbeidet med EiT skulle finne sted de gitte onsdagene, og ikke ellers, så langt det lot seg gjøre. 
Å gjennomføre et effektivt gruppearbeid ble derfor fra starten av et viktig fokus. 

I denne seksjonen skal det diskuteres hvordan og hvorvidt gruppen arbeidet effektivt og effisient.
Gruppens utgangspunkt blir kartlagt gjennom kompetansetrekanten og samarbeidavtalen.
Deretter skal det drøftes rundt effektivitet kontra kvalitet og hvordan arbeidsplanlegging og arbeidsfordeling ble tilpasset dette.
Til slutt vil denne seksjonen diskutere effektivitet kontra trivsel i gruppen og hvordan disse punktene påvirker hverandre. 

\subsubsection{Kompetanse}
En av de første prosessoppgavene gruppen skulle utføre var å kartlegge gruppens kompetanse.
Ordet kompetanse er et komplekst begrep og det er derfor vanlig å dele det inn i fire underbegreper \cite{orgorg}:
\begin{enumerate}
\item \textit{Kunnskaper} som handler om teoretisk innsikt.
\item \textit{Ferdigheter} som omhandler evnen til å utøve en bestemt atferd.
\item \textit{Evner} omhandler potensialet til å tilegne seg kunnskaper eller ferdigheter.
\item \textit{Holdninger} tar for seg de stabile og organiserte oppfatninger, følelser og handlingsintensjoner i forhold til objekter eller saker av sosial art.
\end{enumerate}
Kartleggingen ble utført med en øvelse kaldt \emph{"kompetansetrekant"}.
Kompetansen ble evaluert innenfor tre kategorier: teoretisk kunnskap (innenfor sitt fagfelt), praktisk erfaring (tar for seg ferdighetene) og personlige trekk (holdninger).
Evner kan være vanskelig å vurdere i en slik øvelse og vil dessuten delvis inngå i personlige trekk.

Gruppemedlemmene laget først individuelle trekanter for så å presentere disse for gruppen.
Til slutt samarbeidet gruppen om å lage en trekant som representerte gruppens samlede kompetanse.
Det ble fort avdekket at dette var en faglig heterogen gruppe, noe som kan være en stor fordel.
Stor bredde i kompetanse, innsikt og personlighet fører til at problemstillinger blir grundigere behandlet.
Mangfoldet gir grunnlag for en dynamisk diskusjon der kritiske spørsmål oftere blir stilt og meninger belyst.
Til tross for stor bredde i kompetansen hadde fortsatt alle gruppemedlemmene \emph{"teknisk bakgrunn"}, noe som ga en viss form for homogenitet.
Dette er viktig fordi altfor heterogene grupper har en tendens til å slite med sterke kontroverser, og konflikter har en større tendens til å oppstå (dette behandles senere i rapporten).
Gruppemedlemmene bør derfor ha et felles grunnsyn eller verdisyn som kan fungere som en rettesnor ved konflikter.
Kompetansetrekanten avdekket at alle gruppemedlemmene var interessert i å utvikle samarbeidsegenskaper og samarbeidskunnskaper.
Dette kan ha vært et ønske som har fungert som et felles verdisyn. 

Kompetansetrekanten gjorde ikke bare gruppen oppmerksom på at den var faglig heterogen, men også hvilken kompetanse de forskjellige gruppemedlemmene satt inne med.
Det er viktig å avdekke dette tidlig i prosessen for å kunne legge til rette for et effisient arbeid.
Hvis ikke gruppemedlemmene er klar over de andres styrker og svakheter, må dette erfares, noe som kan koste dyrebar tid.
Ved å definere kompetansen til deltakerne på et tidlig stadium, var det lettere å utdele individuelt arbeid, samt avgjøre hvem som bør arbeide med hvem på de forskjellige underaktivitetene.
Dette er en av faktorene som påpekes som viktig innenfor effisient og effektivt gruppearbeid; utnyttelse av individuelle forskjeller.
Medlemmenes særpreg og særpregede kompetanse må komme til sin rett, og dette må avbalanseres mot hensynet til samhørighet og felles måloppnåelse \cite{orgorg}. 

Et viktig spørsmål som må stilles i forbindelse med en gruppes effektivitet er hvorvidt gruppens medlemmer innehar den nødvendige kompetansen for å løse oppgaven.
Det gjelder både faglig kompetanse og kompetanse til å få gruppeprosessen til å fungere.
Kompetansetrekant viste at gruppens medlemmer hadde liten til ingen innsikt i biokjemi (som tross alt er en viktig del av oppgaven).
\textit{"En studentgruppe kan aldri levere et godt prosjekt hvis den ikke tar seg tid til faglig fordypning"} \cite{orgorg}
Det var ingen av gruppedeltakerene som meldte seg på denne landsbyen på grunn av interessen for biokjemi.
De fleste hadde en interesse for romfart, og ordene \emph{"skattejakt"} og \emph{"Mars"} kan derfor ha påvirket valget av fag.
Det å sette av tid til å kunne forstå grunnleggende biokjemi ble likevel sett på som en viktig faktor.

\subsubsection{Regler for effektivitet}
En samarbeidskontrakt er en skriftlig avtale mellom medlemmene i en gruppe som stadfester punkter vedrørende både praktisk og sosialt samarbeid og ambisjonsnivå. 
Hensikten med avtalen er at gruppen som helhet får diskutert og kommet til enighet om rammene og målet med samarbeidet, og ikke minst at det finnes noe konkret å henvise til ved eventuelle konflikter og brudd på kontraktspunkter. 


Ambisjonen om å gjennomføre et effektivt gruppearbeid ble gjenspeilet i flere punkter. 

En forutsetning for å kunne samarbeide som gruppe er at gruppen faktisk møtes. 
Prosedyre ved fravær ble dermed behandlet i flere punkter. 
Gruppen var enig om at både forsinkelse og fravær skulle varsles, men at det ikke var nødvendig med kontraktsbestemt sanksjon hvis slikt inntraff. 
Ved planlagt fravær skulle vedkommende bli tildelt en oppgave for å jobbe inn tapt tid. 
Å behandle fravær som kontraktsfestede punkt bidro til at medlemmene hadde dem i bakhodet selv om de ikke ble referert til aktivt.
Fravær og forsinkelser ble i stor grad meldt fra om, og gruppen var klar over at både Karsten og Simen ville ha planlagt fravær i forbindelse med påskeferien. 
På den måten kunne gruppen på forhånd anmode om å få utsatt prosesssamtalen med læringsassistenter og landsbyleder til en onsdag hvor alle gruppemedlemmene var til stede.
Fravær og forsinkelser ble aldri et stort problem for gruppen, og det ble ikke ansett som nødvendig å innføre strengere rammer for fravær under revisjonen av kontrakten midtveis i arbeidsperioden. 

Neste punkt i samarbeidsavtalen lød enkelt og konkret at alle skal gjøre en ærlig innsats i løpet av landsbydagen.
Enda verre enn at gruppemedlemmer er fraværende, er passive og kanskje til og med motarbeidende deltakere.
På det tidspunktet kontrakten ble forfattet hadde gruppemedlemmene enda ikke fått innsikt i hverandres arbeidsmentalitet og arbeidsinnstilling.
Dette punktet, som direkte omhandlet ønsket om effektivitet, var derfor i større grad ment som motivasjon og oppfordring, enn som aksjon på en inntruffet situasjon.
I ettertid har gruppen reflektert over at alle medlemmenes innstilling og pågangsmot for å gjøre godt arbeid har vært utelukkende av den positive sort.
Flere punkter som slår ned på ikke-bidragende deltagelse har derfor aldri vært aktuelt.

I grupperefleksjonen fra 2. landsbydag står det at det ble en del drøfting rundt semantikken i samarbeidsavtalen. 
Diskusjon i plenum rundt språklig formulering av setninger fikk unødvendig mye oppmerksomhet.
Gruppen hadde fått sitt første møte med temaet ineffektiv diskusjon.
For å forhindre tap av tid på næringsfattige ting ble gruppen enig om å innføre en møteleder og en referent fra og med neste landsbydag.
Møtelederen skulle ha hovedansvar for å holde arbeidsplanen samt være ordstyrer i diskusjoner.
Referenten skulle notere fra innsjekk, utsjekk, oppgaver som omhandlet teambuilding og den avsluttende grupperefleksjonen.
Det ble vedtatt å innføre en ordning slik at de to rollene gikk på omgang mellom medlemmene.
I ettertid kan det stilles spørsmål ved om en mer effektiv ordning ville ha vært å ha en fast ordstyrer og en fast referent.
De aktuelle personene ville kunne ha lært av sine feil og gjort justeringer fra gang til gang i motsetning til å hoppe inn i rollen hver 6. uke.
Valget om rullering stod ved like fordi tapet av effektivitet ble vurdert til lavt i forhold til den verdien hvert medlem ville få ut av å få prøvd seg i de to rollene.
Innføringen av ordstyrer og referent var et valg gruppen aldri diskuterte å endre på. 
Verdien av rollene gjorde seg mest gjeldende når gruppen utførte oppgaver som krevde diskusjon, refleksjon og deling av synspunkt.
Gruppen som diskusjonsforum hadde sjeldent problemer med at medlemmer ikke lot andre få ordet, men å inkludere alle og ikke minst bryte gjennom ved digresjoner var en viktig jobb for ordstyreren. 
Spesielt som tiden gikk ble stemningen naturligvis mer og mer kameratslig slik at tulling og tøysing kom lettere.
Dermed fikk ordstyrer også bryne seg på oppgaven i å skille mellom relevant og urelevant utvikling av et diskusjonstema.

Å skille mellom når en diskusjon spinner inn i en relevant eller urelevant retning er sjelden en lett oppgave.
En av fordelene med å arbeide i en tverrfaglig heterogen gruppe er at personer med ulik erfaring og bakgrunn tenker ulikt når en id\'{e} blir presentert.
Gruppen var fra starten klar på at dette mangfoldet skulle verdsettes og at terskelen for å presentere en ufullstendig id\'{e} skulle være lav.
Ønsket var at resten av gruppen skulle kunne bygge videre på en grunnide og åpne for nye og interessante vinklinger. 
I samarbeidsavtalen het det: \textit{Alle i gruppen skal vise hverandre respekt. Dette gjennom å lytte, bidra med egne meninger, gi konstruktiv kritikk og bygge videre på andres ideer.}.
For å realisere dette punktet diskuterte gruppen ofte ved å la hvert medlem legge frem sitt synspunkt etter tur.
Å ta en slik runde rundt bordet var sjelden den mest effisiente måten å oppnå et resultat på.
Grundig diskusjon krever mye tid. 
I starten, da det ble gjennomført tidsbegrensede oppgaver, merket gruppen at tidsressursen raskt ble brukt opp.
Likevel kan det argumenteres for at metoden bidro til å oppnå effektivt arbeid ved at slutningene fra en slik diskusjon ga det mest suksessfulle forholdet mellom oppnådd og ønsket resultat.
Med denne synsvinkelen kan det tenkes at effisient og effektivt arbeid må behandles som et kompromiss der begge umulig kan være på topp samtidig.

\subsubsection{Tidspress - Effektivitet vs. kvalitet}
Tidspress var også et tema gruppen tidlig tok opp i plenum og diskuterte grundig.
Alle medlemmene var svært fornøyd med måten ting ble diskutert på, og at konsensus ble tilstrebet under alle avgjørelser.
En forhastet beslutning som gjerne kun er basert på en enkeltpersons id\'{e} fremkaller sjeldent tilhørighet og eierskap hos en hel gruppe.
Å føle eierskap til en id\'{e}, og føle tilhørighet til en gruppe og det den skal produsere, er alfa og omega for god motivasjon.
At alle medlemmer i en gruppe har et ønske om og er villig til å arbeide for å nå gruppens mål vil i lengden føre til både effektivt og effisient arbeid.
Arbeid uten tilstrekkelig entusiasme er hverken trivelig eller av god kvalitet.
Som definert tidligere betyr god effektivitet at det er stort samsvar mellom oppnådd og ønsket resultat.
Så lenge det er satt høye kriterier til det ønskede produktet en gruppe vil oppnå, vil effektivitet dermed også gjenspeile kvalitet.
Effisient arbeid derimot trenger ikke å gjenspeile god kvalitet.
Ved å bruke så få ressurser som mulig, for eksempel ikke bruke tid på diskusjon, ikke prøve å finne alternative id\'{e}er, ikke utvikle id\'{e}er videre, men rett og slett gå for den raskeste, enkleste og minst krevende id\'{e}en, vil man også få produsert et sluttresultat.
Kvantitativt har gruppen oppnådd det samme, nemlig fått skrevet to rapporter, men kvaliteten på innholdet vil være dårligere enn om flere ressurser hadde blitt benyttet.

Under grupperefleksjonen på landsbydag 3 ytret flere at de var bekymret for om gruppen arbeidet for tregt.
Det ble også påpekt at ordningen med ordstyrer og referent fungerte bra, og at det føltes som at denne aksjonen hadde forbedret tidsbruken.
Samme dag påpekte Jonas at han foretrekker grundig arbeid selv om det typisk er tregere.
Hele gruppen var i grunn enig i dette, men det måtte selvsagt eksistere en balanse mellom grundighet og progresjon.
I etterkant kan det påpekes at det har vært en fordel for gruppen å ha medlemmer som verdsetter å bruke god tid og medlemmer som husker å iverksette tidsfrister.
Ved at ulike medlemmer har innehatt en form for motsatte roller med tanke på tidsbruk, har den ønskede balansen mellom det å ta seg tid til grundig diskusjon og samtidig ha stor nok progresjon blitt ivaretatt. 
Her kommer det tydlig fram en av fordelen med en heterogen gruppe.

Utfordringen med å drive aktivt samarbeid er at det er tidkrevende.
Spesielt når ord skal formuleres til setninger, er det lite effisient å jobbe flere sammen.
Løsningen ble å tilpasse samarbeidsmetoden etterhvert som arbeidsoppgavene endret seg.
I startfasen av prosjektoppgaven var det lett å samarbeide hele gruppen i plenum.
Da gjennomførte gruppen brainstorming, id\'{e}myldring og diskusjon rundt mulige problemstillinger i fellesskap.
I ettertid er det grunn til å tro at denne gode og åpne dialogen, villigheten til å diskutere og få innspill på sine id\'{e}er og utforme en problemstilling alle følte eierskap til, la grunnlaget for at samarbeidet fungerte godt også når arbeidet av praktiske grunner ble mer individuelt og tidspresset mer reelt.

\subsubsection{Arbeidsfordeling og planlegning}
Etter at de første prosessøvelsene var på plass, begynte gruppen smått å jobbe med id\'{e}er og id\'{e}myldring til oppgaven.
Oppgaven som fokuserte på roveren Curiousity ble valgt, denne ble delt inn i tre hoveddeler for å tilfredsstille problemstillingen:
\begin{enumerate}
\item Litt astronomisk historie som bygger opp til nåtidens leting
\item Hvordan vi leter i dag med hovedfokus på roveren
\item Hvor og hvordan denne letingen kan bli i framtiden
\end{enumerate}
For å oppnå en viss form for fordeling mellom gruppedeltakerne ble det avgjort i refleksjon fra 6. landsbydag at to gruppemedlemmer skulle ha hovedansvaret på hver del.
Altså to gruppemedlemmer med hovedansvar på del en, to med hovedansvar på del to og de siste to med hovedansvar for del tre.
For å sikre godt gruppearbeid (ingen skulle jobbe på bare en del) skulle medlemmene rullere på hvem som jobbet på de forskjellige delene.
I refleksjonen ble det påpekt at det kunne oppstå forvirring og misforståelser hvis de to som skrev på del en den ene dagen skiftet til noe annet neste dag.
De to nye som da kom på del en ville ha problemer med å sette seg inn i hva som var funnet ut og hva som var planen videre.
Aksjonen for å unngå dette var at det alltid skulle være en med hovedansvar på hver del.
Det skulle altså rulleres på å rullere.

På grunn av misnøye med arbeidet 9. landsbydag bestemte gruppen seg for at det skulle utarbeides en grundigere disposisjon og arbeidsplan neste landsbydag.
Den gamle planen (to personer på hver del, med rullerende mønster) ga ikke effisient nok arbeid.
En ny plan ble derfor laget med spesifikke punkter, ansvarsområder og tidsfrister.
Dette ble da en plan med delmål og milpæler for prosjektet.
Det finnes en del teori rundt planlegging av arbeid og fastsettelse av mål.
Det er viktig at målene har en viss kvalitet, hvis ikke vil forvirring lett oppstå, noe som igjen kan føre til lavere motivasjon.
I denne sammenheng listes det opp en del punkter som mer eller mindre bør innfris for å sikre god kvaltiet \cite{prosjekteringsledelse}:
\begin{enumerate}
\item \textit{Tydelig.} Må oppfattes likt og entydig av de som skal nå målene
\item \textit{Utfordrende.} Skal fremtvinge læring og tvinge frem det beste i alle involverte
\item \textit{Realiserbart.} Målet må kunne nås
\item \textit{Målbart.} Må kunne måles ved slutt og underveis
\item \textit{Akseptert.} De som skal nå målet må også være motivert for målet
\item \textit{Tidfestet.} Målet skal være nådd innenfor et visst tidspunkt
\end{enumerate}
Målene oppfylte mer eller mindre alle disse punktene.
Målene var tydelig, de utfordret gruppemedlemmene til å gå utenfor sin faglige komfortsone (i henhold til samarbeidsavtalen) og de var selvfølgelig realiserbare (prosjektet ble jo ferdig).
Arbeidet er målbart, gruppen vet hvor langt arbeidet er kommet.
Målene var akseptert av alle gruppemedlemmene siden disse ble utformet i plenum.
Som nevnt hadde også målene spesifikke tidspunkter de skulle være ferdig. 

Gruppemedlemmene var svært fornøyde med den nye inndelingen.
I grupperefleksjonen fra 10. landsbydag presiserer flere av deltakerne at de følte at denne planen var nødvendig.
Ingelin kommenterer for eksempel at hun nå føler at gruppen kommer i mål.
Det var en merkbar økning i motivasjon og dermed også effektivitet.

Den nye planen la også til rette for at gruppemedlemmene kunne styre arbeidet bedre selv ved å for eksempel jobbe hjemme i tillegg til de oppsatte onsdagene.
Gruppen uttrykte likevel at denne arbeidsmetoden strider med samarbeidskontrakten om å ha så mye samarbeid som mulig.
Ideelt sett ønsket gruppen å arbeide så lite som mulig med individuelt arbeid.
Alle gruppemedlemmene hadde lyst til å utnytte at dette var et fag spesielt tilrettelagt for samarbeid.
Det ferdige produktet skulle representere kunnskap hele gruppen hadde lært seg i løpet av arbeidstiden, og ikke seks individuelle menneskers fagfelt sydd mer eller mindre sammen til ett.
Det ble da viktigere å benytte metoden med å ta runden rundt bordet for å raskt presentere hva hver enkelt arbeidet med, hvordan det gikk og planen videre.
På den måten var alle gruppens medlemmer oppdatert på hverandres arbeid til en hver tid, og alle kunne føle seg a jour kunnskapsmessig.

Midtveisframføringen 12. landsbydag gjorde gruppen oppmerksom på at rapporten foreløpig hadde en litt rotete struktur.
Oppgaven dekker mange tema som skaper bekymringer for om gruppen klarer å holde den røde tråden gjennom oppgaven.
Det ble enighet om at første utkast (all tekst) skulle være ferdig til 13. landsbydag slik at gruppen kunne gå gjennom rapporten samt finpusse på overgangene mellom de forskjellige delene.
Resultatet av disse planene var at gruppen jobbet mer effektivt.
Tidsplanen ble i hovedsak fulgt selv om ikke alt var ferdig innenfor de gitte tidsrammer. 

Den 13. landsbydagen diskuterte gruppen i plenum hvilke tema prosessrapporten burde inneholde og utformet en grov disposisjon for strukturen.
Påfølgende landsbydag la Simen fram et nytt forslag til disposisjon, og gruppen tok seg tid til å diskutere det nye forslaget.
Ingen av gruppens medlemmer hadde noen gang skrevet en slik rapport tidligere, og skjønte fort at det var mange mulige måter å gjøre det på.
Landsbydag 14 ble avsluttet uten at gruppen hadde tatt en avgjørelse på disposisjonen.
I utsjekken fra samme dag sa Jonas at han var svært glad for at gruppen tok denne diskusjonen så grundig.
Den gjorde at han ble skikkelig engasjert.
Simen satte pris på at gruppen ikke tok hans nye id\'{e} for god fisk, men stilte spørsmål ved hans oppsett og diskuterte det videre.
Ingelin var også glad for at så mange id\'{e}er var lagt fram og at gruppen hadde vurdert mange muligheter, men samtidig var hun bekymret for at tiden begynte å bli knapp.
Gruppen endte opp med å møtes en dag utenom oppsatt tid, og fikk utarbeidet en detaljert disposisjon med innslag av mange id\'{e}er.
I ettertid viste det seg at selve arbeidet med å ferdigstille prosessrapporten gikk raskt i og med at disposisjonen var detaljert og alle medlemmene var inneforstått med innholdet.
Her er det enkelt å se at gruppen hadde tatt lærdom av oppsettet til prosjektrapporten.
Likevel førte denne grundigheten til at gruppen var nødt til å møtes utenfor oppsatt tid, og dermed ikke klarte å gjennomføre ønsket om å bli ferdig ved å kun arbeide på onsdager.
Forhåpentligvis resulterte grundigheten i at hvert medlem fikk et konkret mål på hva som skulle inkluderes i rapporten slik at det videre arbeidet ble mer effisient enn hva det hadde blitt med den første løse planen. 

Som seksjon \ref{beslutningsprosess} antyder, kan det være en balanse mellom effektivitet og effisiens.
Det at gruppen generelt sett har latt alle medlemmers synspunkter komme fram, har gått på bekostning av effektiviteten.
De jevnlige justeringene av hvilke gruppemedlemmer som skulle fordype seg i hvilke emner, viser hvordan gruppen gjentatte ganger har tilpasset seg for å takle denne utfordringen.

\subsubsection{Effektivitet vs. trivsel}
En utfordring for alle grupper er å balansere faglig fokus og sosiale avbrekk.
En nyoppstartet gruppe der alle medlemmene er ukjente vil av naturlige årsaker kaste bort liten tid på interne vitser, tullete digresjoner og andre gøyalheter.
Slik var også denne gruppen i starten. Det tar alltid tid før man finner sin plass, plasserer andre og blir komfortabel i en gruppe. 

Det tok ikke mange landsbydagene før gruppen utviklet seg til å ikke bare være en faglig samarbeidsgruppe, men også en gruppe som trivdes sammen sosialt. 
Allerede etter de første tegnene til at praten satt løsere og medlemmene begynte å lirke av seg morsomheter, ble det stilt reflekterende spørsmål ved utviklingen av triveligheten.
Ville det trivelige miljøet i gruppen ha negativ effekt på det faglige arbeidet?
Var trivselen nødvendig for å holde motivasjonen oppe? 
Hvor går skillet mellom syklubb og faggruppe med god trivsel?
Det finnes ingen fasitsvar, og det er vanskelig å bedømme fordi alle andre scenario enn det faktiske blir svært hypotetiske.
Selvsagt kan enhver gruppe bli mer tidseffektiv ved å forby alt annet enn faglig arbeid, men det er også grunn til å tro at det faglige arbeidet blir bedre av at også trivselen er stor.
En annen vinkling av saken er om det er mulig å oppnå god trivsel dersom det ikke i tillegg presteres faglig.
Det er ikke utenkelig at humøret til en gruppe ville ha blitt preget av at de faktiske målene ikke ble nådd.
Fra og med landsbydag 6 valgte gruppen å arbeide på et eget grupperom kun med sporadiske besøk av læringsassistenter og landsbyleder.
Gruppen ble dermed flink til å arbeide uten "oppsyn".
Denne egenskapen tok gruppen med seg da den senere også ble nødt til å møtes utenom oppsatte arbeidstider.
Det har aldri vært diskutert at en landsbydag har bestått av for mye tull og fritidsprat som har gått på bekostning av framdrift i arbeidet.
Gruppen som helhet er enig om at trivselen mellom medlemmene utelukkende har ført til å bedre arbeidsmiljøet og ønsket om å prestere bra sammen.
