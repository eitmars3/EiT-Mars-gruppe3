\subsection{Effektivitet}

%Effektivitet (sakte progresjon, brukt tid p≈ ≈ bli ferdig, grundig vs tidskrevende)
%Hovedansvar: Simen & Ingelin

%Vil se p≈ hvordan gruppa har jobbet effektivt med prosjektrapporten og prosessoppgavene. Starte med ≈ se p≈ hvordan vi laget problemstilling og disposisjon, fordelte arbeidsoppgaver og konkret jobbet med dem. Under prosessoppgavene kan det v∆re viktig ≈ ta opp kommunikasjonsmÿnstre i tillegg.

%%% FERDIG %%%

%Konkret: Samarbeidskontrakt (spesifikk respect for andres forslag)

%Konkret: Rullering av gruppeleder/referent, hvilke innsikter dette har gitt og hvordan dette har p≈virket gruppen

%Lagt mye fokus p≈ konsensus og at vi skal ha en felles forst≈else for arbeidet (SITRA kan nevnes som eksempel)

%%% GJENST≈R %%%

%Konkret: Se p≈ trekanten: Trivsel, leveranse, l∆ring, prÿve og plassere oss. Gjenspeiles i samarbeidsavtalen
%Konkret: Utviklingen av samarbeidsindikatoren
%Abstrakt: Hvordan utspiller det seg n≈r gruppen jobber
%Abstrakt: Arbeidsfordeling, hvordan skal folk f≈ brukt kunnskapen sin
%Frav∆r av aktivt system for ‚Äúto-do‚Äù (f eks Trello)
%Ang≈ende prosessrapport: Fÿrste gang vi skriver en s≈nn rapport, s≈ det er vanskelig ≈ komme i gang
%Abstrakt: Pros and cons av jobbing for et m≈l - milep∆ler, deadlines, disposisjon med tidsfrister
%Pros and cons ved grundig planlegging

PÂ engelsk eksisterer det to begrep som mÂler grad av suksess ved et resultat. 
I $Oxford Dictionary of English$ er $effectivity$ definert som i hvilken grad et produsert resultat gjenspeiler det ¯nskede mÂlet, mens $efficiency$ omhandler hvor mye ressurser som har blitt brukt p  skape det produserte resultatet. 

≈ utvikle et effektivt gruppearbeid er ansett som et hovedmÂl i karakterisering av en velfungerende gruppe. 
Som inngÂr i klassifiseringen av en gruppe, m den arbeide mot et felles mÂl. 
Oftest vil dette mÂlet ogs vÊre tidsfestet slik at en viss progresjon er krevd for hver gang gruppen m¯tes. 
I det gruppearbeidet som omtales i denne rapporten var oppgaven, som nevnt tidligere,  forfatte en populÊrvitenskapelig artikkel med tema $Biologisk skattejakt p planeten Mars$, gjennomf¯re en rekke oppgaver innen teambuilding samt reflektere over disse og til slutt presentere gruppeprosessen i en egen rapport. 
Dette arbeidet skulle gjennomf¯res i l¯pet av 15 uker hvor gruppen var planlagt og m¯tes hver onsdag fra 08.30\endash16.00. 
Allerede 2. landsbydag ble det enighet om at arbeidet med EiT skulle finne sted de gitte onsdagene, og ikke ellers, s langt det lot seg gj¯re. 
≈ gjennomf¯re et effektivt gruppearbeid ble derfor fra starten av et viktig fokus. 

%Samarbeidskontrakt
En samarbeidskontrakt er en skriftlig avtale mellom medlemmene i en gruppe som stadfester punkter vedr¯rende bÂde praktisk og sosialt samarbeid og ambisjonsnivÂ. 
Hensikten med avtalen er at gruppen som helhet fÂr diskutert og kommet til enighet om rammene og mÂlet med samarbeidet, og ikke minst at det finnes noe konkret  henvise til ved eventuelle konflikter og brudd p kontraktspunkter. 

Rammeverket for gruppearbeidet ble utarbeidet s tidlig som 2. landsbydag med mulighet for revisjon gjennom hele arbeidsperioden. 
Hovedemnene var leveranse, lÊring og trivsel.
Ambisjonen om  gjennomf¯re et effektivt gruppearbeid ble gjenspeilet i flere punkter. 

En forutsetning for  kunne samarbeide som gruppe er at gruppen faktisk m¯tes. 
Prosedyre ved fravÊr ble dermed behandlet i flere punkter. 
Gruppen var enig om at bÂde forsinkelse og fravÊr skulle varsles, men at det ikke var n¯dvendig med kontraktsbestemt sanksjon hvis slikt inntraff. 
Ved planlagt fravÊr skulle vedkommende bli tildelt en oppgave for  jobbe inn tapt tid. 
≈ behandle fravÊr som kontraktsfestede punkt bidro til at medlemmene hadde dem i bakhodet selv om de ikke ble referert til aktivt.
FravÊr og forsinkelser ble i stor grad meldt fra om, og gruppen var klar over at bÂde Karsten og Simen ville ha planlagt fravÊr i forbindelse med pÂskeferien. 
P den mÂten kunne gruppen p forhÂnd anmode om  f utsatt prosesssamtalen med lÊringsassistenter og landsbyleder til en onsdag hvor alle gruppemedlemmene var til stede.
FravÊr og forsinkelser ble aldri et stort problem for gruppen, og det ble ikke ansett som n¯dvendig  innf¯re strengere rammer for fravÊr under revisjonen av kontrakten midtveis i arbeidsperioden. 

Neste punkt i samarbeidsavtalen l¯d enkelt og konkret at alle skal gj¯re en Êrlig innsats i l¯pet av landsbydagen.
Enda verre enn at gruppemedlemmer er fravÊrende, er passive og kanskje til og med motarbeidende deltakere.
PÂ det tidspunktet kontrakten ble forfattet hadde gruppemedlemmene enda ikke fÂtt innsikt i hverandres arbeidsmentalitet og arbeidsinnstilling.
Dette punktet, som direkte omhandlet ¯nsket om effektivitet, var derfor i st¯rre grad ment som motivasjon og oppfordring, enn som aksjon p en inntruffet situasjon.
I ettertid har gruppen reflektert over at alle medlemmenes innstilling og pÂgangsmot for  gj¯re godt arbeid har vÊrt utelukkende av den positive sort.
Flere punkter som slÂr ned p ikke-bidragende deltagelse har derfor aldri vÊrt aktuelt.

I grupperefleksjonen fra 2. landsbydag stÂr det at det ble en del dr¯fting rundt semantikken i samarbeidsavtalen. 
Diskusjon i plenum rundt sprÂklig formulering av setninger fikk un¯dvendig mye oppmerksomhet.
Gruppen hadde fÂtt sitt f¯rste m¯te med temaet ineffektiv diskusjon.
For  forhindre tap av tid p nÊringsfattige ting ble gruppen enig om  innf¯re en m¯teleder og en referent fra og med neste landsbydag.
M¯telederen skulle ha hovedansvar for  holde arbeidsplanen samt vÊre ordstyrer i diskusjoner.
Referenten skulle notere fra innsjekk, utsjekk, oppgaver som omhandlet teambuilding og den avsluttende grupperefleksjonen.
Det ble vedtatt  innf¯re en ordning slik at de to rollene gikk p omgang mellom medlemmene.
I ettertid kan det stilles sp¯rsmÂl ved om en mer effektiv ordning ville ha vÊrt  ha en fast ordstyrer og en fast referent.
De aktuelle personene ville kunne ha lÊrt av sine feil og gjort justeringer fra gang til gang i motsetning til  hoppe inn i rollen hver 6. uke.
Valget om rullering stod ved like fordi tapet av effektivitet ble vurdert til lavt i forhold til den verdien hvert medlem ville f ut av  f pr¯vd seg i de to rollene.
Innf¯ringen av ordstyrer og referent var et valg gruppen aldri diskuterte  endre pÂ. 
Verdien av rollene gjorde seg mest gjeldende nÂr gruppen utf¯rte oppgaver som krevde diskusjon, refleksjon og deling av synspunkt.
Gruppen som diskusjonsforum hadde sjeldent problemer med at medlemmer ikke lot andre f ordet, men  inkludere alle og ikke minst bryte gjennom ved digresjoner var en viktig jobb for ordstyreren. 
Spesielt som tiden gikk ble stemningen naturligvis mer og mer kameratslig slik at tulling og t¯ysing kom lettere.
Dermed fikk ordstyrer ogs bryne seg p oppgaven i  skille mellom relevant og urelevant utvikling av et diskusjonstema.

≈ skille mellom nÂr en diskusjon spinner inn i en relevant eller urelevant retning er sjelden en lett oppgave.
En av fordelene med  arbeide i en tverrfaglig heterogen gruppe er at personer med ulik erfaring og bakgrunn tenker ulikt nÂr en ide blir presentert.
Gruppen var fra starten klar p at dette mangfoldet skulle verdsettes og at terskelen for  presentere en ufullstendig ide skulle vÊre lav.
ÿnsket var at resten av gruppen skulle kunne bygge videre p en grunnide og Âpne for nye og interessante vinklinger. 
I samarbeidsavtalen het det: $Alle i gruppen skal vise hverandre respekt. Dette gjennom  lytte, bidra med egne meninger, gi konstruktiv kritikk og bygge videre p andres ideer.$.
For  realisere dette punktet diskuterte gruppen ofte ved  la hvert medlem legge frem sitt synspunkt etter tur.
≈ ta en slik runde rundt bordet var sjelden den mest effisiente mÂten  oppn et resultat pÂ.
Grundig diskusjon krever mye tid. 
I starten, da det ble gjennomf¯rt tidsbegrensede oppgaver, merket gruppen at tidsressursen raskt ble brukt opp.
Likevel kan det argumenteres for at metoden bidro til  oppn effektivt arbeid ved at slutningene fra en slik diskusjon ga det mest suksessfulle forholdet mellom oppnÂdd og ¯nsket resultat.
Med denne synsvinkelen kan det tenkes at effisient og effektivt arbeid m behandles som et kompromiss der begge umulig kan vÊre p topp samtidig.

%%% Effektivitet, trivsel, motivasjon, kvalitet, ineffektivitet
%Tidspress, ulike tanker om hva som er viktig, noen i gruppen mere opptatt av tid enn andre
Tidspress var et tema gruppen tidlig tok opp i plenum og diskuterte grundig.
Alle medlemmene var svÊrt forn¯yd med mÂten ting ble diskutert pÂ, og at konsensus ble tilstrebet under alle avgj¯relser. 
Under grupperefleksjonen p landsbydag 3 ytret flere at de var bekymret for om gruppen arbeidet for treigt.
Det ble ogs pÂpekt at ordningen med ordstyrer og referent fungerte bra, og at det f¯ltes som at denne aksjonen hadde forbedret tidsbruken.
Samme dag pÂpekte Jonas at han foretrekker grundig arbeid selv om det typisk er treigere.
Hele gruppen var i grunn enig i dette, men det mÂtte selvsagt eksistere en balanse mellom grundighet og progresjon.
I etterkant kan det pÂpekes at det har vÊrt en fordel for gruppen  ha medlemmer som verdsetter  bruke god tid og medlemmer som husker  iverksette tidsfrister.
Ved at ulike medlemmer har innehatt en form for motsatte roller med tanke p tidsbruk, har den ¯nskede balansen mellom det  ta seg tid til grundig diskusjon og samtidig ha stor nok progresjon blitt ivaretatt. 
%Effektivitet vs. kvalitet
En forhastet beslutning som gjerne kun er basert p en enkeltpersons ide fremkaller sjeldent tilh¯righet og eierskap hos en hel gruppe.
≈ f¯le eierskap til en ide, og f¯le tilh¯righet til en gruppe og det den skal produsere, er alfa og omega for god motivasjon.
At alle medlemmer i en gruppe har et ¯nske om og er villig til  arbeide for  n gruppens mÂl vil i lengden f¯re til bÂde effektivt og effisient arbeid.
Arbeid uten tilstrekkelig entusiasme er hverken trivelig eller av god kvalitet.
Som definert tidligere betyr god effektivitet at det er stort samsvar mellom oppnÂdd og ¯nsket resultat.
S lenge det er satt h¯ye kriterier til det ¯nskede produktet en gruppe vil oppnÂ, vil effektivitet dermed ogs gjenspeile kvalitet.
Effisient arbeid derimot trenger ikke  gjenspeile god kvalitet.
Ved  bruke s f ressurser som mulig, for eksempel ikke bruke tid p diskusjon, ikke pr¯ve  finne alternative ideer, ikke utvikle ideer videre, men rett og slett g for den raskeste, enkleste og minst krevende ideen, vil man ogs f produsert et sluttresultat.
Kvantitativt har gruppen oppnÂdd det samme, nemlig fÂtt skrevet to rapporter, men kvaliteten p innholdet vil vÊre dÂrligere enn om flere ressurser hadde blitt benyttet.
Et eksempel fra denne gruppens arbeid er diskusjonen og avgj¯relsen om hvordan prosessrapporten skulle struktureres og ferdigstilles.
Med kun to uker igjen til innlevering hadde gruppen enda ikke tatt en avgj¯relse p disposisjonen av prosessrapporten.
Tidspresset var dermed et faktum.
Den 13. landsbydagen diskuterte gruppen i plenum hvilke tema prosessrapporten burde inneholde og utformet en grov disposisjon for strukturen.
PÂf¯lgende landsbydag la Simen fram et nytt forslag til disposisjon, og gruppen tok seg tid til  diskutere det nye forslaget.
Ingen av gruppens medlemmer hadde noen gang skrevet en slik rapport tidligere, og vi skj¯nte fort at det var mange mulige mÂter  gj¯re det pÂ.
Landsbydag 14 ble avsluttet uten at gruppen hadde tatt en avgj¯relse p disposisjonen.
I utsjekken fra samme dag sa Jonas at han var svÊrt glad for at gruppen tok denne diskusjonen s grundig. Den gjorde at han ble skikkelig engasjert.
Simen satte pris p at gruppen ikke tok hans nye ide som god fisk, men stilte sp¯rsmÂl ved hans oppsett og diskuterte det videre.
Ingelin var ogs glad for at s mange ideer var lagt fram og at gruppen hadde vurdert mange muligheter, men samtidig var hun bekymret for at tiden begynte  bli knapp.
Gruppen endte opp med  m¯tes en dag utenom oppsatt tid, og fikk utarbeidet en detaljert disposisjon med innslag av mange ideer.
I ettertid viste det seg at selve arbeidet med  ferdigstille prosessrapporten gikk raskt i og med at disposisjonen var detaljert og alle medlemmene var inneforstÂtt med innholdet.
Likevel f¯rte denne grundigheten til at gruppen var n¯dt til  m¯tes utenfor oppsatt tid, og dermed ikke klarte  gjennomf¯re ¯nsket om  bli ferdig ved  kun arbeide p onsdager.
ForhÂpentligvis resulterte grundigheten i at hvert medlem fikk et konkret mÂl p hva som skulle inkluderes i rapporten slik at det videre arbeidet ble mer effisient enn hva det hadde blitt med den f¯rste l¯se planen. 



%Runde rundt bordet for ≈ holde alle oppdatert
%Hatt mye fokus p≈ samarbeid, at man ikke skal arbeide alene og at alle skulle ha god kontroll p≈ store deler av prosjektet til enhver tid
Metoden med  ta runden rundt bordet ble ogs benyttet for  raskt presentere hva hver enkelt arbeidet med, hvordan det gikk og planen videre.
P den mÂten var alle gruppens medlemmer oppdatert p hverandres arbeid til en hver tid, og alle f¯lte seg a jour kunnskapsmessig.
Ideelt sett ¯nsket gruppen  arbeide s lite som mulig med individuelt arbeid.
Alle gruppemedlemmene hadde lyst til  utnytte at dette var et fag spesielt tilrettelagt for samarbeid.
Det ferdige produktet skulle representere kunnskap hele gruppen hadde lÊrt seg i l¯pet av arbeidstiden, og ikke seks individuelle menneskers fagfelt sydd mer eller mindre sammen til ett.
Utfordringen med  drive aktivt samarbeid er at det er tidkrevende.
Spesielt nÂr ord skal formuleres til setninger, er det lite effisient  jobbe flere sammen.
L¯sningen ble  tilpasse samarbeidsmetoden etterhvert som arbeidsoppgavene endret seg.
I startfasen av prosjektoppgaven var det lett  samarbeide hele gruppen i plenum.
Da gjennomf¯rte gruppen brainstorming, idemyldring og diskusjon rundt mulige problemstillinger i fellesskap.
I ettertid er det grunn til  tro at denne gode og Âpne dialogen, villigheten til  diskutere og f innspill p sine ideer og utforme en problemstilling alle f¯lte eierskap til, la grunnlaget for at samarbeidet fungerte godt ogs nÂr arbeidet av praktiske grunner ble mer individuelt og tidspresset mer reelt.




%Kommentere p≈ effektivitet vs trivsel. Se p≈ hvordan balansegangen i trivsel p≈virker effektivitet og motivasjon for ≈ arbeide. 
%For mye sosialt, mindre fokus p≈ arbeidsoppgaver som skal lÿses?
En utfordring for alle grupper er  balansere faglig fokus og sosiale avbrekk.
En nyoppstartet gruppe der alle medlemmene er ukjente vil av naturlige Ârsaker kaste bort liten tid p interne vitser, tullete digresjoner og andre g¯yalheter.
Slik var ogs denne gruppen i starten. Det tar alltid tid f¯r man finner sin plass, plasserer andre og blir komfortabel i en gruppe. 
% Konkret tidspunkt der noen kommenterte at ting hadde blitt mer sosialt i gruppa? Mer tull og t¯ys? Etter landsbydag ?? ble det kommentert at 

Det tok ikke mange landsbydagene f¯r gruppen utviklet seg til  ikke bare vÊre en faglig samarbeidsgruppe, men ogs en gruppe som trivdes sammen sosialt. 
Allerede etter de f¯rste tegnene til at praten satt l¯sere og medlemmene begynte  lirke av seg morsomheter, ble det stilt reflekterende sp¯rsmÂl ved utviklingen av triveligheten.
Ville det trivelige milj¯et i gruppen ha negativ effekt p det faglige arbeidet?
Var trivselen n¯dvendig for  holde motivasjonen oppe? 
Hvor gÂr skillet mellom syklubb og faggruppe med god trivsel?
Det finnes ingen fasitsvar, og det er vanskelig  bed¯mme fordi alle andre scenario enn det faktiske blir svÊrt hypotetiske.
Selvsagt kan enhver gruppe bli mer tidseffektiv ved  forby alt annet enn faglig arbeid, men det er ogs grunn til  tro at det faglige arbeidet blir bedre av at ogs trivselen er stor.
En annen vinkling av saken er om det er mulig  oppn god trivsel dersom det ikke i tillegg presteres faglig.
Det er ikke utenkelig at hum¯ret til en gruppe ville ha blitt preget av at de faktiske mÂlene ikke ble nÂdd.
Fra og med landsbydag 6 valgte gruppen  arbeide p et eget grupperom kun med sporadiske bes¯k av lÊringsassistenter og landsbyleder.
Gruppen ble dermed flink til  arbeide uten "oppsyn".
Denne egenskapen tok gruppen med seg da den senere ogs ble n¯dt til  m¯tes utenom oppsatte arbeidstider.
Det har aldri vÊrt diskutert at en landsbydag har bestÂtt av for mye tull og fritidsprat som har gÂtt p bekostning av framdrift i arbeidet.
Gruppen som helhet er enig om at trivselen mellom medlemmene utelukkende har f¯rt til  bedre arbeidsmilj¯et og ¯nsket om  prestere bra sammen. 
























