\subsection{Kommunikasjon}
Kommunikasjon, roller og normer, personligheter

%Hvordan gruppa har snakket med hverandre. Informasjonsflyt. Hvem sier hva. Initiativ. Kommunikasjonsmønster. Hvordan formidler vi kommunikasjon til hverandre? Er vi flinke til å være tydelig og oppklarende, eller bygger vi på antagelser og egne tolkninger? Hvilke roller har de forskjellige medlemmene i gruppa. Er de formet av oss selv eller faget og rammene vi har gitt oss? Faglig nivåforskjell/hetrogenitet/forkunnskaper.

God kommunikasjon er en av de viktigste karakteristikkene ved et effektivt team.
Dersom man skal klare å jobbe godt sammen for å levere et produkt kreves blant annet utveksling av kunnskap og diskusjoner om viktige valg.
I vår gruppe har kommunikasjonsmønstret utviklet seg fra start til slutt.
Gjennom flere ulike øvelser og felles arbeid har vi fått testet gruppa, fått tilbakemeldinger og muligheter til å foreta aktive endringer i hvordan vi kommuniserer med hverandre.

\begin{itemize}
\item Samarbeidskontrakt (spesifikk respect for andres forslag)
\item SITRA - Effekten av dette på gruppen (semantikk)
\item Sosiogram + effekter
\item Grunnregler for effektive team
\item Samarbeidsindikatoren
\end{itemize}

\input{"samarbeiskontrakt.tex"}
%\input{"SITRA.tex"}
%\input{"Sosiogram.tex"}
%\input{"Grunnregler.tex"}
\input{"samarbeidsindikator.tex"}

\item Teori: stjernekommunikasjon
\item Figur: Elementær kommunikasjonsteori (se bilde)

\item Konkret: Utviklingen av samarbeidsindikatoren
\item Abstrakt: Pros and cons ved grundig planlegging
\item Ønske om å engasjere alle, alle skal få bidra med sitt og bli hørt. Oppfordrer til å delta.
\item Folk kommuniserer anderledes, “introspektiv” tenker før snakker, “ekstrospektiv” snakker for å tenke
\item Kommunikasjon: Får de som snakker mindre mere “priority?” F.eks. at Jonas velger Ingelin over Karsten, siden Karsten snakker mere
\end{itemize}





















