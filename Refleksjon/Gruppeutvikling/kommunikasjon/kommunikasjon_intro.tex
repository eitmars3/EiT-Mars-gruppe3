\subsection{Kommunikasjon}
%Kommunikasjon, roller og normer, personligheter

%Hvordan gruppa har snakket med hverandre. Informasjonsflyt. Hvem sier hva. Initiativ. Kommunikasjonsmønster. Hvordan formidler vi kommunikasjon til hverandre? Er vi flinke til å være tydelig og oppklarende, eller bygger vi på antagelser og egne tolkninger? Hvilke roller har de forskjellige medlemmene i gruppa. Er de formet av oss selv eller faget og rammene vi har gitt oss? Faglig nivåforskjell/hetrogenitet/forkunnskaper.

God kommunikasjon er en av de viktigste karakteristikkene ved et effektivt team.
Dersom man skal klare å jobbe godt sammen for å levere et produkt kreves blant annet utveksling av kunnskap og diskusjoner om viktige valg.
I gruppen har kommunikasjonsmønstret utviklet seg fra start til slutt.
Gjennom flere ulike øvelser og felles arbeid har gruppen fått testet seg, fått tilbakemeldinger og muligheter til å foreta aktive endringer i hvordan gruppemedlemmene kommuniserer med hverandre.

Gruppen består av medlemmer som opprinnelig hadde veldig ulike \textit{tilnærminger} til åpne diskusjoner og samtaler.
For eksempel sier både Karsten og Jonas at de er klar over at de tar mye initiativ og \textit{tar stor plass} i diskusjoner.
Derimot forholder Simen og Ingelin seg forholdsvis stille, og tar gjerne ikke ordet like ofte.
Anna og Martin plasserer seg omtrent midt mellom de to andre \textit{grupperingene}.
Dette mønsteret i kommunikasjonen er noe gruppen var klar over ganske tidlig, og har jobbet med i løpet av prosjektperioden.
Andre kommunikasjonstrender som har blitt observert i gruppen er en økt grad av frie samtaler, småprat og humor. Spesielt Martin og Karsten har hatt lett for å skape digresjoner, slik at resten av gruppen har måttet be dem om å beholde fokus. 
Kommunikasjonsmønsteret i gruppen har under hele prosessen endret seg.
Noen ganger gradvis og sakte, andre ganger raskere. 

\subsubsection{Samarbeidskontrakt}
En samarbeidsavtale ble utarbeidet i starten av prosessen for å oppnå en felles oppfatning innad i gruppen om hvordan gruppearbeidet skulle utføres, og for å bestemme hvilke krav gruppens medlemmer kunne stille til hverandre.
Det ble foreslått at avtalen skulle ha tre hovedfokus; leveranse, læring og trivsel.
Utenom et utkast fra en samarbeidsavtale en annen gruppe hadde utarbeidet jobbet gruppen på egenhånd og formet avtalen i fellesskap.
Samarbeidsavtalen ligger i vedlegg \ref{Ved:samarbeidsavtale}.
Avtalen inneholder flere viktige punkter som legger til rette for effektivt gruppearbied, og punkt nummer 8, 10 \& 11 ble bestemt spesifikt for å legge til rette for god kommunikasjon innad i gruppen:
\begin{itemize}
	\item Gruppen skal drive kunnskapsutveksling gjennom diskusjon og drøfting av faglig innhold, samt utfordre hverandre til å gå utenfor den faglige komfortsonen.
	\item Gruppen skal ha rullerende møteleder og sekretær for hver landsbydag.
	\item Alle i gruppen skal vise hverandre respekt. Dette gjennom å lytte, bidra med egne meninger, gi konstruktiv kritikk og bygge videre på andres id\`{e}er.
\end{itemize}

Disse punktene valgte gruppen å ha med i samarbeidsavtalen på grunn av et stort ønske om at alle skulle få plass i diskusjoner og at alle skulle bidra.

\paragraph{Kunnskapsutveksling}
EiT legger opp til at alle medlemmene som jobber med prosjektet skal få noe faglig ut av arbeidet.
Både innenfor sine egne vandte fagområder og resten av gruppens.
For å oppnå dette ønsket gruppen at det skulle være lett å dele den kunnskapen de enkelte medlemmene satt med fra før.
Blant annet ville det være viktig å utnytte Ingelins forhåndskunnskaper om generell kjemi for at resten av gruppen skulle kommer raskt igang med arbeidet.
Siden samarbeidsavtalen var noe som ble utviklet tidlig, og før prosjektarbeidet hadde kommer skikkelig i gang, er det vanskelig å se om dette punktet har ført til noen spesiell utvikling i gruppen.
Likevel kommer det frem av hvordan gruppen jobber med prosjektet at det i alle fall har vært en av muligens flere årsaker til en åpen kommunikasjon.

Blant annet har flere av gruppemedlemmene vært takknemmelig for hvor enkelt det har vært å spørre om andres kompetanse under prosjektarbeidet.
Karsten som har jobbet mye med de biologirettede delene av prosjektrapporten har gitt uttrykk for at Ingelin sine forhåndskunnskaper var til stor hjelp, og at hun var villig til å avbryte eget arbeid for å hjelpe til.
Det er også av interesse å høre om hva de ulike gruppemedlemmene har av faglige lidenskaper for å kunne skape et godt sosialt og løsningorientert miljø i gruppen.

Et annet eksempel på hvordan kunnskapsutveksling har foregått er de tekniske løsningene rundt dokumentdeling og ferdigstilling av prosjekt- og prosessrapporten.
For å legge til rette for individuelt arbeid valgte gruppen å bruke en programvare som heter Git til å sørge for versonskontroll og at dokumentene det ble arbeidet med var oppdaterte.
Programvaren tok litt tid å sette opp, men etter dette har medlemmene kunnet skrive selvstendig samtidig som det ikke var behov for å bekymre seg for at noen andres arbeid ble slettet i prosessen.
Ferdigstillingen av rapportene har blitt gjort i \Latex.

Begge disse tekniske løsningene har noen implementeringsutfordringer som Jonas og Martin har jobbet med underveis.
Det har krevd at de begge bidro med å dele sin forhåndskunnskap og samarbeide om å finne ny informasjon som kunne belyse problemer som oppstod underveis.



\paragraph{Rullerende møteleder og sekretær}



\paragraph{Respekt gjennom å lytte, kritisere og engasjere seg for andres id\`{e}er}
Trenden med at medlemmene i gruppen hadde ulik mengde initiativ i diskusjoner var også lagt til grunn for regel \#11.
Som Jonas skriver i refleksjonen om hva han lærte om gruppen i løpet av 14. januar:
"
Vi har veldig ulike bakgrunner, både sosialt og faglig. Likevel virker alle åpne. Det er flere av oss som liker å ta ordet. Det er en bra start, men vi må være obs. på at alle får sagt det de ønsker.
"



















\subsubsection{SITRA}

Øvelsen SITRA gikk ut på at gruppemedlemmene leste to personlige refleksjoner av en situasjon som kunne ha oppstått.
Medlemmene skulle først individuelt kategorisere setningene som situasjon, refleksjon, aksjon eller teori i de to tekstene.
Så skulle disse tolkningene sammenlignes felles i gruppa.
Til å begynne med var det store ulikheter i oppfatningene om hvilke av disse kategoriene som gjaldt for hver setning.
Etter diskusjoner ble det imidlertid enighet om hvor skillet mellom de ulike kategoriene skulle være.

\begin{center}
	\includegraphics[width=0.5\textwidth]{Kommunikasjon.PNG}
\end{center}

Denne øvelsen trente gruppedeltakerne på kommunikasjon og da spesielt i problemene som oppstår ved semantisk støy.
Semantisk støy handler om hvordan ord og uttrykk blir tolket.
Hvis sender og mottaker operer med ulike referanserammer så kan budskapet bli tolket på en annen måte enn det var tiltenkt eller det kan bli fullstendig uforståelig \cite{prosjekteringsledelse}.
Formålet med SITRA-øvelsen var å bevisstgjøre gruppa på disse referanserammene.

\subsubsection{Sosiogram}
På landsbydag 4 jobbet gruppen med å utarbeide en problemstilling.
Dette var et arbeid som foregikk i fellesskap under en åpen diskusjon.
I løpet av rundt 5 minutter hadde Erik observert kommunikasjonsmønsteret i gruppen og laget et sosiogram underveis.
Et sosiogram er en grafisk fremstilling av hvordan de ulike grupemedlemmene henvender seg til hverandre og gruppen som helhet når de snakker i en åpen samtale.
Erik spurte gruppen om den kunne ta en liten pause og se på sosiogrammet i fellesskap før den fortsatte arbeidet.

\begin{figure}
\label{fig:sosiogram}
\caption{Sosiogram for gruppen fra diskusjon under utarbeiding av problemstilling den - landsbydagen.}
\begin{center}
%	\includegraphics[width=0.45\textwidth]{sosiogram.jpg}
\end{center}
\end{figure}

Sosiogrammet til gruppen vises i figur \ref{fig:sosiogram}. Det er tydelig at medlemmene i gruppen fokuserte mye på å snakke med enkeltpersoner og ikke gruppen som helhet.
I tillegg var det ingen som henvendte seg direkte til hverken Anna eller Ingelin, på tross av den ellers direkte kommunikasjonen.
Ellers kommer det frem at Jonas og Karsten hadde mye toveis kommunikasjon seg imellom.
Begge disse punktene ble gruppemedlemmene enige om at de ønsket å forbedre.

Etter denne hendelsen har gruppen i stor grad endret kommunikasjonsmønsteret.
Enkeltmedlemmene er bevisst på hvordan de ønsker å kommunisere med hverandre, og møtelederen er ekstra observant for å unngå at noen blir ekskludert fra diskusjoner.
Dette fører til en mer åpen kommunikasjon medlemmene imellom, og gjør det enklere for alle å bidra med det de ønkser.

\subsubsection{Grunnregler for effektive grupper}

En øvelse som ble gjennomgått i gruppa var å diskutere et sett med regler hver enkelt kan følge for å oppnå effektivt gruppearbeid.
Dette ble gjordt for at samtlige gruppemedlemmer tidlig i prosjektet skulle bli mer bevisste på hvordan man kan forholde seg for å unngå kommunikasjonsproblemer.
Reglene beskriver visse adferder som er med på å gjøre kommunikasjon i en gruppe mer tydelig [Schwarz 2002]:

\paragraph{Sjekk antakelser og slutninger}
Man kan gjøre flere slutninger daglig uten å være klar over det, samt at de kan være intrikate og fjerne fra observerte data.
At man ikke engang er klar over dette gjør at slutningene kan oppfattes som fakta.
Dermed kan man, ved å stenge ute deler av et budskap og misforstå det i sin helhet, danne seg et feilaktig inntrykk av intensjonene til den som kommer med budskapet.
For å unngå dette må man alltid være åpen for å tolke på flere måter.
Er man i tvil, kan man også spørre om man har tolket budskapet riktig.

\paragraph{Del all relevant informasjon}
Om ikke all relevant informasjon foreligger, vil gruppas kollektive misforståelse av et emne kunne bidra til å ta feil beslutninger.

\paragraph{Eksemplifiser og bli enige om viktige begreper}
Å eksemplifisere mens man forklarer gjør det lettere for motparten å validere utsagnene, siden eksemplene gjerne henviser til observerte eller observerbare data.
Dette bør gjerne brukes for å tydeliggjøre sentrale begreper, da det er viktig for å få en felles oppfattelse av disse så tidlig som mulig.

\paragraph{Forklar ditt resonnement og din intensjon}
Et utsagn kan tolkes på flere måter av mottakeren.
Mottakeren spør heller ikke alltid om budskapet er forstått riktig, men heller feiltolker, jfr. regelen "Sjekk antakelser og slutninger".
Det kan derfor være til hjelp om den som kommer med utsagnet også utdyper det på eget initiativ.

\paragraph{Fokus på interesser, ikke posisjon}
Når et fellesskap skal komme til enighet hender det at noen er tidlig ute med et forslag til løsning.
Da stiller vedkommende seg i en posisjon hvor det er ukjent for de andre hvilke interesser denne personen legger til grunn for en slik løsning.
Problemet er at slike forslag og motforslag kan være inkompatible selv om interessene egentlig stemmer overens.
Hvis man åpner med å dele interesser med de andre, er det lettere å utarbeide løsninger i fellesskap.
Om andre tidlig kommer med løsninger istedet for å begrunne dem, kan man spørre om interessene som ligger til grunn.

\paragraph{}

\paragraph{Finn måter for å sjekke misforståelser}
En uenighet mellom to personer kan fort havne i en posisjonsorientert og låst situasjon.
Partene kan mene at de selv har rett, uansett hvilke beviser som måtte foreligge.
For å løse dette må man blir enige om hva problemet omfatter, for begge kan ha tatt hensyn til hver sine aspekter.
Videre må man prøve å finne ut hvordan begge samtidig kan ha rett.
Ved å lage en testplan sammen kan man da finne ut om noen av partene skulle vise seg å ha ukorrekt informasjon.

\paragraph{Diskuter udiskutable emner}
I grupper kan det være saker som påvirker gruppas oppgave, men som det er vanskelig å diskutere i fellesskap.
Det er gjerne saker som skyldes et medlem, og man vegrer seg for å snakke med denne personen om det.
Emnet holdes kanskje udiskutert, eller man tar det opp med andre i gruppa, noe som ikke løser problemet.
Man kan gjøre gruppa oppmerksom på at man skal diskutere noe udiskutabelt.
Alternativt kan man si ifra til den det gjelder at man kommer til å ta opp temaet i fellesskap.
Det siste gir personen mulighet til å forsvare seg i diskusjonen som kommer, noe som vil virke mest mulig skånsomt.

\paragraph{Bruk en beslutningsprosess for passe enighet}


\subsubsection{Samarbeidsindikator}

Hovedfokuset for samarbeidsindikatoren har vært å undersøke det sosiale samværet gruppen har hatt, samt diskutere hvilke effekter dette har hatt på gruppens produktive kapasitet.
Samarbeidsindikatoren baseres på en spørreundersøkelse som holdes tidlig, midtveis og sent i semesteret.
Det gir tre samarbeidsindikatorer, der man skal kunne se tendenser til utvikling av kommunikasjon i gruppen.
En rekke sosiale kvaliteter særpreger effektivt gruppesamarbeid \cite{orgorg}:

\begin{itemize}
	\item åpen kommunikasjon
	\item gjensidig tillit
	\item sosial støtte
	\item utnyttelse av individuelle forskjeller
\end{itemize}
	
Alle disse punktene kan kommenteres i forbindelse med gruppens sosiale utvikling.
Enighet om mål, og hvordan disse skulle oppnås, ble tidlig etablert i form av samarbeidskontrakten.
Ønsket om karakteren A har preget gruppens innsats, og fastsatte regler om oppmøte og leveringsfrister har fremmet profesjonalitet innad i gruppen.
Den første samarbeidsindikatoren antyder at gruppen har vært preget av lite ærlighet eller direkthet.
Dette tolket gruppen som at kommunikasjonen kunne vært mer åpen.

\begin{center}
	\includegraphics[width=0.5\textwidth]{samarbeidsindikator1.png}
\end{center}

\paragraph{Sosiale inntrykk}

En god sosial kommunikasjon og en god tone i gruppen hjelper til for å skape et positiv og trygt arbeidsmiljø.
Det gir medlemmer mer lyst til å samarbeide, og de føler seg mer komfortable med å jobbe sammen.
Et godt sosialt miljø i gruppen gjorde at medlemmene ble mer klare over hvilke personlighetstyper som fins i gruppen, noe som er svært viktig informasjon i et samarbeid.
Fra begynnelsen av prossessen har hvert enkelt gruppemedlem vært inkluderende og dedikert. 
Derfor ble, for eksempel, innsjekk og utsjekk alltid tatt på alvor.
I disse samlingene har individer fått mulighet til å vise tillit til hverandre og inkludere resten av gruppen i sitt liv. 
Det ble etterhvert en slags tradisjon, hvor vi følger hverandres prosjekt og ikke minst utvikling i andre området av livet.
Crossfit, barn, maraton, røyking, Bahamas, mat og Samfundet er noen temaer som har vært gjennomgått.

At gruppen har hatt så vennlig sosial kontakt medfører til både positive og negative konsekvenser.
Det positive er at samtlige gruppenmedlemmer ble motivert fra begynnelsen til å delta og gjøre en god innsats.
Ikke minst var det viktig for å bevare den gode tonen samt unngå eventuelle konflikter.
En naturlig konsekvens av dette er at medlemmer kanskje ikke tør å ødelegge den positive stemningen ved å gi hverandre for direkte tilbakemeldinger eller kritikk i sammenheng med arbeidet.
Kanskje er det dette fenomenet samarbeidsindikatoren adresserer når den viser at gruppen av vært lite ærlig og direkte.
Den gode tonen gjorde også at gruppen i begynnelsen har slitt noe med tid og struktur og tapte derfor på effektivitet.

Stemningen i gruppen har stort sett vært avslappet med tendenser til vitsing.
Det ble raskt tendenser til distraksjoner under gruppemøter, og behovet for å kalle til ro dukket opp ved flere anledninger, noe som førte til at gruppen innførte ordstyrer.

Det er ingen tvil om at sosialt god kommunikasjon går hånd i hånd med hvordan medlemmer jobber.
Siden alle har vært svært dediktert i arbeidet sitt og etterhvert fått tillit fra gruppen, fikk gruppen sigende et mer avslappende forhold til forsinkelser, fravær og fristforsinkelser.

Medlemmenes inkludering av hverandre og positive forhold til faget gjorde at medlemmene følte eierskap og ansvar i samarbeidet.
Samtlige medlemmer ble sett og hørt, og ingen fikk mulighet til å melde seg ut.
Gruppen har for eksempel bestemt seg for å dra på restaurant for å ferie sammen når rapportene er leverte. 
Det gjør at medlemmene er knyttet sammen og deler ansvar.
Å fokusere på at alt skulle gjøres sammen i sosial sammenheng har hatt positiv konsekvens i det faglige arbeidet.
\iffalse
(Eksempel av Jonas i konferans [forelesningen?] som satt alene, Karsten og Anna gjorde plass sånn at de kunne alle 3 sitte sammen, Crossfit dag, vi slette å ta ut en medlem av gruppe )


<Kort beskrivelse av den sosiale atmosfæren: Avslappet, tendens til vitsing, tendens til distractions under gruppemøter, tendens til å kalle til fokus når det trengs>
etc
Spørsmål:
Har vi hatt positiv innflytelse av at alle var villige til å gjøre en grundig jobb (gå etter A’en, samarbeidskontrakten)?
Var alle det fra start av, eller kom viljen til å gjøre en grundig jobb som følge av god teamfølelse?
“Veldig avslappet” pros and cons
Pros: Har dette gjort at folk ikke blir arrige over forseintkomminger og fristoverskridelser, i og med at alle på teamet føler seg trygge på at en innsatts blir gjort? 
Pros: Har dette gjort det lettere å håndtere uenigheter, siden folk har hatt god vilje og “likt” hverandre?
Har det vært nyttig som ‘sosial judo’ i å kunne snakke om konflikter i tidlig fase, lenge før kokepunkt?
Cons: Har det kostet oss verdiful tid, spesielt i den første halvdelen av faget? Var progresjon i arbeidet jevnlig handicappet? Gikk det ofte veldig sakte fremover? Brukte gruppen mye tid på unødvendig sosial prat? Kunne dette føre til frustrasjon blandt noen medlemmer på gruppen, og mindre hos andre, derved skape mulighet for konflikt (noen jobber hardt, andre prater bare tull)
Innflytelse av god sosial stemning på innsjekk/utsjekk: bryr folk seg om hverandre (referer til fagets beskrivelse av forskjellige team-egenskaper). F eks teamet bryr seg om de individuelle og hvilken overføringsverdi dette har til arbeidet.
 Anna røykestopp/maraton, Simen + sønn Odin, Ingelin CF progresjon, Martin mat (:))) /DJing/tech ansvar på singsaker studenthjem), Karsten mountainbike/Bahamastur, Jonas sine eventyr på samfundet 

<mangler spesifikke referanser til referat; vi må få samlet disse et sted>

\fi
