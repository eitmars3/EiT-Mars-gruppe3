\subsubsection{Samarbeidskontrakt}
En samarbeidsavtale ble utarbeidet i starten av prosessen for å oppnå en felles oppfatning innad i gruppen om hvordan gruppearbeidet skulle utføres, og for å bestemme hvilke krav gruppens medlemmer kunne stille til hverandre.
Det ble foreslått at avtalen skulle ha tre hovedfokus; leveranse, læring og trivsel.
Utenom et utkast fra en samarbeidsavtale en annen gruppe hadde utarbeidet jobbet gruppen på egenhånd og formet avtalen i fellesskap.
Samarbeidsavtalen ligger i vedlegg \ref{Ved:samarbeidsavtale}.
Avtalen inneholder flere viktige punkter som legger til rette for effektivt gruppearbied, og punkt nummer 8, 10 \& 11 ble bestemt spesifikt for å legge til rette for god kommunikasjon innad i gruppen:
\begin{itemize}
	\item Gruppen skal drive kunnskapsutveksling gjennom diskusjon og drøfting av faglig innhold, samt utfordre hverandre til å gå utenfor den faglige komfortsonen.
	\item Gruppen skal ha rullerende møteleder og sekretær for hver landsbydag.
	\item Alle i gruppen skal vise hverandre respekt. Dette gjennom å lytte, bidra med egne meninger, gi konstruktiv kritikk og bygge videre på andres id\`{e}er.
\end{itemize}

Disse punktene valgte gruppen å ha med i samarbeidsavtalen på grunn av et stort ønske om at alle skulle få plass i diskusjoner og at alle skulle bidra.

\paragraph{Kunnskapsutveksling}
EiT legger opp til at alle medlemmene som jobber med prosjektet skal få noe faglig ut av arbeidet.
Både innenfor sine egne vandte fagområder og resten av gruppens.
For å oppnå dette ønsket gruppen at det skulle være lett å dele den kunnskapen de enkelte medlemmene satt med fra før.
Blant annet ville det være viktig å utnytte Ingelins forhåndskunnskaper om generell kjemi for at resten av gruppen skulle kommer raskt igang med arbeidet.
Siden samarbeidsavtalen var noe som ble utviklet tidlig, og før prosjektarbeidet hadde kommer skikkelig i gang, er det vanskelig å se om dette punktet har ført til noen spesiell utvikling i gruppen.
Likevel kommer det frem av hvordan gruppen jobber med prosjektet at det i alle fall har vært en av muligens flere årsaker til en åpen kommunikasjon.

Blant annet har flere av gruppemedlemmene vært takknemmelig for hvor enkelt det har vært å spørre om andres kompetanse under prosjektarbeidet.
Karsten som har jobbet mye med de biologirettede delene av prosjektrapporten har gitt uttrykk for at Ingelin sine forhåndskunnskaper var til stor hjelp, og at hun var villig til å avbryte eget arbeid for å hjelpe til.
Det er også av interesse å høre om hva de ulike gruppemedlemmene har av faglige lidenskaper for å kunne skape et godt sosialt og løsningorientert miljø i gruppen.

Et annet eksempel på hvordan kunnskapsutveksling har foregått er de tekniske løsningene rundt dokumentdeling og ferdigstilling av prosjekt- og prosessrapporten.
For å legge til rette for individuelt arbeid valgte gruppen å bruke en programvare som heter Git til å sørge for versonskontroll og at dokumentene det ble arbeidet med var oppdaterte.
Programvaren tok litt tid å sette opp, men etter dette har medlemmene kunnet skrive selvstendig samtidig som det ikke var behov for å bekymre seg for at noen andres arbeid ble slettet i prosessen.
Ferdigstillingen av rapportene har blitt gjort i \LaTeX.

Begge disse tekniske løsningene har noen implementeringsutfordringer som Jonas og Martin har jobbet med underveis.
Det har krevd at de begge bidro med å dele sin forhåndskunnskap og samarbeide om å finne ny informasjon som kunne belyse problemer som oppstod underveis.



\paragraph{Rullerende møteleder og sekretær}



\paragraph{Respekt gjennom å lytte, kritisere og engasjere seg for andres id\`{e}er}
Trenden med at medlemmene i gruppen hadde ulik mengde initiativ i diskusjoner var også lagt til grunn for regel \#11.
Som Jonas skriver i refleksjonen om hva han lærte om gruppen i løpet av 14. januar:
"Vi har veldig ulike bakgrunner, både sosialt og faglig. Likevel virker alle åpne. Det er flere av oss som liker å ta ordet. Det er en bra start, men vi må være obs. på at alle får sagt det de ønsker."

















