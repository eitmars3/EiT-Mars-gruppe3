\subsubsection{Samarbeidskontrakt}
En samarbeidsavtale ble utarbeidet i starten av prosessen for å oppnå en felles oppfatning innad i gruppen om hvordan gruppearbeidet skulle utføres, og for å bestemme hvilke krav gruppens medlemmer kunne stille til hverandre.
Det ble foreslått at avtalen skulle ha tre hovedfokus; leveranse, læring og trivsel.
Utenom et utkast fra en samarbeidsavtale en annen gruppe hadde utarbeidet jobbet gruppen på egenhånd og formet avtalen i fellesskap.
Samarbeidsavtalen ligger i vedlegg \ref{Ved:samarbeidsavtale}.
Avtalen inneholder flere viktige punkter som legger til rette for effektivt gruppearbied, og punkt nummer 8, 10 \& 11 ble bestemt spesifikt for å legge til rette for god kommunikasjon innad i gruppen:
\begin{itemize}
	\item Gruppen skal drive kunnskapsutveksling gjennom diskusjon og drøfting av faglig innhold, samt utfordre hverandre til å gå utenfor den faglige komfortsonen.
	\item Gruppen skal ha rullerende møteleder og sekretær for hver landsbydag.
	\item Alle i gruppen skal vise hverandre respekt. Dette gjennom å lytte, bidra med egne meninger, gi konstruktiv kritikk og bygge videre på andres id\`{e}er.
\end{itemize}

Disse punktene valgte gruppen å ha med i samarbeidsavtalen på grunn av et stort ønske om at alle skulle få plass i diskusjoner og at alle skulle bidra.

\paragraph{Kunnskapsutveksling og arbeid utenfor komfortsonen}



\paragraph{Rullerende møteleder og sekretær}


\paragraph{Respekt gjennom å lytte, kritisere og engasjere seg for andres id\`{e}er}
Trenden med at medlemmene i gruppen hadde ulik mengde initiativ i diskusjoner var også lagt til grunn for regel \#11.
Som Jonas skriver i refleksjonen om hva han lærte om gruppen i løpet av 14. januar:
\textit{
Vi har vledig ulike bakgrunner, både sosialt og faglig. Likevel virker alle åpne. Det er flere av oss som liker å ta ordet. Det er en bra start, men vi må være obs. på at alle får sagt det de ønsker.
}

















