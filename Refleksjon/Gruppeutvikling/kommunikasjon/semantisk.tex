\subsubsection{Semantisk støy}

Øvelsen SITRA gikk ut på at gruppemedlemmene leste to fiktive refleksjoner av en og samme situasjon.
Medlemmene skulle først individuelt kategorisere setningene som situasjon, refleksjon, aksjon eller teori.
Deretter skulle tolkningene sammenlignes og diskuteres felles i gruppen.
Denne øvelsen trente gruppemedlemmene i kommunikasjon, og da spesielt i problemene som oppstår ved semantisk støy.
Semantisk støy handler om hvordan ord og uttrykk blir tolket.
Til å begynne med var det store ulikheter i oppfatningene av hva som hørte til i hver kategori.
Det krevdes derfor diskusjon og presisjon fra hvert medlem om hvor skillet mellom de ulike kategoriene skulle være.
Øvelsen viste hvor viktig det er å forklare andre medlemmer hva som menes med ulike begreper. 
Dette faktum er også tydeliggjort i \textit{Roger Schwarz} sine \textit{Grunnregler for effektive grupper}\cite{schwarz}. 
Han poengterer viktigheten av å sjekke antagelser og slutninger som trekkes fra det andre presenterer til deg. \\

Det trekkes flere slutninger daglig, ofte automatisk og uten å være gjennomtenkt. 
Uten å være klar over det, kan slutningene oppfattes som fakta.
Dermed kan misforståelser oppstå, ved at deler av budskapet er blitt stengt ute eller mistolket.
Intensjonene til den andre kan dermed tolkes feilaktig i forhold til personens egentlige budskap.
Et utsagn kan tolkes på flere måter av mottakeren.
Mottakeren spør heller ikke alltid om budskapet er forstått riktig, men heller feiltolker.
Det kan derfor være til hjelp om den som kommer med utsagnet også utdyper det på eget initiativ.
Hvis det er tvil om intensjoner og budskap, må medlemmene ikke nøle med å spørre om alt er forstått korrekt.
Ved hjelp av SITRA-øvelsen ble gruppen for første gang introdusert for viktigheten av å presisere hva som legges i begreper. \\

Under grupperefleksjonen landsbydag syv, diskuterte Anna og Karsten oppgavene til referentsrollen.
Begge tolket hverandres synspunkter for fort uten å ha forstått den andres egentlige budskap.
Det resulterte i at de trodde den andre hadde motsatte intensjoner, mens de faktisk hadde veldig like synspunkt om hva referenten hadde som oppgave. 
For utenom problemer med å formulere et synspunkt på en forståelig måte, innrømte flere av gruppemedlemmene at mangel på tålmodighet kunne føre til at slutninger ble trukket før et helt resonement har blitt lagt frem. 
Karsten ytret: \emph{"Min utfordring i forhold til disse grunnreglene, er at jeg har en tendens til å avbryte, ta ordet og trekke egne konklusjoner på andres innspill. Jeg har en tendens til å være utålmodig når andre bruker lang tid til å komme til saken."}
Etter å ha oppdaget hvor lett det er å misforstå andre, prøvde gruppen å hyppig stille spørsmål for å sikre at budskapet hadde blitt forstått riktig.
 Å totalt utelukke at medlemmer trekker sine egne konklusjoner, vil alltid forekomme i en eller annen grad.
 Ved å bli gjort direkte oppmerksom på situasjonen, kunne gruppen prøve å redusere omfanget av den videre i samarbeidet. \\

Et annet tiltak for å redusere semantisk støy er å eksemplifisere forklaringer.
Dette gjør det lettere for motparten å validere utsagnene, siden eksemplene gjerne henviser til observerte eller observerbare data.
Dette bør gjerne brukes for å tydeliggjøre sentrale begreper, da det er viktig for å få en felles oppfattelse av disse så tidlig som mulig.
Simen var svært flink til å tegne for å illustrere sine idéer slik at alle gruppemedlemmene fikk et klart bilde av hva de innebar. \\

En uenighet mellom to personer kan fort havne i en posisjonsorientert og låst situasjon.
Partene kan mene at de selv har rett, uansett hvilke beviser som måtte foreligge.
For å finne måter å sjekke misforståelser, må det bli enighet om hva problemet omfatter. Begge parter kan ha tatt hensyn til hver sine aspekter.
Videre må det diskuteres om begge samtidig kan ha rett.
Et mer gjennomgående og ikke-iscenesatt eksempel på dette er at Anna og Karsten ikke har norsk som morsmål.
Dette har aldri skapt store misforståelser i gruppen, sannsynligvis fordi Anna har vært tydelig fra starten av om at norsken hennes ikke er perfekt.
Landsbydag syv gjentok Anna under grupperefleksjonen: \emph{"Jeg vil gjenta at siden jeg har en språkbarriere vil jeg jobbe for å være enda tydeligere og presis når jeg formulerer mine idéer."}. \\

Disse grunnreglene og andre øvelser, har gjort at gruppen har vært flink til å formulere seg, og ikke minst få og spørre om bekreftelse på at innholdet er forstått korrekt.
Gruppen er fortsatt ydmyk og fullt klar over at det finnes et forbedringspotensiale hos alle gruppemedlemmene. \\
