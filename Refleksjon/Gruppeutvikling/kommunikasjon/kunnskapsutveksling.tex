\subsubsection{Kunnskapsutveksling}
EiT legger opp til at alle medlemmene som jobber med prosjektet skal få et faglig utbytte.
Både innenfor sine egne fagområder og resten av gruppens.
For å oppnå dette ønsket gruppen at det skulle være lett å dele den kunnskapen de enkelte medlemmene satt inne med med fra før.
Så tidlig som 2. landsbydag utarbeidet gruppen en samarbeidskontrakt.
Målet var å oppnå en felles oppfatning innad i gruppen om hvordan gruppearbeidet skulle utføres.
I tillegg skulle den bestemme hvilke krav gruppens medlemmer kunne stille til hverandre.
Hovedemnene var leveranse, læring og trivsel.
Samarbeidsavtalen ligger i vedlegg \ref{Ved:samarbeidsavtale}.
Viktigheten av god kommunikasjon innad i gruppen ble gjenspeilet i flere punkter i kontrakten. \\

Et viktig poeng med EiT er tverrfaglig samarbeid. Dette målet ble reflektert i punkt 8 i samarbeidskontrakten: \textit{Gruppen skal drive kunnskapsutveksling gjennom diskusjon og drøfting av faglig innhold, samt utfordre hverandre til å gå utenfor den faglige komfortsonen.} Blant annet ville det være viktig å utnytte Ingelin sine kunnskaper om generell kjemi for at resten av gruppen skulle komme raskt igang med arbeidet.
Flere av gruppemedlemmene har vært takknemlig for hvor enkelt det har vært å spørre om andres kompetanse under prosjektarbeidet.
Karsten som har jobbet mye med de biologirettede delene av prosjektrapporten har gitt uttrykk for at Ingelin sine forhåndskunnskaper var til stor hjelp, og at Ingelin var villig til å avbryte eget arbeid for å hjelpe til.
Det er også av interesse å høre hva de ulike gruppemedlemmene har av faglige lidenskaper for å kunne skape et godt sosialt og løsningsorientert miljø i gruppen. \\

Kommunikasjon er ikke bare muntlig kunnskapsutveksling, ei heller formelt nødvendigvis.
Et eksempel på hvordan dette har foregått er de tekniske løsningene rundt dokumentdeling og ferdigstilling av prosjekt- og prosessrapporten.
For å legge til rette for individuelt arbeid valgte gruppen å bruke en programvare som heter Git.
Dette programmet sørger for versjonskontroll og at dokumentene ble delte samt oppdaterte.
Programvaren tok litt tid å sette opp, men etter dette har medlemmene kunnet skrive selvstendig samtidig som det ikke var behov for å bekymre seg for at noen andres arbeid ble slettet i prosessen.
Ferdigstillingen av rapportene har blitt gjort i \LaTeX.
Begge disse tekniske løsningene har hatt noen implementeringsutfordringer som Jonas og Martin har jobbet med underveis.
Det har krevd at de begge bidro med å dele sin forhåndskunnskap og samarbeide om å finne ny informasjon som kunne belyse problemer som oppstod underveis. \\


Siden samarbeidsavtalen var noe som ble utviklet tidlig, og før prosjektarbeidet hadde kommet skikkelig i gang, er det vanskelig å se om dette punktet har ført til noen vesentlig utvikling i gruppen.
Likevel viser dette seg å være en av flere årsaker til åpen kommunikasjon i gruppen. \\


























