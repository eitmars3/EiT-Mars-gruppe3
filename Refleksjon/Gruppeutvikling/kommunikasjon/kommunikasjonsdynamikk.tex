\subsubsection{Kommunikasjonsdynamikk}

Trenden med at medlemmene i gruppen hadde ulik mengde initiativ i diskusjoner var grunnlag for regel \#11.: \textit{Alle i gruppen skal vise hverandre respekt. Dette gjennom å lytte, bidra med egne meninger, gi konstruktiv kritikk og bygge videre på andres id\'{e}er.} 
Gruppen består av medlemmer som opprinnelig hadde veldig ulike \textit{tilnærminger} til åpne diskusjoner og samtaler.
For eksempel sier både Karsten og Jonas at de er klar over at de tar mye initiativ og \textit{tar stor plass} i diskusjoner.
Derimot forholder Simen og Ingelin seg forholdsvis stille, og tar gjerne ikke ordet like ofte.
Anna og Martin plasserer seg omtrent midt mellom de to andre \textit{grupperingene}.
Dette mønsteret i kommunikasjonen er noe gruppen var klar over ganske tidlig, og har jobbet med i løpet av prosjektperioden.
Som Jonas skriver i refleksjonen om hva han lærte om gruppen i løpet av 14. januar:
"Vi har veldig ulike bakgrunner, både sosialt og faglig. Likevel virker alle åpne. Det er flere av oss som liker å ta ordet. Det er en bra start, men vi må være obs. på at alle får sagt det de ønsker."
En aksjon for å oppnå punkt \#11 var å la ordet gå på rundgang rundt bordet når viktige avgjørelser skulle tas. Da fikk alle medlemmene muligheten til å legge frem sitt synspunkt. På denne måten ble også medlemmene \textit{tvunget} til å ha en mening om alt som ble diskutert. \\

Andre kommunikasjonstrender som har blitt observert i gruppen er en økt grad av frie samtaler, småprat og humor. Spesielt Martin og Karsten har hatt lett for å skape digresjoner, slik at resten av gruppen har måttet be dem om å beholde fokus. 
Kommunikasjonsmønsteret i gruppen har under hele prosessen endret seg.
Noen ganger gradvis og sakte, andre ganger raskere. \\

For å kontrollere den naturlige dynamikken i gruppen, ble punkt \#10: \textit{Gruppen skal ha rullerende møteleder og sekretær for hver landsbydag. Møteleder har hovedansvar for å overse arbeidsplan og trivsel samt å være ordstyrer.}, innført både som aksjon for å oppnå punkt \#11 og for å bli mer tidseffektive. En ordstyrer vil kunne fordele taletid mellom alle medlemmene i gruppen, og også oppmuntre medlemmer som har vært stille lenge til å komme med innspill. Dette punktet i kontrakten ble overholdt gjennom hele EiT-perioden. Ved å innføre ordningen har kommunikasjonen trolig vært mer åpen og flytende enn uten.
