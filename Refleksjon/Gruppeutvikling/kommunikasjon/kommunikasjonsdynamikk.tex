\subsubsection{Kommunikasjonsdynamikk}

Trenden med at medlemmene i gruppen hadde ulik mengde initiativ i diskusjoner var grunnlag for regel \#11. i samarbeidsavtalen: \textit{Alle i gruppen skal vise hverandre respekt.
Dette gjennom å lytte, bidra med egne meninger, gi konstruktiv kritikk og bygge videre på andres id\'{e}er.} 
Gruppen består av medlemmer som opprinnelig hadde veldig ulike \textit{tilnærminger} til åpne diskusjoner og samtaler.
For eksempel sier både Karsten og Jonas at de er klar over at de tar mye initiativ og \textit{tar stor plass} i diskusjoner.
Derimot forholder Simen og Ingelin seg forholdsvis stille, og tar gjerne ikke ordet like ofte.
Anna og Martin plasserer seg omtrent midt mellom de to andre \textit{grupperingene}.
Dette mønsteret i kommunikasjonen er noe gruppen var klar over ganske tidlig, og har jobbet med i løpet av prosjektperioden.
Som Jonas skriver i refleksjonen om hva han lærte om gruppen i løpet av 2. landsbydag:
\textit{"Vi har veldig ulike bakgrunner, både sosialt og faglig. Likevel virker alle åpne. Det er flere av oss som liker å ta ordet. Det er en bra start, men vi må være obs. på at alle får sagt det de ønsker."}
En aksjon for å oppnå punkt \#11 var å la ordet gå på rundgang når viktige avgjørelser skulle tas eller viktige temaer skulle diskuteres.
Slik fikk alle medlemmene muligheten til å legge frem sitt synspunkt, samt "tvunget" til å ha en mening om alt som ble diskutert.
Fordelen med dette er at ingen av gruppemedlemmene fikk muligheten til å "melde seg ut" av diskusjonene. \\

For å kontrollere at \#11 i samarbeidsavtalen ble overholdt ble punkt \#10 lagt til: \textit{Gruppen skal ha rullerende møteleder og sekretær for hver landsbydag. Møteleder har hovedansvar for å overse arbeidsplan og trivsel samt å være ordstyrer.}
Ordstyreren hadde som oppgave å fordele taletid mellom alle medlemmene i gruppen, og eventuelt oppmuntre medlemmer som hadde vært stille en stund.
Ordstyreren fikk fullmakt til å avbryte andre gruppemedlemmer når han/hun følte at dette var nødvendig.
Det kan likevel være utfordrende for ordstyrer å henge med på hvem som vil komme med innspill til hva i diskusjoner, spesielt når de fleste gruppemedlemmene hadde meninger om saken.
Jonas, som hadde erfaring med samarbeid fra tidligere, nevnte tidlig i prosjektet at innføring av møtetegn kunne være en god id\i{e} om diskusjonene ble for "kaotisk".
Selv om dette ikke ble brukt metodisk, var det tidvis i bruk.
Simen, som ikke er så flink å "presse seg inn" i diskusjonen, skiver for eksempel i refleksjonen fra 3. landsbydag: \textit{"Brukte møtetegn i dag og etter litt tid fikk jeg ordet, og alle fulgte med. Dette kan være fordi gruppen passer på å gi plass til de som vil snakke. Dette var mer effektivt enn å prøve å bryte inn i diskusjonen."}
Gruppen føler at innføringen av ordningen med ordstyrer trolig førte til en mer åpen og flytende kommunikasjon.
Mot slutten av prosjektet, spesielt da gruppen møtte utenfor de fastsatte onsdagene, var ikke behovet for ordstyrer like stort lengere.
Kanskje hadde kommunikasjonsdynamikken "satt seg" ved dette tidspunktet og alle i gruppen var mer oppmerksom på å gi alle plass.
