\subsubsection{Sosiogram}
På landsbydag 4 jobbet gruppen med å utarbeide en problemstilling.
Dette var et arbeid som foregikk i fellesskap under en åpen diskusjon.
I løpet av rundt 5 minutter hadde Erik observert kommunikasjonsmønsteret i gruppen og laget et sosiogram underveis.
Et sosiogram er en grafisk fremstilling av hvordan de ulike grupemedlemmene henvender seg til hverandre og gruppen som helhet når de snakker i en åpen samtale.
Erik spurte gruppen om den kunne ta en liten pause og se på sosiogrammet i fellesskap før den fortsatte arbeidet.

\begin{figure}
\label{fig:sosiogram}
\caption{Sosiogram for gruppen fra diskusjon under utarbeiding av problemstilling den - landsbydagen.}
\begin{center}
%	\includegraphics[width=0.45\textwidth]{sosiogram.jpg}
\end{center}
\end{figure}

Sosiogrammet til gruppen vises i figur \ref{fig:sosiogram}. Det er tydelig at medlemmene i gruppen fokuserte mye på å snakke med enkeltpersoner og ikke gruppen som helhet.
I tillegg var det ingen som henvendte seg direkte til hverken Anna eller Ingelin, på tross av den ellers direkte kommunikasjonen.
Ellers kommer det frem at Jonas og Karsten hadde mye toveis kommunikasjon seg imellom.
Begge disse punktene ble gruppemedlemmene enige om at de ønsket å forbedre.

Etter denne hendelsen har gruppen i stor grad endret kommunikasjonsmønsteret.
Enkeltmedlemmene er bevisst på hvordan de ønsker å kommunisere med hverandre, og møtelederen er ekstra observant for å unngå at noen blir ekskludert fra diskusjoner.
Dette fører til en mer åpen kommunikasjon medlemmene imellom, og gjør det enklere for alle å bidra med det de ønkser.