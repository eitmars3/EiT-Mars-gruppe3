\subsubsection{Grunnregler for effektive grupper}

En øvelse som ble gjennomgått i gruppa var å diskutere et sett med regler hver enkelt kan følge for å oppnå effektivt gruppearbeid.
Dette ble gjordt for at samtlige gruppemedlemmer tidlig i prosjektet skulle bli mer bevisste på hvordan man kan forholde seg for å unngå kommunikasjonsproblemer.
Reglene beskriver visse adferder som er med på å gjøre kommunikasjon i en gruppe mer tydelig \cite{schwarz}:

\paragraph{Sjekk antakelser og slutninger}
Man kan gjøre flere slutninger daglig uten å være klar over det, samt at de kan være intrikate og fjerne fra observerte data.
At man ikke engang er klar over dette gjør at slutningene kan oppfattes som fakta.
Dermed kan man, ved å stenge ute deler av et budskap og misforstå det i sin helhet, danne seg et feilaktig inntrykk av intensjonene til den som kommer med budskapet.
For å unngå dette må man alltid være åpen for å tolke på flere måter.
Er man i tvil, kan man også spørre om man har tolket budskapet riktig.

\paragraph{Del all relevant informasjon}
Om ikke all relevant informasjon foreligger, vil gruppas kollektive misforståelse av et emne kunne bidra til å ta feil beslutninger.

\paragraph{Eksemplifiser og bli enige om viktige begreper}
Å eksemplifisere mens man forklarer gjør det lettere for motparten å validere utsagnene, siden eksemplene gjerne henviser til observerte eller observerbare data.
Dette bør gjerne brukes for å tydeliggjøre sentrale begreper, da det er viktig for å få en felles oppfattelse av disse så tidlig som mulig.

\paragraph{Forklar ditt resonnement og din intensjon}
Et utsagn kan tolkes på flere måter av mottakeren.
Mottakeren spør heller ikke alltid om budskapet er forstått riktig, men heller feiltolker, jfr. regelen "Sjekk antakelser og slutninger".
Det kan derfor være til hjelp om den som kommer med utsagnet også utdyper det på eget initiativ.

\paragraph{Fokus på interesser, ikke posisjon}
Når et fellesskap skal komme til enighet hender det at noen er tidlig ute med et forslag til løsning.
Da stiller vedkommende seg i en posisjon hvor det er ukjent for de andre hvilke interesser denne personen legger til grunn for en slik løsning.
Problemet er at slike forslag og motforslag kan være inkompatible selv om interessene egentlig stemmer overens.
Hvis man åpner med å dele interesser med de andre, er det lettere å utarbeide løsninger i fellesskap.
Om andre tidlig kommer med løsninger istedet for å begrunne dem, kan man spørre om interessene som ligger til grunn.

\paragraph{Kombiner sakføring og granskning}
Diskusjoner kan noen ganger arte seg slik at man snakker forbi hverandre.
For å unngå dette, kan man invitere motparten til å granske ens egne utsagn, gjerne med å stille oppfølgingsspørsmål.
Dette virker sakførende, siden det bidrar til en felles forståelse av saken mellom partene.
Retoriske spørsmål vil kunne virke sakførende, men ikke granskende, da det kan få motparten til å bli defensiv og holde tilbake informasjon.
Grunnen til at man bør granske, er at man kan avklare om motparten har forskjellige antakelser eller baserer seg på annen informasjon enn en selv.

\paragraph{Finn måter for å sjekke misforståelser}
En uenighet mellom to personer kan fort havne i en posisjonsorientert og låst situasjon.
Partene kan mene at de selv har rett, uansett hvilke beviser som måtte foreligge.
For å løse dette må man blir enige om hva problemet omfatter, for begge kan ha tatt hensyn til hver sine aspekter.
Videre må man prøve å finne ut hvordan begge samtidig kan ha rett.
Ved å lage en testplan sammen kan man da finne ut om noen av partene skulle vise seg å ha ukorrekt informasjon.

\paragraph{Diskuter udiskutable emner}
I grupper kan det være saker som påvirker gruppas oppgave, men som det er vanskelig å diskutere i fellesskap.
Det er gjerne saker som skyldes et medlem, og man vegrer seg for å snakke med denne personen om det.
Emnet holdes kanskje udiskutert, eller man tar det opp med andre i gruppa, noe som ikke løser problemet.
Man kan gjøre gruppa oppmerksom på at man skal diskutere noe udiskutabelt.
Alternativt kan man si ifra til den det gjelder at man kommer til å ta opp temaet i fellesskap.
Det siste gir personen mulighet til å forsvare seg i diskusjonen som kommer, noe som vil virke mest mulig skånsomt.

\paragraph{Bruk en beslutningsprosess for passe enighet}\label{beslutningsprosess}
Selv om alle i en gruppe idéelt sett kunne vært enige, som når man har oppnådd konsensus, kan det være krevende eller umulig å oppnå.
I en konsultativ beslutningsprosess tar lederen en avgjørelse etter at temaet er diskutert.
For en delegativ beslutningsprosess gir lederen beslutningsoppgaven til en del av gruppen, gjerne gitt endel begrensninger.
Noen gruppemedlemmer kan kreve å få viljen sin for å kunne utføre sin oppgave, og noen ganger er konsensus nødvendig.
Man må avveie hvor stor grad av enighet som trengs i forhold til hvor mye som skal til for å oppnå dette.