\subsubsection{Grunnregler for effektive grupper}

En øvelse som ble gjennomgått i gruppa var å diskutere et sett med regler hver enkelt kan følge for å oppnå effektivt gruppearbeid.
Dette ble gjordt for at samtlige gruppemedlemmer tidlig i prosjektet skulle bli mer bevisste på hvordan man kan forholde seg for å unngå kommunikasjonsproblemer.
Reglene beskriver visse adferder som er med på å gjøre kommunikasjon i en gruppe mer tydelig [Schwarz 2002]:

\paragraph{Sjekk antakelser og slutninger}
Man kan gjøre flere slutninger daglig uten å være klar over det, samt at de kan være intrikate og fjerne fra observerte data.
At man ikke engang er klar over dette gjør at slutningene kan oppfattes som fakta.
Dermed kan man, ved å stenge ute deler av et budskap og misforstå det i sin helhet, danne seg et feilaktig inntrykk av intensjonene til den som kommer med budskapet.
For å unngå dette må man alltid være åpen for å tolke på flere måter.
Er man i tvil, kan man også spørre om man har tolket budskapet riktig.

\paragraph{Del all relevant informasjon}
Om ikke alle på gruppa deler informasjon de har, vil gruppas kollektive forståelse av et emne kunne bidra til å ta feil beslutninger.

\paragraph{Eksemplifiser og bli enige om viktige begreper}
Å eksemplifisere mens man forklarer gjør det lettere for motparten å validere utsagnene, siden eksemplene gjerne henviser til observerte eller observerbare data.
Dette bør gjerne brukes for å tydeliggjøre sentrale begreper, da det er viktig for å få en felles oppfattelse av disse så tidlig som mulig.

\paragraph{Forklar ditt resonnement og din intensjon}

\paragraph{Fokus på interesser, ikke posisjon}

\paragraph{}

\paragraph{Finn måter for å sjekke misforståelser}

\paragraph{Diskuter udiskutable emner}

\paragraph{}