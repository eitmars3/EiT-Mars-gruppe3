\subsection{Kommunikasjon}
Kommunikasjon, roller og normer, personligheter

Hvordan gruppa har snakket med hverandre.
Informasjonsflyt.
Hvem sier hva.
Initiativ.
Kommunikasjonsmønster.
Hvordan formidler vi kommunikasjon til hverandre?
Er vi flinke til å være tydelig og oppklarende, eller bygger vi på antagelser og egne tolkninger?
Hvilke roller har de forskjellige medlemmene i gruppa.
Er de formet av oss selv eller faget og rammene vi har gitt oss?
Faglig nivåforskjell/hetrogenitet/forkunnskaper.

\begin{itemize}
\item Teori: stjernekommunikasjon
\item Figur: Elementær kommunikasjonsteori (se bilde)
\item Konkret: sosiogram + effekter
\item Konkret: SITRA, effekten av dette på gruppen, semantikk
\item Konkret: Samarbeidskontrakt (spesifikk respect for andres forslag)
\item Konkret: 8 regler for å snakke til folk, effekten av disse (bruk som teori evt)
\item Konkret: Utviklingen av samarbeidsindikatoren
\item Abstrakt: Pros and cons ved grundig planlegging
\item Ønske om å engasjere alle, alle skal få bidra med sitt og bli hørt. Oppfordrer til å delta.
\item Folk kommuniserer anderledes, “introspektiv” tenker før snakker, “ekstrospektiv” snakker for å tenke
\item Kommunikasjon: Får de som snakker mindre mere “priority?” F.eks. at Jonas velger Ingelin over Karsten, siden Karsten snakker mere
\end{itemize}