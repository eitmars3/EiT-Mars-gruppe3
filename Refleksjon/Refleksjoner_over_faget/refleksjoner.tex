\subsubsection{Anna Solveig Julia Solheim Testani\`{e}re}
Dette er første gang jeg som student har jobbet i en gruppe.
Jeg tvilte ikke på at dette kom til å bli lærerikt, noe det ble.
Først og fremst har jeg lært at en ikke må bytte sin personlighet eller å følge en unaturlig oppskrift for å være det perfekte gruppemedlem. 
Det handler mer om hvordan man må respektere og akseptere forskjellige trekk ved hverandres personlighet og se hvordan disse kan utnyttes på en positiv måte. 
Jeg har lært å gi tilbakemeldinger, noe som viste seg å være enklere enn det jeg hadde trodd.
I tillegg har jeg lært å få tilbakemeldinger. 
Selv om dette fortsatt er litt ubehagelig, har jeg et mye mer avslappende forholdet til kritikk nå i forhold til begynnelsen. 
Mitt personlig mål var å gå utenfor min faglige komfortsone og dykke dypere inn i andre fagfelt. 
Dette følte jeg at jeg klarte å oppnå ved å sette meg inn i både biologi samt roverens teknologi, felt jeg hadde svært liten innsikt i til å begynne med.
Jeg er fornøyd med min personlige utvikling og gruppens samlede innsats og synes at EIT har vært en veldig positiv opplevelse.\\
\subsubsection{Ingelin Garmann}
Eksperter i team har vært et tidkrevende og mentalt krevende fag. 
Spesielt i starten var det en uvant påkjenning å reflektere, diskutere og gi så mye av seg selv i en gruppe. 
På grunn av god innstilling, respekt og åpenhet mellom gruppemedlemmene ble det raskt veldig enkelt å slappe av og være seg selv.
Jeg er fast bestemt på at oppbyggingen av EiT med tilrettelegging for god dialog i samarbeidet har dannet grunnlaget for at vår gruppe har utviklet seg til det den er i dag.
Jeg har lært viktige momenter jeg vil ha i bakhodet når jeg i framtiden skal samarbeide med andre.
Jeg har også fått utviklet mine lederegenskaper og fått bekreftet at jeg er en person som kan fremstå stille, men likevel har gode ideer å komme med.
Faglig var jeg den som hadde best forkunnskaper i biologi og biokjemi, men følte meg absolutt ikke selvsikker innen feltet.
Det tok dermed tid før jeg skjønte at jeg faktisk hadde mye kunnskap å dele, og at man ikke trenger bachelorgrad i et emne før man sitter på mer kunnskap enn andre.
Når andre studenter spør meg om min erfaring med EiT kommer jeg til å svare at om man møter opp med god innstilling og villighet til å lære, så vil man få mye viktig ut av emnet, som både kan benyttes som medmenneske og arbeidspartner.\\
\subsubsection{Jonas Sandøy Misund}
I løpet av prosjekt- og prosessarbiedet har jeg blitt mer bevisst på hvordan jeg arbieder med og påvirker en gruppe rundt meg.
Det har vært lærerikt å være så oppmerksom på selve gruppedannelsen i løpet av tiden med EiT.
I tillegg har jeg lært mer om grunnleggende biologi, og hvilke store utfordringer som finnes ved å selv lete etter spor av liv, eller forhold hvor liv kan oppstå og bestå.
Presentasjonene fra de andre gruppene har også vært et positivt bidrag til både kunnskap og interesse for fagfeltet til landsbyen.\\
\subsubsection{Karsten Olav Kjensmo}
Faget har fasilitert at studenten får en sjanse til å undersøke seg selv grundig i gruppesamarbeid. 
Som informatikkstudent- og som frivillig i verv, hvor jeg har fått mye erfaring med ledelse, har jeg vært involvert i gruppearbeid hvert semester. 
Ingen av disse vervene eller fagene har vært lagt opp rundt det å lære studenten gruppearbeid, og det har kunne merkes. 
Jeg mener at EiT er et fag som hører til i første eller andre semester, i hvert fall når det gjelder informatikk. 
Jeg har fått øye på mye, både hva angår min egen personlighet i gruppeforstand, samt hvordan en gruppe kan utvikle seg generelt. 
Selv mener jeg at jeg kan være en dominerende personlighet, og spiser litt for mye av samtalekaken til tider. 
Dette kan skyldes de litt stille og sky personlighetene jeg er vant med fra gruppearbeid på informatikk, og det kan skyldes erfaringen med ledelse i verv, hvor jeg er vant til å avbryte for å holde flyt og inkludere alle. 
Dette er et aspekt av meg selv jeg kunne ha roet ned i EiT. Det inngikk litt i planen - alle medlemmene på gruppen satt et personlig mål tidlig i prosjektet, og mitt var å ikke avbryte og snakke for mye. 
Jeg har blitt bedre på dette, men det er fortsatt et problem jeg må jobbe med. 
Folk som snakker for mye fyller fort samtalerommet, og dette kan gjøre at andre snakker mindre som kanskje burde snakke mer. 
Jeg har videre lært mye om å forholde seg diplomatisk så mye som mulig, og benytte den sokratiske diskusjonsmetoden for å holde konflikter så objektive som mulig.\\

\subsubsection{Martin Nordal}
EiT skulle vise seg å være en litt annerledes opplevelse enn i andre fag med gruppesamarbeid.
Selv om jeg har vært med på flere gruppeprosjekter tidligere, var dette for første gang med fokus på selve samarbeidsprosessen.
EiT har vært et ganske praktisk rettet fag, istedet for å lese særlig mye om gruppedynamikk har vi heller testet det ut, under fagets rammer.
Dette føler jeg har vært behagelig, før vi mot slutten ble nødt til å møtes langt oftere for å rekke å levere rapportene.
Det har vært satt av betydelig tid til grupperefleksjon i hver landsbydag.
Ved å aktivt bli satt til å reflektere, samt å tenke litt tilbake til tidligere erfaringer med gruppearbeid, føler jeg at faget har gitt mer innsikt i fenomenet samarbeid.
Jeg har fått tilbakemeldinger på egen væremåte, og med oppfølging har jeg kunnet samkjøre meg med de andre og for eksempel øvd på å bli en bedre ordstyrer, disse metodene blir nyttige i senere anledninger.
Likevel synes jeg at dette faget kunne vært holdt tidligere, hvertfall i datateknikkstudiet, da lærdommen kunne vært nyttig i alle studiets gruppearbeider.\\

\subsubsection{Simen Løvøy Hegge}
Før jeg tok dette faget trodde jeg at jeg hadde en tendens til å avbryte mye i diskusjoner og samtaler.
Etter en stund lærte jeg at dette ikke var tilfelle, heller det motsatte.
Jeg avbrøt ikke i det hele tatt og ble definert som en av de mest "stille" i gruppen.
Dette har gitt meg mulighet til å trene på å delta mer med id\'{e}er og innspill.
Jeg oppdaget blant annet at jeg ikke alltid trengte å ha en ide ferdig tenkt ut for å kunne dele den med gruppen.
En av fordelene å jobbe i en slik gruppe er at andre kan bygge videre på mine ufullstendige ideer.\\

En annen viktig ting jeg lærte var at min kunnskap er unik (til en viss grad).
I mitt fagfelt, konstruksjon, er det ganske viktig at jeg alltid regner rett.
Det er svært ugunstig at konstruksjonen bryter sammen og de konsekvenser dette måtte medføre.
Derfor har jeg aldri helt følelsen av å ha nok kunnskap, spesielt ikke når man vanligvis kun jobber sammen med andre som har mye kunnskap på mitt felt.
Arbeidet i denne gruppen har vist meg at jeg faktisk har en del kunnskap som ikke nødvendigvis alle andre sitter inne med. \\

Alt i alt har jeg lært en del om meg selv, og da spesielt hvordan jeg arbeider i en gruppe og hvordan jeg kan dele min kunnskap og mine id\'{e}er.
Selv om jeg ikke hadde denne oppgaven som førstevalg, er jeg svært glad for at jeg havnet der jeg havnet og fikk jobbe sammen med min gruppe. \\
