\subsubsection{Hva har gruppen lært?}

Her skal det diskuteres hva gruppen som en helhet har tilegnet seg av kompetanse innenfor prosessarbeid.
Den nye kompetansen, dog imponerende stor, innenfor prosjektet taler for seg selv i prosjektrapporten.
 
Fellestrekket for alle gruppemedlemmene var at de kom inn med en positiv innstilling til faget.
Selv om de fleste hadde hørt en del negative fortellinger om EiT på forhånd var de innstilt på å gjøre det beste ut av det.
Det å reflektere rundt egen og andres væremåte var uvant for de fleste gruppemedlemmene i starten, men metoden for dette ble raskt innlært.
Disse refleksjonene viste seg å være uvurderlige ved belysning av samarbeidet, både det som fungerte godt og det som fungerte dårlig.

Gjennom refleksjonene, og diskusjoner rundt disse, har gruppemedlemmene oppnådd en dypere innsikt i hvordan egen væremåte påvirker samarbeidet.
Dette på grunn av muligheten til å få konstruktiv tilbakemelding på egen væremåte, både gjennom ris og ros, for så å kunne justere væremåten over tid med mulighet for nye input.
Ved å knytte denne innsikten opp mot relevant gruppeteori har gruppemedlemmene også fått en mer inngående forståelse for hvorfor samarbeidet gikk så \emph{"smertefritt"} som det gikk, samt et faglig begrepsapparat knyttes til dette.

I tillegg har gruppedeltakerne fått en dypere innsikt i viktigheten av å definere ord og uttrykk.
Den tverrfaglige sammensetningen førte til at selv om alle gruppedeltakerne snakket norsk, snakket de ikke nødvendigvis samme språk.
Heldigvis ble viktigheten av dette erfart gjennom nysgjerrighet og oppfølgingsspørsmål, og ikke gjennom konfliktene dette kunne forårsaket.

Den tverrfaglige sammensetningen viste seg å ha en annen fordel også.
Gruppemedlemmene fikk en forsterket følelse av at deres faglige kompetanse var spesiell.
Diskusjoner underveis i prosjektet avdekket at alle hadde en "unik" kompetanse.
  
Alt i alt har gruppen tilegnet seg verdifull kompetanse og erfaring som kan vise seg svært nyttig i arbeidslivet.
Avslutningsvis er det også verd å nevne at gruppemedlemmene ble mer enn bare kollegaer, de ble venner.
