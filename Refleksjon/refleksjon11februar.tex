EiT onsdag 11. februar Landsbydag #6


Ordstyrer: Simen 
Referent: Ingelin

INNSJEKK
Karsten: Spist twisten fra forrige gang, men tok med en nyinnkjøpt. Looka litt siden forrige uke.
Ingelin: Endelig sett Interstellar. Og perset i knebøy!
Martin: Byttet internettleverandør. Badet går sakte. Skal til Paris i helgen. Må få ferdiggjort øving før det!
Simen: Har det bra. Søker sommerjobb. Kjøpt bok for 810 kr til skole :(
Jonas: Har Isfit overnatttingsgjester, som blokkerer kjøleskapet med senga ;) No milk for Jonas. Vært i Oppdal på ski fra søndag til tirsdag.
Anna: Beklager at jeg kom for sent, sendte melding til Ingelin! Satt seg mål å løpe halvmaraton. Har sluttet å røyke (røykfri siden mandag)!

Mål for dagen:
- Sette opp Git
- Få laget rammeverk for prosjektrapport i LaTex.
- Fordele oppgaver og finne ut mer om hvert sitt ansvarsområde - Se Stephen Hawkings «Into the Universe» episode «Aliens»

Utsjekk/Individuell refleksjon:
Anna:
Beklager forsinkelsen. Fint at vi delte i tre grupper; struktur, effektivitet. Ikke alene, men likevel delt. Bra forhåndsjobb forrige uke. Bra vi gikk opp på R73; mindre slitsomt. Ønsker å unngå å jobbe alene. Vil fortsette med at hovedansvaret er delt. Foreslår at en med hovedansvar forblir på sin gruppe og jobber med en med annet hovedansvar. Simen mangler litt autoritet som ordstyrer, for snill. Kunne vært klarere på hvordan ting skulle blitt gjort; «nå tar vi en runde». Vi burde vært mer strukturert. Liker hver enkelt i gruppen; onsdagene blir dermed kjekke.
Karsten:
Til Simen: vanskelig å vite når man skal bryte inn. Men enig med Anna, kunne sagt «shit the fuck up». Bra å jobbe der oppe (R73). Situasjon: at vi var usikre med problemstillingen; ustrukturert diskusjon. Bra at Jonas tegnet opp!
Martin:
Det var godt å komme i gang. Liten støkk i oss at vi er litt usikre på om problemstillingen holder. Godt at vi fikk pratet om det. Godt å bytte rom, men det ble veldig frie tøyler og nesten sånn kosetime. Varierende effektivitet. Ordstyrer kunne vært strengere, men ikke noe negativt i at det ble friere i dag.
Jonas:
Fikk litt trang til å kun se på YouTube filmer om halvveis relevante tema da vi gikk opp. Fikk litt feil motivasjon! Karsten skulle presentere «monty hall» problemet, og Jonas kom med løsningen før problemet ble formulert. Tar selvkritikk. Ønsker å gå gjennom grupperefleksjon fra tidligere. Var lite bevisst på hvordan han selv jobbet fordi ting som har blitt tatt opp før var glemt. Vi burde ta en recap på det etter innsjekk de neste gangene. Ønsker å vite hvorfor Anna var for sen, kunne gitt beskjed til alle. Lurer på om vi var for rask i å godta hans ide om innsnevring av problemstillingen. Burde vi brukt mer tid? Hoppet vi på første og beste?
Simen:
Enig i det alle har sagt om han som ordstyrer. Føler seg ikke som en naturlig ordstyrer, og ikke klarte å leve opp til standarden. Men vi jobbet jo på en annen måte i dag. Mistet kanskje litt effektivitet ved å gå opp på R73. Litt ustrukturert i diskusjonene. Har følt seg litt mindre motivert og trøtt i dag.
Ingelin:
Kritikk til meg selv: Så ikke melding fordi jeg hadde pakket vekk mobilen klar til EiT. Anonym ordstyrer, lite arbeid på sekretær. Bra vi fordelte oppgaver slik at vi fikk progressjon på alle hoveddeler. Bra vi presenterte for alle i plenum slik at alle er up to speed. Dagen har gått fort.


Grupperefleksjon:
Situasjon: Vi bestemte oss for å forflytte oss till grupperom R73
Anna syntes ikke det var noe vondt i det. Ingelin og Anna fikk jobbet bra! Jonas syntes det var bra, det var så stille nede i K3. Ingelin og Anna jobbet så godt helt fra starten. Var det derfor vi gikk glipp av at vi var ineffektive? For liten observasjon av de andre? Tiden gikk fort for alle, og vi syntes det var gøy. Vi tok ingen pause, er det dumt? Burde vi hatt faste pauser? Jonas tror det er lurt. Ordstyrer burde følge litt med og kunne foreslå pauser og få inntrykk av stemningen. Karsten kommer med forslag om luftepause hver 90 min. Kanskje vi ville jobbet mer effektivt da. Martin er redd for at om vi innfører pause er det ikke sikkert det blir gjennomført. Han syns det hadde vært bra å kunne strekke litt på beina og prate litt sjit. Jonas syns pausene ville være et bra tidspunkt å spørre hvordan ting går, ta en innsjekk.
Situasjon: En med hovedansvar skal alltid jobbe på den delen, mens person to kan rullere
Martin foreslår at alle kan komme med ønsker. Jonas tror det er et godt forslag, men man må være flink på å få deltatt på alle tre delene. Anna er enig i at det ikke trenger å være satt før dagen starter, men at det burde bestemmer raskt av ordstyrer på starten av dagen. Ingelin er enig i at gruppene kan bestemmes på starten av dagen.

Konklusjon/Aksjon for neste gang:
Ha regelmessige pauser
Ordstyrer tar innsjekk ila dagen
Ordstyrer fordeler grupper på starten av dagen
Gruppen spleiser på kompendium for prosess

Utsjekk:
Jonas: Det var morsomt å snakke med Karsten om alt! Være fun fact fysiker. Litt sliten, groggy i bakhodet, men prøvde å skule det så godt han kunne. Skal se på LaTex til neste gang.
Simen: Har vært litt trøtt, men det har gått greit. Ønsker å prøve seg som ordstyrer igjen for å utvikle.
Ingelin: Dagen har gått fort så det må ha vært gøy! Fornøyd med å være sekretær på en dag der det var lite å skrive
Anna: Kjekt å jobbe med en person. Gleder seg til å se Martin som ordstyrer og Jonas som referent.
Karsten: Koselig dag, gikk fort. Chill at vi ikke var hele gruppen hele tiden. Destressing. Men vi var flinke til å holde hverandre a jour.
Martin: Bra dag fordi den gikk fort. Bra med twist. Gira på å komme seg hjem!