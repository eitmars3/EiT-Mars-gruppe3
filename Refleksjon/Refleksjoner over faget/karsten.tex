\subsubsection{Karsten}

\paragraph{Reflekjsoner om faget}
Faget har fasilitert at studenten får en sjanse til å undersøke seg selv grundig i gruppesamarbeid. Som informatikkstudent- og som frivillig i verv, hvor jeg har fått mye erfaring med ledelse, har jeg vært involvert i gruppearbeid hvert semester. Ingen av disse vervene eller fagene har vært lagt opp rundt det å lære studenten gruppearbeid, og det har kunne mærkes. Jeg mener at EiT er et fag som hører til i første eller andre semester, i hvert fall når det gjelder informatikk. Jeg har fått øye på mye, både hva angår min egen personlighet i gruppeforstand, samt hvordan en gruppe kan utvikle seg generelt. Selv mener jeg at jeg kan være en dominerende personlighet, og spiser litt for mye av samtalekaken til tider. Dette kan skyldes de litt stille og sky personlighetene jeg er vannt med fra gruppearbeid på informatikk, og det kan skyldes erfaringen med ledelse i hverv, hvor jeg er vant til å avbryte for å holde flyte og inkludere alle. Dette er et aspekt av meg selv jeg kunne ha roet ned i EiT. Det inngikk litt i planen - alle medlemmene på gruppen satt et personlig mål tidligi prosjektet, og mitt var ikke å avbryte og snakke for mye. Jeg har blitt bedre på dette, men det er fortsatt et problem jeg må jobbe med. Folk som snakker for mye fyller fort samtalerommet, og dette kan gjøre at andre snakker mindre som kanskje burde snakke mer. Jeg har envidere lært mye om å forholde seg diplomatisk så mye som mulig, og benytte den sokratiske diskusjonsmetoden for å holde konflikter så objektive som mulig. 
