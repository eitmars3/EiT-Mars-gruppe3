\subsubsection{Reflekjsoner om faget}

\paragraph{Karsten}
Faget har fasilitert at studenten får en sjanse til å undersøke seg selv grundig i gruppesamarbeid. Som informatikkstudent- og som frivillig i verv, hvor jeg har fått mye erfaring med ledelse, har jeg vært involvert i gruppearbeid hvert semester. Ingen av disse vervene eller fagene har vært lagt opp rundt det å lære studenten gruppearbeid, og det har kunne merkes. Jeg mener at EiT er et fag som hører til i første eller andre semester, i hvert fall når det gjelder informatikk. Jeg har fått øye på mye, både hva angår min egen personlighet i gruppeforstand, samt hvordan en gruppe kan utvikle seg generelt. Selv mener jeg at jeg kan være en dominerende personlighet, og spiser litt for mye av samtalekaken til tider. Dette kan skyldes de litt stille og sky personlighetene jeg er vant med fra gruppearbeid på informatikk, og det kan skyldes erfaringen med ledelse i verv, hvor jeg er vant til å avbryte for å holde flyt og inkludere alle. Dette er et aspekt av meg selv jeg kunne ha roet ned i EiT. Det inngikk litt i planen - alle medlemmene på gruppen satt et personlig mål tidlig i prosjektet, og mitt var å ikke avbryte og snakke for mye. Jeg har blitt bedre på dette, men det er fortsatt et problem jeg må jobbe med. Folk som snakker for mye fyller fort samtalerommet, og dette kan gjøre at andre snakker mindre som kanskje burde snakke mer. Jeg har videre lært mye om å forholde seg diplomatisk så mye som mulig, og benytte den sokratiske diskusjonsmetoden for å holde konflikter så objektive som mulig. 

\paragraph{Ingelin}
Eksperter i team har vært et tidkrevende og mentalt krevende fag. Jeg 




\paragraph{Anna}
Det er første gang som student at jeg har jobbet i en gruppe, det var ingen tvil på at det skulle være lærerik og det har vært det.Jeg har lært først og fremst at å være flinkere å jobbe i en gruppe betyr ikke å bytte sin personlighet eller å følge en unaturlig oppskrift for å være den perfekt gruppemedlemmen. Det handler mer om hvordan mann må respekterer og aksepterer forskjellige trekk av hverandres personlighet og se hva disse forskjell kan gi mer enn kan betrakte. Jeg har lært å gi tilbakemeldinger og det ble enklere enn det jeg hadde trodde. Jeg har også lært å få tilbakemeldinger, disse har fortsatt litt ubehagelig men jeg har en mye mer avslappende forholdet til kritikk enn i begynnelse. Min personlig mål for faget var å gå ut av min konfortsone og gå dyp inni andre fagfeltet : jeg følte at jeg klarte det veldig bra og satt meg inn i både biologi og Roversteknologi som jeg kunne nesten ingenting fra før. Jeg er fornøyd med min personligutvikling og gruppensonnsats og synes at EIT har vært en veldig positiv opplevelse. 