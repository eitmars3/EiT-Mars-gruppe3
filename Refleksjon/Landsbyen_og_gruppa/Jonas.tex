Jonas Sandøy Misund
Født i Oslo, 1992.
Oppvokst i Molde fra 1998.

Studerer fysikk og matematikk med spesialisering innen teknisk fysikk.
Fag som er relevante for EiT - "En biologisk skattejakt på planeten Mars" er for eksempel optikk, kjerne- og strålingsfysikk, grunnkurs i kjemi.

Studiene legger ikke vekt på arbeid i grupper utenom lab hvor man som regel er to og to.
Bakgrunner som er relatert til gruppearbeid begrenser seg derfor til frivillige verv som linjeforeninger, Studentersamfundet og UKA.
Har fått noe kursing i ledelse, og er jevnlig en del av prosesser hvor fokus på gruppearbeid er relevant.

Tanker om seg selv i begynnelsen av prosjektet:
"Jeg er klar over at jeg er en person som snakker mye, spesielt i løpet av oppstarten av faglige grupper, eller grupper som skal jobbe med en bestemt problemstilling.
Derfor prøver jeg å passe på at lar alle de andre på gruppen slippe til og forsøker å oppfordre dem til å også komme med innspill.
Ellers var jeg kanskje en av de som har vært gjennom en del bevisstgjøringer rundt gruppearbeid, og hadde noen konkrete tanker rundt hvordan man burde organisere dette.
Når det kommer til det faglige hadde jeg ikke så mye mer å bidra med enn andre i gruppen, da omtrent ingen av oss har et studie som er rettet mot biologi.

Jeg prøver å alltid ha en positiv innstilling til gruppearbeid.
Noe av det kommer enkelt siden jeg generelt synes gruppearbeid er moro.
I tillegg må man gjerne jobbe litt med å holde interessen oppe hvis ting ikke blir helt som man hadde tenkt.
EiT er et fag jeg hadde hørt litt negativt om, men også fått potivitve inntrykk fra gjennom at noen jeg kjenner har vært læringsassisstent.
Dette er ikke noe jeg har snakket med personen om, men assosiasjonen i seg selv er positiv.
Ellers hadde jeg ikke mange formeninger om faget, men var bestemt på at det er bra å lære mer om hvordan man kan jobbe godt i gruppe med andre mennesker!

Forventningene til gruppersamarbeidet var ganske normale vil jeg påstå.
Jeg ønsket at alle skulle gjøre en god jobb, men var også klar over at noen kanskje syntes EiT var unødvenig, så prøvde tidlig å gi uttrykk for at man kan lære en del gjennom slike fag.
Dette var ikke ting jeg nødvendigvis sa kun til gruppa, men også til resten av landsbyen blant annet gjennom å svare på spørsmål i plenum og delta i stedet for å være passiv."