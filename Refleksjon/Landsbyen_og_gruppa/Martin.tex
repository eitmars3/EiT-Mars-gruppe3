\subsubsection{Martin Nordal}

\paragraph{Personalia}
Jeg er 24 år og kommer fra Oslo.
Interessen for programmering og skripting fikk meg til datateknikk-studiet, etterhvert med retning kunstig intelligens.
Jeg har også interesse for romfart, noe som førte til at jeg søkte på denne EiT-landsbyen. Jeg har løpt maraton, og nyter å løpe og tenke. 

\paragraph{Forventninger til EiT}
Forhåndsinntrykket til EiT har til dels vært basert på beretninger fra eldre personer som har hatt faget tidligere.
Det gjentakende i deres beretninger har vært at faget inneholder for mye obligatoriske aktiviteter som ikke har vært oppfattet som nyttig. Det har blitt alt for liten tid til å arbeide med eventuelle prosjekter eller de endelige rapportene.
Gjennom studiene så langt har jeg selv vært med på mye gruppearbeid.
Det har stort sett vært trivelig og produktivt, men enkelte gruppesammensetninger, og kanskje spesielt de store gruppene, har ikke fungert så bra.
EiT kunne derfor være behjelpelig for å lære hvordan man i fellesskap skal takle menneskelige utfordringer i en gruppe.
