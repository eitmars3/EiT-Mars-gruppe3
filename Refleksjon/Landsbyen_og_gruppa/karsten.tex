\subsubsection{Karsten Olav Kjensmo}

\paragraph{Personalia}
25 år. Har vokst opp i bl. a. Danmark, Nigeria, Namibia og USA. Opprinnelig født i Bærum, men har har ikke gått på norsk skole eller bodd lenge i Norge før universitetstiden. Har gjennomført en bachelorgrad i informatikk, og går mastergrad i informatikk på linjen databaser og søk; hvor jeg skriver oppgave innen kunstig intelligens og webteknologi. Har derutover faglig interesse i kognitive arkitekturer og designteori knyttet til menneske-maskin interakjson. På fritiden sykler jeg på fjellet, driver med hobbybryting og knoter med datateknologi. 

\paragraph{Forventninger til EiT}
Ved fagets start var inntrykket godt. En god studiekamerat har tatt både faget og den spesifikke landsbyen før, og hadde bare gode ting og si. Det ble kommentert på den trivelige stemningen i landsbyen - noe som alle landsbyer ikke har - samt den faglige friheten knyttet til oppgaven og emnet. Det faglige trakk av flere grunner - jeg er amatør på biologifeltet, men mente ikke dette ville være et problem, da jeg trodde det ville være mange biologer på landsbyen og at jeg med datavitenskap ville kunne bringe en ny vinkel på deres felt, spesifikt med datavitenskapen knyttet til romfart og utforskningsroboter. Romfart har altid fasinert meg. Det sosiale var også en faktor, da det ikke er mangel på skrekkhistorier om EiT grupper (nok noe overdrevet eller selvpåført) som ikke hang sammen sosialt. Andre studiekamerater har kommentert på dette, og har mistrivdes stort, og dette var et element i avgjørelsen da jeg valgte å søke landsbyen for biologisk skattejakt på Mars. Ved fagets start var jeg altså innstillt på å lære fra andre fagfelter, samt teste meg selv i en sosial arena med et introspektivt fokus. 




