\subsubsection{Karsten Olav Kjensmo}

\paragraph{Personalia}
Jeg er 25 år, og vokste opp i blant annet. Danmark, Nigeria, Namibia og USA. Jeg er opprinnelig født i Bærum, men har har ikke gått på norsk skole eller bodd lenge i Norge før universitetstiden. Jeg har gjennomført en bachelorgrad i informatikk, og tar mastergrad på linjen databaser og søk. Jeg tar fag innen kunstig intelligens og webteknologi. Har faglig interesse i kognitive arkitekturer og designteori knyttet til menneske-maskin-interakjson. På fritiden sykler jeg på fjellet, driver med hobbybryting og knoter med datateknologi. 

\paragraph{Forventninger til EiT}
Før faget startet var inntrykket av EiT godt. En god studiekamerat har tatt både faget og den spesifikke landsbyen før, og han hadde bare gode ting og si. Det ble kommentert på den trivelige stemningen i landsbyen - noe ikke alle landsbyer har - samt den faglige friheten knyttet til oppgaven og emnet. Det faglige trakk av flere grunner  selvom jeg er amatør på biologifeltet. Dette trodde jeg ikke ville være et problem, da det antageligvis det ville være mange biologer på landsbyen.  Med min datavitenskap knyttet til romfart og utforskningsroboter kunne jeg bringe en ny vinkel på deres felt. Romfart har alltid fasinert meg. Det sosiale var også en faktor, da det ikke er mangel på skrekkhistorier om EiT grupper (nok noe overdrevet eller selvpåført) som ikke fungerte sosialt. Andre studiekamerater har kommentert på dette og det var et element i avgjørelsen da jeg valgte å søke landsbyen. Ved fagets start var jeg altså innstillt på å lære fra andre fagfelt, samt teste meg selv i en sosial arena med et introspektivt fokus. 
