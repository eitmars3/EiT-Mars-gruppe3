\subsection{Innsjekk, reflektering og utsjekk}

Ved begynnelsen og avslutningen av hver landsbydag holdt gruppen felles møte.
Ved starten av dagen var det innsjekk og arbeidsplanlegging.
Dette innebar en koselig prat hvor hvert medlem i gruppen fortalte litt om hva som hadde skjedd i deres privatliv siden sist gang.
Deretter gikk det videre til fastsettelse av arbeidsplanen for dagen.
Etter gjennomført landsbydag reflekterte gruppemedlemmene rundt dagens faglige og samarbeidsmessige forløp.
Her ble dagens hendelser stykket opp i situasjon, teori, refleksjon og aksjon. Situasjonen beskriver hva som skjedde, teorien belyser hendelsen fra faglitteraturen, reflekjsonen redegør for gruppens tanker rundt emnet, og akjsonen defineres ved handlingene tatt for å korrigere problemet.  
Gruppen har tatt utgangspunkt i denne oppstykkingen under skrivingen av prosessrapporten, men har ikke brukt denne terminologien eksplisitt. 
Dette ble først gjort individuelt for så å bli diskutert i felleskap.
Siste post for dagen var utsjekk, hvor hvert medlem rundet av dagen og snakket om privatliv.