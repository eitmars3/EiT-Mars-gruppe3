\section{Introduksjon til EiT}

Eksperter i team (EiT) er et emne alle studenter som tar mastergrad på NTNU må gjennomføre.
Som forklart i emnebeskrivelsen\cite{eitlaeringsmaal} av faget, er hensikten med EiT å være et yrkesforberedende fag hvor studenten får erfaring med å arbeide i en tverrfaglig gruppe.
Emnet er utviklet på bakgrunn av at det i arbeidslivet ofte skal samarbeides i tverrfaglige prosjekter, en setting de færreste studenter har erfaring med før EiT.
Innsikt i gruppedynamikk skal læres ved å reflektere over og diskutere oppgaver både innenfor et faglig prosjekt og en samarbeidsprosess.
Målet er at studentene skal bli bedre kjent med seg selv som en del av en gruppe og hvordan gruppen og studenten påvirker hverandre.
Studenten skal utvikle sine samarbeidende egenskaper, bli komfortabel i å kommunisere og lære hvordan å unngå samt håndtere konflikter.
Et like viktig mål med emnet er at studentene skal kunne formidle og anvende sin egen kunnskap og lære fra andres fagområder.
Læringsmålet innen samarbeid og læringsmålet innen tverrfaglig arbeid skal presenteres i to separate rapporter som vektes likt i det resulterende karakterresultatet.